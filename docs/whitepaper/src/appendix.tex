\subsection{A Complete Description of the Spectrum Protocol}\label{subsec:a-complete-description-of-the-spectrum-protocol}
The purpose of this section is to formally specify the code of the Spectrum protocol that each participant executes.
Each party $P$ is assigned a session ID, $\text{sid}$.
Party is connected to all global setups and functionalities\
with which it shares the same session ID\@.

\subsubsection{The Main Protocol}
\phantomsection
\textbf{Spectrum protocol} uses a number of functionalities, namely, $\mathcal{G}_{\text{Ledger}},
\mathcal{G}_{\text{ImpLClock}}$ and $\mathcal{F}^{\Delta}_{\text{N-MC}}$ which are described in detail in~\cite{cryptoeprint:2019/838}.

All protocol participants use the imperfect local clocks functionality $\mathcal{G}_{\text{ImpLClock}}$ to proceed at\
approximately the same speed with the upper bound $\Delta^{\text{clock}}$ on the drift between any\
two honest parties.
We also assume a diffusion network in which all messages sent by honest parties are guaranteed to be fetched by\
other protocol participants after a specific delay $\Delta^{\text{net}}$.
Additionally, the network guarantees that once a message has been fetched by an honest party, this message is\
fetched by any other honest party within a delay of at most $\Delta^{\text{net}}$.
We will use a broadcasting network for message diffusion described by the functionality $\mathcal{F}^{\Delta}_{\text{N-MC}}$.

The main protocol is as follows:\label{apndx:spectrum-protocol}
\begin{protocol}
    \caption{$\textsf{Spectrum}(P, \text{sid}; \mathcal{G}_{\text{Ledger}}, \mathcal{G}_{\text{ImpLClock}}, \mathcal{F}^{\Delta}_{\text{N-MC}})$}
    \begin{algorithmic}
        \State $\textbf{Global variables:}$
        \begin{itemize}
            \item[\textbf{--}] Read-only: $R, l_{\text{VRF}}, K_{\text{f}}, K_{\text{g}}, \mathbf{S}^{\text{set}} = \{S_k^{\text{id}}\}_{k=1}^K, \mathbf{f}_{\text{lead}} = \{f^{\text{lead}}_k\}_{k=1}^K, \mathbf{f}_{\text{cons}} = \{f^{\text{cons}}_k\}_{k=1}^K$.
            \item[] \lstinline|// Hereinafter, the indices and values of epochs and slots are|
            \item[] \lstinline|// interchangeable, i.e. e_n = n, s_j = j.  Also, the id of the|
            \item[]  \lstinline|// k-th connected system S_k usualy is simply denoted by k.|
            \item[\textbf{--}] Read-write: $v^{\text{vrf}}_P, v^{\text{kes}}_P, sl_j, e_n, \mathbf{S}_P^{\text{set}}, \mathbf{T}^{\text{cons}},  \mathbf{T}^{\text{lead}}, \mathbf{T}^{\text{sync}},$
            \item[] $\textsf{localTime}, \textsf{lastTick}, \textsf{EpochUpdate(.)}, \textsf{state}_{j}, \textsf{buffer}, \textsf{syncBuffer}, \textsf{isSync},$
            \item[] $\textsf{fetchCompleted}, \textsf{futureChains}$.
            \item[]  \lstinline|// Bold font upper denotes the sets corresponding to the set|
            \item[]  \lstinline|// of external systems (with id-s in the S^set) to which the|
            \item[]  \lstinline|// participant P is connected.|
        \end{itemize}

        \State $\textbf{Interacting with the main Ledger:}$
        Upon receiving a ledger-specific input $I$ verify first that all resources are available.
        If not all resources are available, then ignore the input, otherwise execute one of the following\
        steps depending on the input $I$:
        \begin{itemize}
            \item[\textbf{--}] \textbf{If} $I = (\textsf{\scriptsize{SUBMIT}}, \text{sid}, \textsf{tx}):$
            \item[] set $\textsf{buffer} \leftarrow \textsf{buffer} || \textsf{tx}$;
            \item [] send $(\textsf{\scriptsize{MULTICAST}}, \text{sid}, \textsf{tx})$ to $\mathcal{F}^{\Delta}_{\text{N-MC}}$.
            \item[\textbf{--}] \textbf{If} $I = (\textsf{\scriptsize{MAINTAIN-LEDGER}}, P, \text{sid}):$
            \item[] invoke the protocol \hyperref[apndx:main-ledger-protocol]{$\textsf{LedgerMaintenance}$}$(P, R, \text{sid}, \mathcal{C}_{\text{loc}})$;
            \item[] if $\textsf{LedgerMaintenance}$ halts then halt the $\textsf{Spectrum}$ protocol execution and ignore all future inputs.
            \item[\textbf{--}] \textbf{If} $I = (\textsf{\scriptsize{READ}}, \text{sid}):$
            \item[] return actual local chain $\textsf{state}$;
            \item[\textbf{--}] \textbf{If} $I = (\textsf{\scriptsize{EXPORT-TIME}}, \text{sid}):$
            \item[] if $\textsf{isSync} = \textsf{false}$ then return $\textsf{false}$ to party $P$;
            \item[] Otherwise call \hyperref[apndx:update-time-protocol]{$\textsf{UpdateTime}$}$(P, R)$ and do:
            \begin{enumerate}
                \item Set the highest epoch value $e_n \leftarrow \textsf{EpochUpdate(.)}$.
                \item Return $(\textsf{localTime}, e_n)$ to the caller.
            \end{enumerate}
        \end{itemize}

        \State $\textbf{Handling calls to the shared setup:}$
        \begin{itemize}
            \item[\textbf{--}] \textbf{If} $I = (\textsf{\scriptsize{CLOCK-GET}}, \text{sid}_\text{C}):$ forward it\
            to $\mathcal{G}_{\text{ImpLClock}}$ and return its response.
            \item[\textbf{--}] \textbf{If} $I = (\textsf{\scriptsize{CLOCK-UPDATE}}, \text{sid}_\text{C}):$ record that a clock-update was received in the current round.
            If the party is registered to all its setups,  then do nothing further.
            Otherwise,  do the following operations before concluding this round:
            \begin{enumerate}
                \item If this instance is currently time-aware but otherwise stalled or offline, then set $\textsf{localTime} \leftarrow \textsf{UpdateTime}(P, R)$\
                and update the KES signing key using $\mathcal{F}_{\text{KES}}$.
                If the party has passed a synchronization slot, then set $\textsf{isSync}  \leftarrow  \textsf{false}$.
                \item If this instance is only stalled but $\textsf{isSync}  =  \textsf{true}$, then additionally\
                fetch actual chains, extract all new synchronization beacons from the fetched chains,\
                record their arrival times and set $\textsf{fetchCompleted}  \leftarrow  \textsf{true}$.
                Any unfinished interruptible execution of this round is marked as completed.
                \item Forward $(\textsf{\scriptsize{CLOCK-UPDATE}}, \text{sid}_\text{C})$ to $\mathcal{G}_{\text{ImpLClock}}$ to finally conclude the round.
            \end{enumerate}
            \item[\textbf{--}] \textbf{If} $I = (\textsf{\scriptsize{EVAL}}, x):$ forward $x$\
            to the $\mathcal{H}$ and output the response.
        \end{itemize}


    \end{algorithmic}\label{alg:spectrum-protocol}
\end{protocol}

\subsubsection{Fetching information, stake distribution and time update}
\phantomsection
\textbf{Time Update}.\label{apndx:update-time-protocol}
\begin{protocol}
    \caption{$\textsf{UpdateTime}(P, R)$}
    \begin{algorithmic}[1]
        \noindent
        \lstinline|//NB: Only executed if time-aware.|
        \State Send $(\textsf{\scriptsize{CLOCK-GET}}, \text{sid}_\text{C})$ to $\mathcal{G}_{\text{ImpLClock}}$\
        and parse $\textsf{tick}$ from the response.
        \If {$\textsf{lastTick} \neq \textsf{tick}$}
            \State Set $\textsf{lastTick} \leftarrow \textsf{tick}$.
            \State Set $\textsf{localTime} \leftarrow \textsf{localTime} + 1$.
            \State Set $\textsf{fetchCompleted} \leftarrow \textsf{false}$.
        \EndIf
        \State Set $\textsf{ep} \leftarrow \lceil \textsf{localTime}/\\R \rceil \leftarrow \textsf{false}$.
        \State Set $\textsf{sl} \leftarrow \textsf{localTime}$.

    \end{algorithmic}\label{alg:update-time-protocol}
\end{protocol}

\bigbreak
\bigbreak
\noindent
\phantomsection
\textbf{Synchronization Procedure}.\label{apndx:sync-proc}
\begin{protocol}
    \caption{$\textsf{SyncProc}(P, R)$}
    \begin{algorithmic}[1]

        todo
    \end{algorithmic}\label{alg:sync-proc}
\end{protocol}

\bigbreak
\bigbreak
\noindent
\phantomsection
\textbf{Fetching information}.\label{apndx:fetch-info-protocol}
\begin{protocol}
    \caption{$\textsf{FetchInformation}(P, \text{sid})$}
    \begin{algorithmic}[1]

        todo
    \end{algorithmic}\label{alg:fetch-info-protocol}
\end{protocol}

\bigbreak
\bigbreak
\noindent
\phantomsection
\textbf{Reading blockchain state}.\label{apndx:read-state-protocol}
\begin{protocol}
    \caption{$\textsf{ReadState}(P, \text{sid}, R, \mathcal{C}_{\text{loc}})$}
    \begin{algorithmic}[1]

        todo
    \end{algorithmic}\label{alg:read-state-protocol}
\end{protocol}


\subsubsection{Validity Checks}
\phantomsection
\textbf{Block validation}.\label{apndx:prepare-block-validation}
A core procedure to validate an incoming blocks.
Block validation implies a procedure for preparing the necessary constants to check the validity of the party $P'$ to issue the block.
The preparation algorithm is described below:
\begin{algo}
    \caption{$\textsf{PrepareForBlockValidation}(sl, R, v_{P'}^{\text{vrf}}, C_{\text{loc}})$}
    \begin{algorithmic}[1]
        \noindent
        \lstinline|// Parse and calculate all necessary values for block validation.|
        \State Calculate $e_n$ for the given $sl$.

        \noindent
        \lstinline|// Main ledger state_m is calculated according to the last block|
        \noindent
        \lstinline|// produced up to m-th slot.|
        \State Parse $\textsf{state}_{(n - 4)\cdot R},
        \textsf{state}_{(n - 2)\cdot R}$ from $C_{\text{loc}}$.
        \State Parse verified participants table $V{\text{ver}}$ from $\textsf{state}_{(n - 4)\cdot R}$.
        \State Parse an associated committee id $k$ for the given $v_{P'}^{\text{vrf}}$ from $V{\text{ver}}$.
        \noindent
        \lstinline|// Set epoch randomness:|
        \State Set ${\eta_{n-2} \leftarrow \mathcal{F}_{\text{LB}}(e_{n-2}, C_{\text{loc}})}$. \lstinline|// For the consensus lottery.|
        \State Set ${\eta_{n} \leftarrow \mathcal{F}_{\text{LB}}(e_{n}, C_{\text{loc}})}$. \lstinline|// For the leader and sync lotteries.|

        \noindent
        \lstinline|// Stakeholders distribution used for the leader lottery:|
        \State Parse $k$-th committee stakeholders distribution $S_k^{\text{cons}, n - 2}$ from the $\textsf{state}_{(n - 2)\cdot R}$.

        \noindent
        \lstinline|// Stakeholders distribution used for the consensus group lottery:|
        \State Parse verified and equipped with $\mathcal{F}^k_{\text{ConnSys}}$ functionality stakeholders distribution $S_k^{\text{ver}, {n - 4}}$\
        from $\textsf{state}_{(n - 4)\cdot R}$.

        \noindent
        \lstinline|// Set lotteries thresholds:|
        \State Set participant's consensus group lottery threshold for $k$-th committee as $T_{P', n-2}^{\text{cons}, k}$ calculated using $S_k^{\text{ver}, {n - 4}}$.
        \State Set participant's leader lottery threshold as $T_{P', n}^{\text{lead}, k}$ calculated using $S_k^{\text{cons}, {n - 2}}$.
        \State Set participant's synchronization lottery threshold as $T_{P', n}^{\text{sync}}$ calculated using $S_k^{\text{cons}, {n - 2}}$.
        \State \Return ${(e_n, V_{\text{ver}}, \eta_{n}, \eta_{n - 2}, T_{P', n-2}^{\text{cons}, k}, T_{P', n}^{\text{lead}, k}, T_{P', n}^{\text{sync}})}$
    \end{algorithmic}\label{alg:prepare-block-validation}
\end{algo}

\bigbreak
\bigbreak
\noindent
\phantomsection
\textbf{Main Block Validation Protocol}\label{apndx:block-validation-protocol} is as follows:
\begin{protocol}
    \caption{$\textsf{IsValidBlock}(P, R, B, C_{\text{loc}})$}
    \begin{algorithmic}[1]
        \noindent
        \lstinline|// All indexes except the epoch index are omitted.|
        \State Parse $B$ as ${(h, sl, \textsf{state}, v^{\text{vrf}}, \pi^{\text{sl}}, \sigma_{\text{KES}})}$. \lstinline|// value h is a block body hash.|

        \noindent
        \lstinline|// Prepare constants:|
        \State Set $\textsf{preparation\_out} \leftarrow \textsf{PrepareForBlockValidation}(sl, R, v^{\text{vrf}}, C_{\text{loc}})$.
        \State Set $(e_n, V_{\text{ver}}, \eta_{n}, \eta_{n - 2}, T^{\text{cons}}, T^{\text{lead}}, T^{\text{sync}}) \leftarrow \textsf{preparation\_out}$.

        \noindent
        \lstinline|// Check consensus membership:|
        \State Parse ${\pi^{\text{e}}}$ from $\textsf{state}$.
        \State Set $\textsf{valid\_epoch\_proof} \leftarrow \mathcal{F}_{\text{VRF}}\textsf{.verify(}\
        v^{\text{vrf}}, \mathcal{H}(\eta_{n - 2} || e_n),\pi^{\text{e}} \textsf{)}$.
        \State Set ${r^{\text{e}} \leftarrow \mathcal{F}_{\text{VRF}}\textsf{.extract\_random\_number(}\pi^{\text{e}}\textsf{)}}$.
        \State Set ${y^{\text{cons}} \leftarrow {\mathcal{H}(r^{\text{e}} || \textsf{CONS})}}$.
        \State Set ${\textsf{valid\_member} \leftarrow (v^{\text{vrf}} \in V_{\text{ver}}) \wedge (y^{\text{cons}} < T^{\text{cons}}) \wedge \textsf{valid\_epoch\_proof}}$.

        \noindent
        \lstinline|// Check the leadership:|
        \State Set $\textsf{valid\_slot\_proof} \leftarrow \mathcal{F}_{\text{VRF}}\textsf{.verify(}\
        v^{\text{vrf}}, \mathcal{H}(\eta_n || sl),\pi^{\text{sl}} \textsf{)}$.
        \State Set ${r^{\text{sl}} \leftarrow \mathcal{F}_{\text{VRF}}\textsf{.extract\_random\_number(}\pi^{\text{sl}}\textsf{)}}$.
        \State Set ${y^{\text{lead}} \leftarrow {\mathcal{H}(r^{\text{sl}} || \textsf{LEAD})}}$.
        \State Set ${\textsf{valid\_leader} \leftarrow (y^{\text{lead}} < T^{\text{lead}}) \wedge \textsf{valid\_slot\_proof}}$.

        \noindent
        \lstinline|// Check KES signature:|
        \State Parse $v^{\text{kes}}$ from $V_{\text{ver}}$.
        \State Parse $b^{\text{set}}$ from $\textsf{state}$.
        \State Set $\textsf{valid\_signature} \leftarrow \mathcal{F}_{\text{KES}}\textsf{.verify(}\
        (h, \textsf{state}, sl, b^{\text{set}}, \pi^{\text{sl}}), \sigma_{\text{KES}}, v^{\text{kes}}) \textsf{)}$.

        \noindent
        \lstinline|// Check synchronization beacons:|
        \If{${\exists b^{\text{sync}} \in B: sl > (e_n - 1) \cdot R + 2 \cdot R \// 3}$}
            \State Set ${\textsf{valid\_sync} \leftarrow \textsf{false}}$.
        \ElsIf
                {$\exists b^{\text{sync}} \in B: (b^{\text{sync}}\textsf{.get(}sl\textsf{)} > sl)
            \vee (b^{\text{sync}}\textsf{.get(}sl\textsf{)} \notin [(e_n - 1) \cdot R + 1, e_n \cdot R])$}
            \State Set ${\textsf{valid\_sync} \leftarrow \textsf{false}}$.
        \EndIf
        \For {each $b^{\text{sync}} \in B$}
            \State Parse $b^{\text{sync}}$ as $(v^{\text{vrf}'}, sl', \pi^{\text{sl}'})$.
            \If
            {$\mathcal{C}_{\text{loc}}$ contains more than one beacon with $(v^{\text{vrf}'}, sl', .)$}
                \State Set ${\textsf{valid\_sync} \leftarrow \textsf{false}}$.
            \EndIf
            \State Set $\textsf{valid\_slot\_proof} \leftarrow \mathcal{F}_{\text{VRF}}\textsf{.verify(}\
            v^{\text{vrf}}, \mathcal{H}(\eta_n || sl'),\pi^{\text{sl}'} \textsf{)}$.
            \State Set ${r^{\text{sl}'} \leftarrow \mathcal{F}_{\text{VRF}}\textsf{.extract\_random\_number(}\pi^{\text{sl}'}\textsf{)}}$.
            \State Set ${y^{\text{sync}} \leftarrow {\mathcal{H}(r^{\text{sl}'} || \textsf{SYNC})}}$.
            \State Set ${\textsf{valid\_sync} \leftarrow (y^{\text{sync}} < T^{\text{sync}}) \wedge \textsf{valid\_slot\_proof}}$.
        \EndFor
        \If {($\textsf{valid\_parent} \wedge \textsf{valid\_member} \wedge \textsf{valid\_proof} \wedge
        \textsf{valid\_leader} \wedge \textsf{valid\_signature} \wedge \textsf{valid\_sync})$}
            \State \Return ${\textsf{false}}$
        \EndIf
    \end{algorithmic}\label{alg:block-validation-protocol}
\end{protocol}

\bigbreak
\bigbreak
\noindent
\phantomsection
\textbf{Chain validation}.\label{apndx:chain-validation-protocol}
A core procedure is to distinguish valid chains from the invalid:
\begin{protocol}
    \caption{$\textsf{IsValidChain}(P, R, \mathcal{C}_{\text{loc}})$}
    \begin{algorithmic}[1]
        \If {${\exists B \in \mathcal{C}: B\textsf{.get(}sl\textsf{)} > sl^{\text{loc}}}$}
            \State \Return ${\textsf{false}}$
        \EndIf
        \For {each $e_j \in \mathcal{C}_{\text{loc}}$}

            \For {each block $B \in \mathcal{C}_{\text{loc}}\ | \ B\textsf{.get(}sl\textsf{)} \in e_j$}

                \par\hskip\algorithmicindent
                \lstinline|// Check parent:|
                \State Set $\textsf{valid\_parent} \leftarrow (\mathcal{H}(B^{-1}) = h) \wedge (B^{-1}\textsf{.get(}sl\textsf{)} < sl)$,\
                \par\hskip\algorithmicindent
                where $B^{-1}$ is the last block before $B$.
                \State Set $\textsf{valid\_block} \leftarrow \textsf{IsValidBlock}(P, R, B, \mathcal{C}_{\text{loc}})$.
                \If {$(\textsf{valid\_block} \wedge \textsf{valid\_parent})$}
                    \State \Return ${\textsf{false}}$
                \EndIf
            \EndFor
        \EndFor
        \State \Return ${\textsf{true}}$
    \end{algorithmic}\label{alg:chain-validation-protocol}
\end{protocol}

\bigbreak
\bigbreak
\noindent
\phantomsection
\textbf{The synchronisation beacon validity}.\label{apndx:sync-beacon-validity-protocol}
Beacons validity is related to chain validity as one has to verify validity of leadership:
\begin{protocol}
    \caption{$\textsf{ValidBeacon}(P, R, b, \mathcal{C}_{\text{loc}})$}
    \begin{algorithmic}[1]
        \State Parse synchronization beacon $b$ as $(v_{P'}^{\text{vrf}}, sl, \pi^{\text{sl}})$.
        \State Calculate $e_n$ for the given $sl$.
        \If {$\nexists B \in \mathcal{C}_{\text{loc}} | B\textsf{.get(sl)} \in e_n$}
            \State \Return \textsf{false}
        \EndIf

        \noindent
        \lstinline|// Check synchronization lottery results for patry P':|
        \State Set ${\eta_{n} \leftarrow \mathcal{F}_{\text{LB}}(e_{n}, C_{\text{loc}})}$.
        \State Parse $\textsf{state}_{(n - 2)\cdot R}$ from $C_{\text{loc}}$.
        \State Parse all committees stakeholders distribution $S^{\text{cons}, n - 2}$ from the $\textsf{state}_{(n - 2)\cdot R}$.
        \State Set participant's synchronization lottery threshold as $T_{P'}^{\text{sync}}$ calculated using $S^{\text{cons}, {n - 2}}$.
        \State Set $\textsf{valid\_slot\_proof} \leftarrow \mathcal{F}_{\text{VRF}}\textsf{.verify(}\
        v_{P'}^{\text{vrf}}, \mathcal{H}(\eta_n || sl),\pi^{\text{sl}} \textsf{)}$.
        \State Set ${r^{\text{sl}} \leftarrow \mathcal{F}_{\text{VRF}}\textsf{.extract\_random\_number(}\pi^{\text{sl}}\textsf{)}}$.
        \State Set ${y^{\text{sync}} \leftarrow {\mathcal{H}(r^{\text{sl}} || \textsf{SYNC})}}$.
        \State Set ${\textsf{valid\_sync} \leftarrow (y^{\text{sync}} < T^{\text{sync}}) \wedge \textsf{valid\_slot\_proof}}$.
        \State \Return \textsf{valid\_sync}

    \end{algorithmic}\label{alg:sync-beacon-validity-protocol}
\end{protocol}


\subsubsection{Chain Selection Rules}
Chain selection consists of two steps: filtering out valid chains, and second compare them\
using the Genesis rule~\cite{Badertscher2018}.

\bigbreak
\noindent
\phantomsection
\textbf{Maximum Stake Rule}.\label{apndx:max-stake-rule}
Rule to resolve conflicts that arise after applying the Genesis rule.
\begin{algo}
    \caption{${\textsf{maxStakeChain}(\mathcal{C}_i}, \mathcal{C}_{i'})$}
    \begin{algorithmic}[1]
        \If {$\sum \{B_k\textsf{.get(}s\textsf{)}, \forall B_k \in  \mathcal{C}_i\} > \sum \{B_k\textsf{.get(}s\textsf{)}, \forall B_k \in  \mathcal{C}_{i'}\}$}

            \noindent
            \lstinline|// Used above value s is the stake of the leader|

            \noindent
            \lstinline|// who produced the block B_k.|
            \State Set ${\mathcal{C}_{\max} \leftarrow \mathcal{C}_{i}}$.
        \Else
            \noindent
            \lstinline|  // It is assumed that the input chains are the same size.|
            \State Set ${\mathcal{C}_{\max} \leftarrow \mathcal{C}_{i'}}$.
        \EndIf
        \State \Return ${\mathcal{C}_{\max}}$
    \end{algorithmic}\label{alg:max-stake-rule}
\end{algo}

\bigbreak
\bigbreak
\noindent
\phantomsection
\textbf{Maximum Valid Rule}.\label{apndx:max-valid-rule}
The Genesis chain selection rule:
\begin{algo}
    \caption{${\textsf{maxValidChain}(\mathcal{C}_{\text{loc}}, \{\mathcal{C}_i\}_{i=1}^N, K_{\text{f}}, K_{\text{g}})}$}
    \begin{algorithmic}[1]
        \noindent
        \lstinline|// C_loc is the local chain of the validator.|
        \noindent
        \lstinline|// Set local chain as initially maximum valid chain:|
        \State Set ${\mathcal{C}_{\max} \leftarrow \mathcal{C}_{\text{loc}}}$.
        \For {${\mathcal{C}_i \in \{\mathcal{C}_i\}_{i=1}^N}$}
            \If {$\mathcal{C}_i$ forks from $\mathcal{C}_{\max}$ at most $K_{\text{f}}$ blocks}
                \If {$|\mathcal{C}_i| > |\mathcal{C}_{\max}|$}
                    \State Set ${\mathcal{C}_{\max} \leftarrow \mathcal{C}_{i}}$.
                \ElsIf
                        {$|\mathcal{C}_i| == |\mathcal{C}_{\max}|$}
                    \State Set ${\mathcal{C}_{\max} \leftarrow \textsf{maxStakeChain}(\mathcal{C}_{i}, \mathcal{C}_{\max})}$.

                \EndIf
            \Else

                \State Let ${j  \leftarrow \max \{j' \geq 0 \ |\  \mathcal{C}_{\max} \text{ and } \mathcal{C}_{i} \
                \text{have the same block in } sl_{j'}\}}$.
                \If {$|\mathcal{C}_i[j: j + K_{\text{g}}]| > |\mathcal{C}_{\max}[j: j + K_{\text{g}}]|$}
                    \State Set ${\mathcal{C}_{\max} \leftarrow \mathcal{C}_{i}}$.
                \ElsIf
                        {$|\mathcal{C}_i| = |\mathcal{C}_{\max}|$}
                    \State Set ${\mathcal{C}_{\max} \leftarrow \textsf{maxStakeChain}(\mathcal{C}_{i}, \mathcal{C}_{\max})}$.
                \EndIf
            \EndIf

        \EndFor
        \State \Return ${\mathcal{C}_{\max}}$
    \end{algorithmic}\label{alg:max-valid-rule}
\end{algo}

\bigbreak
\bigbreak
\noindent
\phantomsection
\textbf{Chain Selection Protocol}.\label{apndx:max-selection-protocol}
The main chain selection protocol is as follows:
\begin{protocol}
    \caption{$\textsf{SelectChain}(P, \text{sid}, \mathcal{C}_{\text{loc}}, R, K_{\text{f}}, K_{\text{g}}, \mathcal{N}_0,$
        $\mathbf{S}_k^{\text{set}} = \{S_k^{\text{id}}\}_{k=1}^K, \mathbf{f}_{k, \text{lead}}^{\text{set}} =  \{f^{\text{lead}}_k\}_{k=1}^K,
    \mathbf{f}_{k, \text{cons}}^{\text{set}} = \{f^{\text{cons}}_k\}_{k=1}^K)$}
    \begin{algorithmic}[1]
        \State Initialize $\mathcal{N}_{\text{valid}} \leftarrow \emptyset$.

        \noindent
        \lstinline|// Filter all valid chains:|
        \For {$\mathcal{C} \in \mathcal{N}_0$}
            \State Set $\textsf{is\_valid\_chain} \leftarrow \textsf{IsValidChain}(P, \text{sid}, \mathcal{C}, R, \{S_k^{\text{id}}\}_{k=1}^K, \{f^{\text{lead}}_k\}_{k=1}^K,$
            \par\hskip\algorithmicindent
            $\{f^{\text{cons}}_k\}_{k=1}^K)$.
            \If {$\textsf{is\_valid\_chain} = \textsf{true}$}
                \State Update $\mathcal{N}_{\text{valid}} \leftarrow \mathcal{N}_{\text{valid}} \cup \mathcal{C}$.
            \EndIf
        \EndFor

        \noindent
        \lstinline|// Set local chain as maximum valid chain:|
        \State Set $\mathcal{C}_{\text{loc}} \leftarrow \textsf{maxValidChain}(\mathcal{C}_{\text{loc}}, \mathcal{N}_0, K_{\text{f}}, K_{\text{g}})$.

    \end{algorithmic}\label{alg:max-selection-protocol}
\end{protocol}


\subsubsection{Ledger Maintenance}
When a protocol is executed, every party $P$ performs different actions depending on its role and the current local time.
The main logic with all necessary actions are included into the main $\textsf{LedgerMaintenance}$ procedure.
At different points in time, participants perform auxiliary protocols, which we will describe below.

\bigbreak
\noindent
\phantomsection
\textbf{Evaluation Protocol}.\label{apndx:eval-protocol}
In normal protocol execution, each participant performs the following procedure:
\begin{protocol}
    \caption{$\textsf{EvaluationProcedure}(P, \text{sid}, R, sl, \textsf{buffer}, \textsf{syncBuffer}, \mathcal{C}_{\text{loc}}, \{S_k^{\text{id}}\}_{k=1}^K,$
    $\{f^{\text{lead}}_k\}_{k=1}^K, \{f^{\text{cons}}_k\}_{k=1}^K)$}
    \begin{algorithmic}[1]

        \noindent
        \lstinline|// Leader lottery:|
        \State Set actual $T^{\text{lead}}$ from the \textsf{context}.
        \State Evaluate $\mathcal{F}_{\text{VRF}}$ with input $x = \eta_n|| sl$ to get slot proof $\pi^{\text{sl}}$.
        \State Set ${r^{\text{sl}} \leftarrow \mathcal{F}_{\text{VRF}}\textsf{.extract\_random\_number(}\pi^{\text{sl}}\textsf{)}}$
        \State Set ${y^{\text{lead}} \leftarrow {\mathcal{H}(r^{\text{sl}} || \textsf{LEAD})}}$
        \State Set ${\textsf{valid\_leader} \leftarrow y^{\text{lead}} < T^{\text{lead}}}$.

        \noindent
        \lstinline|// Synchronization lottery:|
        \State Set ${y^{\text{sync}} \leftarrow {\mathcal{H}(r^{\text{sl}} || \textsf{SYNC})}}$.
        \State Set ${\textsf{valid\_sync} \leftarrow y^{\text{sync}} < T^{\text{sync}}}$.

        \If {$\textsf{valid\_leader}$}
            \State Set $\textsf{state}$ extracted from $\textsf{buffer}$
            \State Set $h \leftarrow \mathcal{H}(\mathcal{C}_{\text{loc}}\textsf{.head()})$
            \If {$sl \in [R \cdot n, R \cdot n + 2 /\\ 3 \cdot R]$}
                \State {Set valid synchronization beacon set as:}.
                \State {${b^{\text{set}} \leftarrow \{b' \in \textsf{syncBuffer} | \textsf{validBeacon}(b', \mathcal{C}_{\text{loc}}) = \textsf{true}\}}$}
                \For
                {$\all b$ in $b^{\text{set}}$}
                    \State Set $sl^* \leftarrow b.\textsf{get(} sl \textsf{)})$
                    \State Set $v^{\text{vrf}}^* \leftarrow b.\textsf{get(} v^{\text{vrf}} \textsf{)}$
                    \If {$(sl^* > sl) \vee (sl^* \geq (n - 1) \cdot R) \vee (\exists b' \in \mathcal{C}_{\text{loc}} | (b'.\textsf{get(} v^{\text{vrf}}\
                    \textsf{)}) = v^{\text{vrf}}^* \wedge b'.\textsf{get(} sl \textsf{)} = sl^*))$}
                        \State Remove $b$ from $b^{\text{set}}$.
                    \EndIf

                \EndFor
            \EndIf
            \State Set $\sigma_{\text{KES}} \leftarrow \mathcal{F}_{\text{KES}}\textsf{.sign(} h || \textsf{state} || b^{\text{set}} || sl || \pi^{\text{sl}} \textsf{)}$.
            \State Create new block $B$ with $\sigma_{\text{KES}}$ signature included.
            \State Update $\mathcal{C}_{\text{loc}}$ with new block $B$ and broadcast the update to known peers.
        \EndIf
        \If {$\textsf{valid\_sync}$}
            \If {$sl \in [R \cdot n, R \cdot n + 1 /\\ 6 \cdot R]$}
                \State Set $b \leftarrow (v^{\text{vrf}}, sl, \pi^{\text{sl}})$.
                \State Broadcast $b$ to known peers.

            \EndIf

        \EndIf

    \end{algorithmic}\label{alg:eval-protocol}
\end{protocol}


\bigbreak
\bigbreak
\noindent
\phantomsection
\textbf{Consensus Lottery Protocol}.\label{apndx:consensus-lottery-protocol}
When moving between epochs, a new consensus group must be selected.
To do this, each participant performs the following protocol:
\begin{protocol}
    \caption{$\textsf{ConsensusLottery}(P, \text{sid}, e_n, \mathcal{C}_{\text{loc}}, K_{\text{f}}, K_{\text{g}}, \{S_k^{\text{id}}\}_{k=1}^K, \{f^{\text{cons}}_k\}_{k=1}^K)$}
    \begin{algorithmic}[1]
        \State Set ${\pi^{\text{e}}} \leftarrow \mathcal{F}_{\text{VRF}}\textsf{.eval(}\mathcal{H}(\eta_n || e_n) \textsf{)}$.
        \State Set ${r^{\text{e}} \leftarrow \mathcal{F}_{\text{VRF}}\textsf{.extract\_random\_number(}\pi^{\text{e}}\textsf{)}}$
        \State Parse $k$-th committee stakeholders distribution $S_k^{\text{cons}, n - 2}$ from the $\textsf{state}_{(n - 2)\cdot R}$
        \State Calculate consensus group lottery threshold for $k$-th committee as $T_n^{\text{cons}, k}$ calculated using $S_k^{\text{ver}, {n - 2}}$
        \State Set ${y^{\text{cons}} \leftarrow {\mathcal{H}(r^{\text{e}} || \textsf{CONS})}}$
        \State Set ${\textsf{is\_member} = (y^{\text{cons}} < T^{\text{cons}})}$.
        \If
            $\textsf{is\_member} = \textsf{true}$
            \State Broadcast message with consensus membership proof $\textsf{ConsLotteryResTx(}v^{\text{vrf}, \pi^{\text{e}}}\textsf{)}$.
        \EndIf

    \end{algorithmic}\label{alg:consensus-lottery-protocol}
\end{protocol}

\bigbreak
\bigbreak
\noindent
\phantomsection
\textbf{Main Ledger Maintenance Protocol}.\label{apndx:main-ledger-protocol}
\begin{protocol}
    \caption{$\textsf{LedgerMaintenance}(P, \text{sid}, \mathcal{C}_{\text{loc}}, R, K_{\text{f}}, K_{\text{g}}, \{S_k^{\text{id}}\}_{k=1}^K, \{f^{\text{lead}}_k\}_{k=1}^K, \{f^{\text{cons}}_k\}_{k=1}^K)$}
    \begin{algorithmic}[1]

        \noindent
        \lstinline|// Normal operation:|
        \State Call $\textsf{FetchInformation}(P)$ and denote the output by $(\{\mathcal{C}_m\}_{m=1}^M, \{\textsf{tx}_k\}_{i=k}^K)$.
        \State Add $\{\mathcal{C}_m\}_{m=1}^M$ into $\textsf{futureChains}$.
        \State Add $\{\textsf{tx}_k\}_{k=1}^K$ into $\textsf{buffer}$.
        \State Call $\textsf{UpdateTime}(P)$.

        \noindent
        \lstinline|// Process arrived synchronisation beacons:|
        \State Extract beacons $b^{\text{set}} \leftarrow \{b^{\text{sync}}_n\}_{n=1}^N$ contained in $\{\mathcal{C}_m\}_{m=1}^M$ and not yet contained in $\textsf{syncBuffer}$.
        \State Call $\textsf{ProcessBeacons}(P, \text{sid}, R, \{b^{\text{sync}}_n\}_{n=1}^N, \mathcal{C})$.

        \noindent
        \lstinline|// Filter chains:|
        \State Let $\mathcal{N}_0$ be the subsequence of $\textsf{futureChains}\ |\ \forall B \in C:\ B{\textsf{.get(} sl \textsf{)}} \leq \textsf{localTime}$.
        \For {$\mathcal{C} \in \mathcal{N}_0$}
            \State Remove $\mathcal{C}$ from $\textsf{futureChains}$.
        \EndFor
        \State Set $\mathcal{C}_{\text{loc}} \leftarrow \textsf{SelectChain}(P, \text{sid}, \mathcal{C}_{\text{loc}}, R, K_{\text{f}}, K_{\text{g}}, \mathcal{N}_0, \{S_k^{\text{id}}\}_{k=1}^K, \{f^{\text{lead}}_k\}_{k=1}^K, \{f^{\text{cons}}_k\}_{k=1}^K)$.

        \noindent
        \lstinline|// Perform actions according to the current local|
        \noindent
        \lstinline|// stage of the protocol:|
        \State Set $sl \leftarrow \textsf{localTime}$.
        \If {$sl < sl^{\text{work}}$}
            \State Call $\textsf{EvaluationProcedure}(P, \text{sid}, R, sl, \textsf{buffer}, \textsf{syncBuffer}, \mathcal{C}_{\text{loc}}, \{S_k^{\text{id}}\}_{k=1}^K,$
            \par\hskip\algorithmicindent
            $\{f^{\text{lead}}_k\}_{k=1}^K, \{f^{\text{cons}}_k\}_{k=1}^K)$.
            \State Set $sl^{\text{work}} \leftarrow sl$.
            \If {$sl \bmod R = 0$}
                \State Call $\textsf{UpdateStakeDistribution}(P, \text{sid}, R, \mathcal{C}_{\text{loc}}, \{S_k^{\text{id}}\}_{k=1}^K, \{f^{\text{cons}}_k\}_{k=1}^K)$.
                \State Calculate $e_n$ for the given $sl$.\lstinline|  // The value of e_n is equal to n.|
                \State Parse $\textsf{state}_{(n - 2)\cdot R}$ from $C_{\text{loc}}$.
                \For {every party's $P$ $k$-th connected system}
                    \State Call $\textsf{ConsensusLottery}(P, \text{sid}, e_n, \mathcal{C}_{\text{loc}}, K_{\text{f}}, K_{\text{g}}, \{S_k^{\text{id}}\}_{k=1}^K, \{f^{\text{cons}}_k\}_{k=1}^K)$.
                \EndFor
                \State Call $\textsf{SyncProc}(P, \text{sid}, R)$.
            \EndIf
        \EndIf

    \end{algorithmic}\label{alg:main-ledger-protocol}
\end{protocol}


\newpage

\subsection{List of Symbols}\label{subsec:list-of-symbols}
\textbf{Functionalities:}
\begin{itemize}
    \item[\fontencoding[\ding{113}] $\mathcal{H}$ -- ideal hash function (random oracle).
    \item[\ding{113}] $\mathcal{F}_{\text{VRF}}$ -- verifiable random function.
    \item[\ding{113}] $\mathcal{F}_{\text{KES}}$ -- key evolving digital signature scheme.
    \item[\ding{113}] $\mathcal{F}_{\text{LB}}$ -- leaky beacon.
    \item[\ding{113}] $\mathcal{F}_{\text{AggSig}}$ -- collective signature aggregation functionality.
    \item[\ding{113}] $\mathcal{F}_{\text{Init}}$ -- functionality providing the genesis block.
    \item[\ding{113}] $\mathcal{F}^{\Delta}_{\text{N-MC}}$ -- functionality providing the genesis block.
    \item[\ding{113}] $\mathcal{F}^k_{\text{ConnSys}}$ -- functionality to interact with $k$-th\
    connected distributed system $S_k$.
    \item[\ding{113}] $\mathcal{G}_{\text{ImpLClock}}$ -- imperfect local clock functionality.
    \item[\ding{113}] $\mathcal{G}_{\text{PerfLClock}}$ -- perfect local clock functionality.
    \item[\ding{113}] $\mathcal{G}_{\text{Ledger}}$ -- the ledger functionality.

\end{itemize}
\bigbreak

\noindent
\textbf{Main State Variables of The Spectrum protocol:}
\begin{legal}
    \item[\ding{113}] $sl_i \in \mathbb{N}$ -- the smallest discrete time unit used in the protocol.
    \item[\ding{113}] $e_n \in \mathbb{N}$ -- the largest discrete time unit used in the protocol.
    \item[\ding{113}] $R \in \mathbb{N}$ -- epoch boundaries, each epoch consists of $R$ slots.
    \item[\ding{113}] $K_{\text{f}} \in \mathbb{N}$ -- number of blocks to achieve finality in the $L^+$.
    \item[\ding{113}] $K_{\text{g}} \in \mathbb{N}$ -- number of blocks to consider chain growth (is used in chain selection).

\end{legal}
\bigbreak
\noindent
\textbf{Main State Variables of The Spectrum protocol participant:}
\begin{legal}
    \item[\ding{113}] $P$ -- Spectrum protocol participant (party).
    \item[\ding{113}] $PK_{\text{P}}$ -- Spectrum's public key of the party $P$.
    \item[\ding{113}] $PK^k_{\text{P}}$ -- public key of the $k$-th connected external system of the party $P$.
    \item[\ding{113}] $s_{\text{P}}$ -- stake value of the party $P$.
    \item[\ding{113}] $\eta_n$ -- random seed of the epoch $e_n$ (epoch randomness).
    \item[\ding{113}] $\pi^{\text{sl}, n}_{\text{P}, j}$ -- VRF slot proof produced by the party $P$\
    for slot $sl_j \in e_n$ (used in the leader lottery and in the synchronization lottery).
    \item[\ding{113}] $\pi^{\text{e}}_{\text{P}, n}$ -- VRF epoch proof produced by the party $P$ for epoch $e_n$\
    (used in the consensus group lottery).
    \item[\ding{113}] $v^{\text{vrf}}_{\text{P}}$ -- VRF public key of the party $P$.
    \item[\ding{113}] $v^{\text{kes}}_{\text{P}}$ -- KES scheme public of the party $P$.
    \item[\ding{113}] $T^{\text{cons}, k}_{\text{P}, n}$ -- consensus group lottery threshold (for related $V_k$ committee) calculated for\
    the party $P$ for epoch $n$.
    \item[\ding{113}] $T^{\text{lead}, k}_{\text{P}, n}$ -- leader lottery threshold (for related $V_k$ committee) calculated for\
    the party $P$ for epoch $n$.
    \item[\ding{113}] $T^{\text{sync}}_{\text{P}, n}$ -- synchronization beacon lottery threshold calculated for\
    the party $P$ for epoch $n$.
    \item[\ding{113}] $\textsf{isInit}$ -- a variable to keep track of whether initialization is complete.
    \item[\ding{113}] $\textsf{isSync}$ -- the party's stores synchronization status.
    \item[\ding{113}] $\textsf{buffer}$ -- the buffer of transactions.
    \item[\ding{113}] $\textsf{localTime}$ -- the party's current local slot.
    \item[\ding{113}] $\textsf{lastTick}$ -- the last tick received from $\mathcal{G}_{\text{PerfLClock}}$.
    \item[\ding{113}] $\textsf{syncBuffer}$ -- the buffer of the synchronization beacons.
    \item[\ding{113}] $\textsf{futureChains}$ -- a buffer to store chains that are not yet processed.
    \item[\ding{113}] $\textsf{EpochUpdate(·)}$ -- a function table to remember which clock adjustments have been done already.
    \item[\ding{113}] $\textsf{fetchCompleted}$ -- a variable to store whether the round messages have been fetched.
    \item[\ding{113}] $\mathcal{C}_{\text{loc}}$ -- the local chain the party adopts based on which it does evaluation and
    exports the ledger state.

\end{legal}
\bigbreak

\noindent
\textbf{Main Spectrum's Consensus Entities, Actors and Variables:}
\begin{legal}
    \item[\ding{113}] $L^+$ -- the main Spectrum's super-ledger.
    \item[\ding{113}] $L^{\text{loc}, k}$ -- ledger of the $k$-th connected distributed system.
    \item[\ding{113}] $V^k_n$ -- validators set of $k$-th connected distributed system active in the epoch $e_n$.
\end{legal}







