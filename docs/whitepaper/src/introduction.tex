Following the success of Bitcoin, many blockchain-based cryptocurrencies have been developed and deployed.
To meet different requirements in various scenarios, a great number of heterogeneous blockchains have emerged.
However, most of the presented blockchain platforms are isolated systems.
Therefore, interoperability between blockchains become one of the key issues
that prevent the blockchain technology from wide adoption.

With fair blockchain interoperability users can potentially conduct transactions across different blockchain networks\
smoothly without any intermediaries.
This guarantees a reduction in the fragmentation of the crypto ecosystem and opens up new horizons and business models.
Implementation of the blockchain interoperability protocol is challenging since different blockchains have\
different security solutions, consensus algorithms and programming languages.
An inaccurate solution can potentially increase the possibility of attacks and create management challenges\
across different connected networks.

\subsection{Existing Solutions}\label{subsec:existing-solutions}

\subsubsection{Centralized Solutions}\label{subsubsec:centralized-solutions}

A classical solution to cross-chain interoperability is a trusted oracle that registers some event on one blockchain\
and performs the required action on the other.
Centralized oracles provide fast and cheap transactions but lack a key feature, decentralization.
The liquidity of a protocol built on this approach is custodial, which is a centralized approach similar to CeFi when\
users deposit their funds to an exchange's wallet.

The disadvantages of this type of protocols are obvious:
\begin{enumerate}
    \item A system is not sustainable when it depends on a single party.
    \item If the oracle goes down unfinished swaps can appear frozen halfway.
    \item A centralized oracle can censor transactions.
    \item A malicious oracle can perform a man-in-the-middle attack by sending an inaccurate data.
\end{enumerate}

\subsubsection{Semi-Centralized Solutions}\label{subsubsec:semi-centralized-solutions}

Another common approach involves intermediate network consisting of fixed number of hand-picked oracles facilitating\
the transfer of data among multiple blockchains.
The consensus mechanism in such protocols is usually proof-of-authority or proof-of-stake and a wide range\
of potential validators is eliminated due to verification procedures or high collateral.
Consequently, network moderation is typically carried out by several dozen of rarely alternating nodes.
Moreover, the common practice is to store the funds transferred between blockchains on some kind of threshold wallets,\
which are generated by the participants of the intermediate network.
This results in all funds being controlled by a fixed group of oracle operators.

Therefore, despite the described above method of the cross-chain interoperability is slightly more decentralized,\
it has almost the same disadvantages as the centralized one.