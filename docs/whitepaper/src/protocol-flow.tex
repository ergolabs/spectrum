Let's summarize all of the above and describe the full flow of the Spectrum protocol.
Protocol is running by a set of manually selected opening consensus groups $\{V^k_1\}_{k=1}^K$\
for $K$ connected distributed systems $\{S_k\}_{k=1}^K$.
Each group consists of at least $M_k|_{\min}$ stakeholders interacting with each other and with the ideal\
functionalities $\mathcal{F}_{\text{Init}}$, ${\mathcal{F}}_{\text{VRF}}$, $\mathcal{H}$, ${\mathcal{F}}_{\text{LB}}$,\
$\mathcal{F}_{\text{AggSig}}$, ${\mathcal{F}}_{\text{KES}}$,\
$\mathcal{G}_{\text{ImpLClock}}$ and $\mathcal{G}_{\text{Ledger}}$ over a sequence of $L = E \cdot R$\
slots ${S=\{sl_1,\dots,sl_L\}}$ consisting of $E$ epochs with $R$ slots each.
Functionality ${\mathcal{F}}_{\text{Init}}$~\cite{Badertscher2018} formalizes the procedure of genesis\
block creation and distribution.
Functionality ${\mathcal{F}}_{\text{AggSig}}$ implements the presented aggregated signature scheme logic.
Functionality $\mathcal{G}_{\text{ImpLClock}}$~\cite{cryptoeprint:2019/838} implements the local\
clock setting and adjusting logic.
Functionality $\mathcal{G}_{\text{Ledger}}$ implements the logic of interaction with the ledger.

Also, each protocol participant maintains at least one functionality unit $\mathcal{F}^k_{\text{ConnSys}}$\
that allows him to interact with the connected $S_k$.

\subsubsection{Bootstrapping}\label{subsubsec:bootstrapping}

The system is bootstrapped in a trusted way.
All members of $\{V^k_1\}_{k=1}^K$ committees perform the following procedure:
\begin{enumerate}
    \item On-chain vaults are initialized with an aggregated public key $aPK^k_1$ of the initial committee.

    \item All committee $V^k_1$ members i.e. $\forall PK_i \in V^k_1$ must generate the tuple of verification keys\
    ${(v_i^{\text{vrf}}, v_i^{\text{kes}})}$, using the ideal functionalities ${\mathcal_{F}}_{\text{VRF}}$ and\
    ${\mathcal{F}}_{\text{KES}}$.

    \item Full set of the verification keys tuples\
    ${V_{\text{init}, k} = \{(v_i^{\text{vrf}}, v_i^{\text{kes}})\}_{i=1}^M_k}$\
    with the initial stakes $S_k = \{s_i\}_{i=1}^M_k$ must be stored in the genesis block and\
    acknowledged by all members of the initial consensus group (meaning members of all $K$ committees and\
    a full set is $\{V_{\text{init}, k}\}_{k=1}^{K}$).

    \item Functionality ${\mathcal{F}}_{\text{LB}}$, parameterized with the confirmed $\{V_{\text{init}, k}\}_{k=1}^{K}$\
    is evaluated independently by every\
    participant to sample an initial random seed value $\eta \leftarrow \{0, 1\}^l$.

    \item Finally, all approved stakeholders should agree on the genesis block\
    ${B_0=\left(\{V_{\text{init}, k}, S_k\}_{k=1}^{K}, \eta\right)}$.
\end{enumerate}

\subsubsection{Chain Extension}\label{subsubsec:chain-extension}
Once the system is bootstrapped, the Spectrum protocol operates in a normal flow.

\begin{legal}
    \item Before the epoch $e_n$ begins all consensus participants receives new\
    epoch seed $\eta_n$ from the ${\mathcal{F}}_{\text{LB}}$ and calculates leader lottery threshold\
    $T_{i, n}^{\text{lead}}$ and synchronization lottery threshold $T_{i, n}^{\text{sync}}$.

    \item In the epochs first (synchronization) slot all $PK_i$ adjusts their local clocks according to previously\
    collected synchronization beacons set $B_i^{\text{sync}, n - 1}$.
    The corresponding $\text{shift}_{i, n}$ value is calculated with the functionality $\mathcal{G}_{\text{ImpLClock}}$.

    \item During the epoch all online $V^k_n$ member collects existing chains from $L^+$ and verifying\
    that for every chain, every block, produced up to $K_{\text{f}}$ blocks before contains correct data about the\
    corresponding slot $sl'$ leader $PK'$.
    The validator must verify that $PK'$ is indeed the winner of the leader lottery for slot $sl'$ as well a valid\
    member of the legitimate committee.
    All forks must be resolved by the densest chain and largest stake rules in the corresponding priority.

    \item During the epoch, for every slot $sl_j$ every committee $V^k_n$ member $PK_i$ separately evaluates\
    ${\mathcal{F}}_{\test{VRF}}$ with his own input ${x_{i, j}^{\text{lead}, n} = \eta_n || sl_j}$ \ and compares the\
    value ${y_{i, j}^{\text{lead}, n} = \mathcal{H}(\pi_{i, j}^{\text{sl}*, n}||\textsf{LEAD})}$ with the associated\
    threshold $T_{i, n}^{\text{lead}}$.
    If successful $PK_i$ is the slot $sl_j$ leader and he can initiate the notarization round in his local committee.

    In addition, during the first $R\//6$ slots of the epoch all $PK_i$ evaluates $\mathcal{F}_{\text{VRF}}$\
    functionality with input ${x_{i, j}^{\textsf{sync}, n} = (\eta_n || sl_{i, j}^{\text{loc}, n})}$ to get a\
    proof $\pi_{i, j}^{\text{sl}, n}$ and check if\
    he has the right to release a synchronization beacon comparing the pseudo-random value\
    ${y_{i, j}^{\textsf{sync}, n} = \mathcal{H}(\pi_{i, j}^{\text{sl}, n} || \textsf{SYNC})}$\
    with a corresponding threshold $T_{i, n}^{\text{sync}}$.
    If successful then the participant broadcasts a beacon message\
    $b_{i, j}^{\text{sync}, n} = (v^{\text{vrf}}_i, sl_{i, j}^{\text{loc}, n}, \pi_{i, j}^{\text{sl}, n})$.

    \item All committee $V^k_n$ members observe events in their systems $S_k$ and in the $L+$ mempool.
    If $PK_i$ is a slot $sl_j$ leader, then he is able to propose a batch $b^n_j$ of events observed in $S_k$,\
    which should be notarized by other members of the $V^k_n$ using the aggregated signature functionality\
    ${\mathcal{F}}_{\text{AggSig}}$ and then added to the local ledger $L_k$.

    \item Notarized batch $(b^n_j)^*$ can first be formed by any member of the $V^k_n$.
    The batch must be immediately sent to the leader who initiated its notarization and to the\
    members of other committees.
    After the leader receives enough batches he forms a block  $B^n_j$ consisting\
    of all external collected batches and batches from the local $L_k$ that have not yet been added to $L^+$.
    He must include into the block the proof of his leadership $ \pi_{i, j}^{\text{sl}, n}$,\
    sign the block with ${\mathcal{F}}_{\text{KES}}$,\
    and broadcasts it to all committees.

    \item After the finality $K_{\text{f}}$ blocks are passed since $B^n_j$ settlement in the $L^+$,\
    all members of all committees that participated in the formation of the block $B^n_j$ can claim their rewards.

\end{legal}

\subsubsection{Epoch Transition}\label{subsubsec:epoch-transition}
\begin{legal}
    \item \textbf{Consensus Group Lottery}.
    At the end of the epoch ${e_{n-2} \gt 3}$ every verified $PK_i$ receives new epoch seed $\eta_n$\
    from the ${\mathcal{F}}_{\text{LB}}$.
    New consensus lottery threshold $T_{i, n}^{\text{cons}} = \phi(\alpha_{i, n}, f^{\text{cons}, n})$\
    is calculated by every $PK_i$\ using stake distribution (to get the relative stake $\alpha_{i, n}$)\
    from the last block used by ${\mathcal{F}}_{\text{LB}}$ to sample $\eta_n$.
    Parameter $f_n$ of the associated function $\phi$ is ${f^{\text{cons}, n} = M_n /\/ N_n}$,\
    where $M_n$ is a pre-defined number of new consensus group members to select at epoch $e_n$\
    and $N_n$ is the total number of verified stakeholders.
    When every $PK_i$ evaluates ${\mathcal{F}}_{\text{VRF}}$ with input\
    $x_{i, n}^{\textsf{cons}} = \eta_n || e_n $ and calculates\
    the associated random number $y_{i, n}^{\textsf{cons}}$ from the received proof $\pi_{i, n}^{\text{e}}$, i.e.\
    ${y_{i, j}^{\textsf{cons}} = \mathcal{H}(\pi_{i, n}^{\text{e}*}||\textsf{CONS})}$
    In addition, all participants must multiply all stakes by the corresponding multipliers $A^{i, n}_{\text{m}}$,\
    calculated from the activity of participants in the previous epoch.
    $PK_i$ compares value $y_{i, n}^{\textsf{cons}}$\
    with the threshold $T_{i, n}^{\text{cons}}$.
    If successful then $PK_i$ is a member of $V^k_n$.

    \item \textbf{Committee key aggregation}.
    Once the new committee is selected, nodes in the $V^k_n$ aggregate their individual public keys $PK_i$ into
    a joint one $aPK^k_n$, which is needed to sign the batch applying transactions with the external events:\
    inbound value transfers, outbound value transfers, boxes eliminations.

    \item \textbf{Committee transition}.
    Nodes in the $V^k_{n - 1}$ publish cross-chain message ${m^k_n : (aPK^k_n, \sigma^k_{n-1})}$,\
    where $\sigma^k_{n-1}$ is an aggregated signature such that ${verify: (\sigma^k_{n-1}, aPK^k_{n-1}, m^k_n) = 1}$.
    Finally, vaults are updated such that ${vault^k\{(e_{n-1}, aPK^k_{n-1})\} \coloneqq(e_n, aPK^k_n)}$.
\end{legal}

\subsubsection{Registration}\label{subsubsec:registration}
Any Spectrum stakeholder can register to become a committee member of his local system $S_k$.
To get a chance to be included in the set of validators $V^k_n$ of the epoch $e_n$\
participant $PK_i$ should register in the lottery during the epoch $e_{n-2}$ by publishing his verification tuple\
${(v_i^{\text{vrf}}, v_i^{\text{kes}})}$ into the $L^+$.
Once $K_{\text{f}}$ blocks are added on top of this publication the participant is considered as verified.
Before verification, $PK_i$ must synchronizes with the network by restoring the current chain $C$ from the\
genesis block $B_0$ received from the functionality $\mathcal{F}_{\text{Init}}$.
He also must adjust his local clock based on the synchronization beacons of the current global epoch using\
the functionality $\mathcal{G}_{\text{ImpLClock}}$.
When all synchronization processes are completed and the first slot of the epoch $e_{n + 2}$ begins, $PK_i$\
is considered a valid participant of the Spectrum protocol.

In the manner described, the Spectrum protocol reaches consensus and implements the cross-chain interoperability.
Our solution is fairly decentralized, fast and scalable, and thus can be used in\
a large number of applications and scenarios.