In this section we will describe the main components and general assumptions which is essential to\
conceptualize and construct the Spectrum protocol.

\subsection{Security Model Preliminaries}\label{subsec:security-model-preliminaries.}
We consider a semi-synchronous setting where protocol participants have a somewhat accurate common notion\
of the elapsed time and the network has an upper bound on the message delay, which is not known\
to the participants and is used as a security parameter.

We also assume that our model operates in the dynamic availability setting ~\cite{Badertscher2018},\
where an arbitrary (but upper-bounded) number of the consensus members may not be fully operational,\
e.g., due to network problems, reboots or software updates\
that affect some of their local resources including their network interface and clock.

\textbf{Time and Slots.}
We consider a setting where time is divided into discrete units called slots.
Each slot $sl_r$ is indexed by an integer $r \in \{1, 2, ..\}$, and\
the ledger associates one time slot with at most one block.
All actions of protocol participants necessary for its correct execution are also associated with specific slots.
The largest units of time in the protocol are epochs, each consisting of $R$ slots.

\textbf{Synchrony.}
A common assumption in known blockchains with a semi-synchronous setting is that\
all participants are equipped with roughly synchronized clocks and have access\
to the global clock setup for the synchronization.
An existing synchronization techniques are inapplicable to the standard model used for\
the analysis of Nakamoto-style consensus protocols, thus, there are no strong security\
guarantees in such a model in the case where the agreement on the current slot is\
replaced by the assumption of potentially unsynchronized local clocks that proceed at roughly the same speed.

We adopt a provably secure approach to global clock synchronization\
in the dynamic participation setting~\cite{cryptoeprint:2019/838}.
This approach assumes that members of the initial consensus group have access to local clocks\
and any discrepancies between parties' local time are insignificant in comparison with the slot duration.
This is still a typical approach, however, the key feature is an imperfect version of the clock functionality used\
as a global setup.
It allows parties to advance to a next epoch even before\
every honest member has finished with his current epoch.
Once in an epoch, participants synchronize their clocks based on public blockchain data.
Therefore, this mechanism ensures that all parties, both active and those who later join the protocol,\
can synchronize with other participants\
and will remain synchronized as long as they faithfully follow the protocol.

\textbf{Random Oracle.}
We assume that an ideal random oracle is available to each member of the consensus.
Random oracle models a function ${\mathcal{H} : \{0, 1\}^* \rightarrow \{0, 1\}^l, }$ which samples a\
uniformly random string from the ${\{0, 1\}^l}$ in response to some query, while any repeated\
queries are answered consistently.

\textbf{Security Configuration.}
We consider an untrustworthy network environment that allows for adversarial-controlled message delays\
and immediate adaptive corruption.
Namely, we allow the adversary $A$ to selectively delay any messages sent
by an honest party for up to $\Delta^{\text{net}}$ slots and corrupt parties without delay.

The Spectrum protocol is executed by a set of nodes $N$, where each node $n \in N$:
\begin{itemize}
    \item Is associated with a unique wallet holding a stake of tokens $s_n$.
    \item Is able to generate key-pairs ${(PK, SK)}$ without trusted public key infrastructure.
    \item Is able to sign messages ${sign: (SK, m) \rightarrow \sigma}$.
    \item Is able to verify signatures ${verify: (\sigma, PK, m) \rightarrow 0 | 1}$.
    \item Has access to a random oracle $\mathcal{H}$.
\end{itemize}

We assume that at any time $t$ a subset ${V \subseteq N}$ of nodes can be controlled\
by an adversary and are considered faulty.
Byzantine nodes can divert from the protocol and collude to attack the system while the remaining honest nodes follow\
the protocol.

\subsection{External Systems}\label{subsec:external-systems.}
We also assume multiple independent distributed systems ${S_1, \dots, S_K}$ with underlying ledgers ${L_1, \dots, L_K}$\
as defined in~\cite{cryptoeprint:2019/1128}.
For each ledger ${L_k, k \in K}$ there is a process $P_k$ that can influence the state evolution\
of the underlying ledger $L_k$ by committing a transaction $TX_k$ into it.
We extend the model defined in~\cite{cryptoeprint:2019/1128} by assuming that all ledgers allow for execution of\
simple predicates upon validation of transactions: ${verify: C \rightarrow 0 | 1}$, where $C$ is\
a \emph{context} that contains description of state the transaction interacts with.
There is also a function ${desc: TX_k \rightarrow DESC^{TX_k}}$ that maps transaction $TX_k$ to\
some \emph{description}, e.g.\ specifying the transaction value, recipient address, etc.
For each $S_k$ there is a corresponding functionality unit $\mathcal{F}^k_{\text{ConnSys}}$ that\
allows any node equipped with the unit to interact with $S_k$.
Each node $n \in N$ is equipped with at least one such functionality unit and at most $K$ functionality units.

\subsection{Transaction Ledger}\label{subsec:transaction-ledger.}
We adopt the definition of transaction ledger from~\cite{cryptoeprint:2016/889}.
A protocol $\Pi$ implements a robust transaction ledger, provided that $\Pi$ is divided into blocks that determine\
the order in which transactions are incorporated into the ledger.
Each block in this model is assigned to a specific time slot and the ledger must satisfy the following properties:
\begin{enumerate}
    \item \emph{Persistence.} Once a node of the system proclaims a certain transaction $TX$ as stable, the remaining\
    nodes, if queried, will either report $TX$ in the same position in the ledger or will not report as stable any\
    transaction in conflict to $TX$.
    Here the notion of stability is a predicate that is parameterized by a security parameter $K_{\text{f}}$, specifically, a\
    transaction is declared stable if and only if it is in a block that is more than $K_{\text{f}}$ blocks deep in the ledger.
    \item \emph{Liveness.} If all honest nodes in the system attempt to include a certain transaction then,\
    after time expires corresponding to $U_{\text{c}}$ slots (called the transaction confirmation time), all nodes, if queried\
    and responding honestly, will report the transaction as stable.
\end{enumerate}
