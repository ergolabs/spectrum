\subsection{Decentralized Cross-Chain Oracle}\label{subsec:cross-chain-oracle}

In oracle mode of operation the system is capable of providing a notarized set of events\
observed on supported blockchains.
Cross-Chain Oracle is a simple yet solving the cross-chain interoperability solution.

\subsection{Custodial Asset Management}\label{subsec:custodial-asset-management}

In custodial mode of operation the system is capable of managing user assets which are stored on corresponding\
blockchains in vaults.
Each vault stores epoch number $n$, an aggregated public key $aPK_n$ of the current validator set $V_n$ and
is guarded with a script (smart-contract) capable of performing verification of
an aggregated signature ${verify: (\sigma_n, m_n, aPK_n) \rightarrow 0 | 1}$.


\paragraph{Natively Cross-Chain Applications}

Decentralized custodial management in conjunction with a computational layer can be highly beneficial for expanding\
the capabilities of the system.
This moves us beyond simple bridges to what we call Natively Cross-Chain Applications (NCCAs).

NCCAs are applications that are deployed in cross-chain network and are capable of interacting with other blockchains\
without the need of external oracles or bridges.
Compared to single-chain dApps, NCCAs unlocks an additional functionality by taking advantage\
of multiple chains simultaneously.
They make possible to aggregate fragmented liquidity on different chains\
into one chain or a coordinated pool of assets and improve the user experience by enabling the localization\
and customization of parameters and feature sets of the same application on different chains.
These unique advantages make them the future of web3 dApps.
