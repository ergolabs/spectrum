Although system configuration so far offers a scalable way to ensure strong consistency, which means that clients only need to wait for the next block to verify a committed transaction, the time between blocks can still be significant.
Inspired by the concept of ByzCoin, we propose a system composed of two parallel chains that store information about leaders and transactions.
Epoch blocks store information about consensus group members and leaders, while blocks store information about transactions.
The current leader generates a new block every few seconds, this block is then signed with a collective signature.

As long as Spectrum is an open-membership protocol, assumption on a closed consensus group is nod valid.
Sybil attacks cat break any protocol with security thresholds such as PBFT's assumption that at most $f$ out of ${3 f + 1}$ members are honest, thus, an appropriate dynamic selection of opening consensus group is crucial for network livness and safety.
Consensus group members selection should be performed in a random and trusted way to ensure that a sufficient fraction of the selected members are honest, procedure itself should be independent of any internal or external advisers.
Results of the selection procedure should satisfy the following conditions:
\begin{enumerate}
    \item Results should be verifiable for all network participants.
    \item Fraction of honest members in the selected consensus group should be sufficient to guarantee Byzantine fault tolerance.
    \item Information about selected members of the consensus group shouldn't be accessible to any network participants until the start of the next epoch to avoid targeted attacks.
\end{enumerate}

To meet the above requirements, we introduce a trusted lottery-based mechanism.
Let's pretend that protocol participants can have 4 possible roles:
\begin{itemize}
    \item \textbf{Network participant}: participates in a lottery to become \textquote{Consensus group member} or(and) \textquote{Leadership contender}.
    \item \textbf{Consensus group member}.
    \item \textbf{Leadership contender}: can randomly become a leader once during the current epoch.
    \item \textbf{Leader}.
\end{itemize}

All users willing to serve the network register in protocol as a \textquote{Network participant}.
Each \textquote{Network participant} has a set of specified parameters ${(PK_i, V_i, M_i)}$, there $PK_i$ is public key, $V_i$ is a balance, $M_i$ is a number of issued blocks as a \textquote{Leader} during passed $K$ epochs.
Once previous epoch comes to an end current \textquote{Leader} sends a lottery message.
When \textquote{Network participant} signs this message, the lottery function determines his role in the next epoch:
\begin{equation}
{lotteryRole:(seed, V_{\max}, M_{\max}, T_c^M, T_l^C; H, V_i, M_i) \rightarrow role}
    ,\label{eq:equation3}
\end{equation}
where ${V_{\max}, M_{\max}}$ are predefined maximum values to which the corresponding user parameters ${V_i, M_i}$ are normalized in order to prevent whales and previous leaders domination.
Presence of new members is achieved by the fact that the above normalized parameters are multiplied by numbers ${A, B}$, randomly generated in a given ranges.
Thus, the \textquote{Consensus group members} become those \textquote{Network participant} for which ${A \cdot (V_i\mathbin{/}V_{\max}) + B \cdot (M_i\mathbin{/}M_{\max}) > T_c^M}$.
Similarly, to become \textquote{Leadership contender}: ${C \cdot (V_i\mathbin{/}V_{\max}) + D \cdot (M_i\mathbin{/}M_{\max}) > T_l^C}$.
Finally, two sets of participants are obtained: ${\{PK_j\}_j=0^J}$ \textquote{Consensus group members} and ${\{PK_k\}_k=0^K}$ \textquote{Leadership contenders}.
Role is assigned to each specific participant and determines concrete actions allowed for him within the protocol.
Thus, the lottery procedure is random, the distribution of roles correlates with the contribution of participants to the protocol, and information about roles is available only to the participants themselves.

\subsubsection{Lottery function details}\label{subsubsec:lottery-function-details}
todo

\subsubsection{Leadership transition}\label{subsubsec:leadership-transition}
The role of the \textquote{Leader} plays crucial in the network's reliability, so the following requirements are imposed on the leadership transition procedure:
\begin{enumerate}
    \item \textquote{Leader} position should be verifiable for all network participants.
    \item Leadership transition procedure should be cheap and fast.
    \item Information about next \textquote{Leader} shouldn't be accessible to any network members before the start of the next block to avoid targeted attacks.
\end{enumerate}
During the lottery, a set of the most trusted participants who can claim leadership during the next epoch was defined.
Ideally, each new block should be initiated by a new \textquote{Leader}, so this role can only be used once.
Each \textquote{Leader} includes a random number $P$ in his transaction, $P$ selection range is ${K - E}$, where $E$ is the number of blocks passed since current epoch started.
\textquote{Consensus group members} signing the \textquote{Leader's} transaction and those who also are \textquote{Leadership contenders} check their order (inside their group).
And the first one, for whom the order number will be greater or equal to $P$, takes on the role of the new \textquote{Leader}.

Thus, the leadership transition procedure is cheap and fast, since it is built into the main protocol flow and, in the limit, can be performed before each new block.