This section presents Spectrum protocol design starting from a naive approach based on PBFT and gradually addressing the challenges.

\subsection{Strawman Design: PBFTNetwork}\label{subsec:strawman-design}

For simplicity we begin with a notarization protocol based on PBFT, then iteratively refine it into Spectrum.

PBFTNetwork assumes that a group of ${n = 3f + 1}$ trusted nodes has been pre-selected upfront and fixed and at most $f$ of these nodes are byzantine.
At any given time one of these nodes is the \emph{leader}, who observes events on connected blockchains,
batch them and initiate round of notarization within the consensus group.
Remaining members of the consensus group verify the proposed batches by checking the presence of updates on corresponding blockchains.
Upon successful verification each node signs the batch with its secret key and sends the signature to the leader.

Under simplifying assumptions that at most $f$ nodes are byzantine the PBFTNetwork guarantees livness and safety.
However, the assumption of a fixed trusted committee is not realistic for open decentralized systems.
Moreover, as PBFT consensus members authenticate each other via non-transferable symmetric-key MACs, each consensus
member has to communicate with others directly, what results in $O(n^2)$ communication complexity.
Quadratic communication complexity imposes a hard limit on scalability of the system.
Such a design also scales poorly in terms of adding support for more chains.
The workload of each validator grows lineary with each added chain.

In the subsequent sections we address these limitations in four steps:
\begin{enumerate}
    \item \textbf{Opening consensus group and leaders.} We introduce a lottery-based mechanism for selecting consensus group and leaders dynamically.
    \item \textbf{Replacing MACs by Digital Signatures.} We replace MACs by digital signatures to make authentication transferable
    and thus opening the door for sparser communication patterns that can help to reduce the communication complexity.
    \item \textbf{Scalable Collective Signature Aggregation.} We utilize Byzantine-tolerant aggregation protocol that allows for
    quick aggregation of cryptographic signatures to reduce communication complexity to $O(\log n)$.
    \item \textbf{Eliminating Validator Bottleneck.} We shard consensus groups into units by the type of chain each node is able to handle.
\end{enumerate}

\subsection{Opening Consensus Group}\label{subsec:opening-consensus-group-and-leaders}
Spectrum is an open-membership protocol, so PBFTNetwork's assumption on a closed consensus group is not valid.
Sybil attacks can break any protocol with security thresholds and an appropriate dynamic selection of\
the consensus group becomes crucial for preserving network's liveness and safety.
Consensus group members selection should be performed in a random and trusted way to ensure that a sufficient fraction\
(at most $f$ out of ${3 f + 1}$) of the selected members are honest and of course\
the selection procedure must be independent of any internal or external advisers.

Similar selection mechanics are required in most blockchain protocols.
Bitcoin~\cite{nakamoto2009bitcoin} and many its successors are using Proof-of-Work (PoW) consensus,\
which, in essence, is a robust mechanism that facilitates randomized selection of a leader who is\
eligible to produce a new block.
Later, PoW approach was adapted into a Proof-of-Membership mechanism ~\cite{kokoriskogias2016enhancing}\.
This mechanism allows once in a while to select a new consensus group\
which then executes the PBFT consensus protocol.

A primary consideration regarding PoW-based consensus mechanisms is\
the amount of energy required to operate such systems.
A natural alternative to PoW is a mechanism based on the concept of Proof-of-Stake (PoS)~\cite{King2012PPCoinPC}.
Rather than investing computational resources in order to participate in the leader selection process,\
participants of a PoS system instead run a process that randomly selects one of them proportionally to the stake.
Pure PoS mechanism to solve the PBFT problem was firstly used in~\cite{cryptoeprint:2017/454} to select both consensus\
group members and PBFT rounds leaders and to introduce randomness into this process,\
a verifiable Random Function (VRF) has been applied.

\subsubsection{Verifiable Random Function}

A Verifiable Random Function (VRF)~\cite{Micali1999} is a reliable way to introduce randomness into a protocol.
By definition, a function $\mathcal{F}$ can be attributed to the VRF family if the following methods are defined\
for the $\mathcal{F}$:
\begin{itemize}
    \item Gen: ${Gen(1^k) \rightarrow (PK, SK)}$, where $PK$ is the public key and $SK$ is the secret key;
    \item Prove: ${Prove(x, SK) \rightarrow \pi}$, where $x$ is an input and $\pi \vcentcolon= \Pi(x, SK)$ is\
    the proof, associated with $x$ and mixed with some random value $y$, sampled from $\{0,1\}^{l_{VRF}}$.
    \item Verify: ${Verify(x, \pi, PK) \rightarrow 0 | 1}$, where the output is $1$ if\
    and only if ${\pi \equiv \Pi(x, SK)}$.
\end{itemize}

The most secure nowadays are an Elliptic Curve Verifiable Random Functions (ECVRFs).
Basically, ECVRF is a cryptographic-based VRF that satisfies the trusted uniqueness, trusted collision resistance,\
and full pseudorandomness properties ~\cite{cryptoeprint:2014/905}.
The security of ECVRF follows from the decisional Diffie-Hellman assumption in the random oracle model, thus\
ECVRF is a good source of randomness for a blockchain protocol.
Using ECVRF is also cheap and fast, since single ECVRF evaluation is approximately 100 microseconds on\
x86-64 for a specific curves used in hash functions.
Moreover, there is a great UC-extension for batch verification proposed by ~\cite{cryptoeprint:2022/1045}\
which make it even faster.

\subsubsection{Lottery}
Our lottery mechanism is based on ECVRF as a source of randomness and is generally inspired\
by~\cite{cryptoeprint:2017/573}.
The lottery is designed to achieve two main purposes: \emph{select a consensus group dynamically},
\emph{select a slot leader}.

\textbf{Lottery Function}.
The main selection logic is implemented in the lottery function.
The lottery function ${\mathcal{F}}_{L}$ compares a random number $y$ derived from the generated VRF random\
proof $\pi$ with publicly known threshold value $T$.
It evaluates to $1$ if and only if ${y < T}$, i.e.\
${\mathcal{F}}_{L}(state, f, \pi) \rightarrow 0|1$ where $state$ is a blockchain state snapshot.

The threshold value is calculated according to the formula ${T = 2^{l_{VRF}}\cdot \phi(\alpha, f)}$ where\
${\alpha=s/\\\sum_{l=0}^{l=M} s_l}$ is a relative stake.
Consequently, the probability of winning is calculated as ${p(\alpha, f) = 1-(1-f)^{\alpha}}$\
Winning probability depends on the participants' relative stake and is adjusted by the free parameter $f$.
This is where the PoS concept comes into play: the bigger the stake, the higher the chance of winning the lottery.

\textbf{Consensus Group Lottery}.
The Spectrum protocol initially is running by the manually selected opening consensus group $\{PK_i\}_{i=1}^M$\
of the predefined size $M$.
Stakeholders interact with each other and with locally installed ideal functionalities ${\mathcal{F}}_{LB}$,\
${\mathcal{F}}_{VRF}, {\mathcal{F}}_{L}$ over a sequence of $L = E \cdot R$ slots\
${S=\{sl_1,...,sl_L\}}$ consisting of $E$ epochs with $R$ slots each.
Let's clarify what the mentioned above pre-defined primitives are needed for:
\begin{enumerate}
    \item \emph{Ideal Leaky Beacon} ${\mathcal{F}}_{LB}$: is used to sample an epoch random seed from the\
    blockchain.
    \item \emph{Ideal Verifiable Random Function} ${\mathcal{F}}_{VRF}$: is used as a source of randomness.
    \item \emph{Lottery Function} ${\mathcal{F}}_{L}$: checks if the protocol participant is a lottery winner\
    (by lottery, we mean either the \emph{consensus group lottery} or the \emph{leader lottery} giving\
    the ability to start a batch notarization round, the lottery function in both cases remains the same,\
    only the arguments matter).
\end{enumerate}
More extended formal definition of ${\mathcal{F}}_{LB}$ and ${\mathcal{F}}_{VRF}$ can be found in the original\
Ouroboros Praos paper ~\cite{cryptoeprint:2017/573}.

Each new epoch ${e_j \gt 1}$ has a new consensus group and any protocol participant\
can try to become a member if he is verified.
Participant is verified if his verification key tuple is published and stored in the blockchain for a\
reliable period of time equals to $U_f$ slots.

At the end of the epoch ${e_j \gt 1}$ every verified $PK_i$ requests a\
new epoch seed $\eta_j$ from the ${\mathcal{F}}_{LB}$.
When every $PK_i$ evaluates ${\mathcal{F}}_{VRF}$ and passes the received proof $\pi$ to the ${\mathcal_{F}}_{L}$ to\
reveal the result of the consensus group lottery.
To calculate an appropriate threshold ${T_i^j}$, ${\mathcal{F}}_{L}$ should be parametrized with the same\
stake distribution which was in the last block used by ${\mathcal{F}}_{LB}$ to sample the new $\eta_j$.
Argument of the winning probability function $p$ is ${f = M /\/ N}$, where $M$ is a number of new consensus group\
members to select and $N$ is the total number of the verified stakeholders.

For security purposes, a lower bound number of committee members $M$ is required.
If the number of verified participants is less than ${M_{\min}=256}$, then the consensus group lottery is not held.
The maximum consensus group size is ${M_{\max}=1024}$, this limit arises due to the complexity of communication.

\textbf{Leader Lottery}.
Once a new consensus group is determined, the lottery process does not stop, but this time\
the leader of the group should be determined.

During an epoch, for each slot ${sl_l \in S}$, each member of the consensus group $PK_i$ separately evaluates\
${\mathcal{F}}_{VRF}$ with his own input ${x = \eta_j || sl_l || \emph{nonce}}$.
In response, he receives the associated random proof $\pi$.
If ${\mathcal{F}}_{L}(state, f, \pi) \rightarrow 1$ then $PK_i$ is a slot leader and he can propose a batch\
which should be notarized by at least two-thirds of committee members.

The parameter $f$ that regulates the probability of winning is different from the one used in the consensus\
group lottery.
Here, it is the pre-defined value determines how many slots will have at least one selected leader,\
it is called an active slots coefficient.

The lottery mechanism described in this subsection is fast, secure, and adaptive, since the pre-defined parameters\
can be changed via the voting process.
The same primitives are used to achieve different goals, namely, select a consensus group dynamically\
and select a slot leader.

Regarding the security it is important to note, that participants use their public VRF-keys for\
VRF functionality evaluation in the consensus group lottery and secret VRF-keys in the leader lottery.
This way, slot leaders don't become publicly known in advance.
An attacker can't see who is a slot leader until he initializes batch notarization, thus an attacker can't know\
who specifically to attack in order to control a certain slot.
Opening consensus group members on the over hand should be known ahead of time for the synchronization.
There are hundreds of consensus members in every epoch, so denial of service attacks are difficult to succeed.
Grinding attacks are limited because an adversary can't arbitrarily control $\eta_j$ values,
all he can try to do is to make as many forks as possible to estimate the most advantageous, but according to the\
analysis~\cite{cryptoeprint:2017/573} this advantage doesn't change the security properties of the entire protocol.



\subsection{Replacing MACs by Digital Signatures}\label{subsec:replacing-macs-by-digital-signatures}

todo

\subsection{Scalable Collective Signature Aggregation}\label{subsec:scalable-collective-signature-aggregation}

todo

\subsection{Eliminating Validator Bottleneck}\label{subsec:eliminating-validator-bottleneck}

So far each member of consensus group had to track changes on all connected chains in order to participate in consensus properly.

\textbf{Observation 1:} Events coming from independent systems $S_k$ are not serialized.
Thus, the process of events notarisation can be parallelized.

\textbf{Observation 2:} Outbound transactions on independent systems $S_k$ can be independently signed.

Utilizing those properties we now introduce committee sharding.
We modify protocol in a way such that at each epoch $e$ $M$ distinct committees consisting of nodes equipped with functionality unit $F_{S_k}$ relevant to a specific connected chain $S_k$ are selected in a way described in (5.2.2).
All primitives and source of randomness are equal to different committees, the only difference is in the $f$ parameter of $\phi(\alpha_i, f)$ function, which is unique for every connected blockchain in order to guaranty expected number of members in every committee.
We denote one such committee shard as $V^{e}_{S_k}$, which uniquely maps to $S_k$.
Then, complete mapping of committees to chains at epoch $e$ can be represented as a set of tuples commettee-chain $\{(V^{e}_{S_k}, S_k)\}$.
Throughout epoch $e$ all events and on-chain transactions on $S_k$ are handled exclusively by $V^{e}_{S_k}$.

Nodes in $V^{e}_{S_k}$ maintain a robust local ledger $L^{local}_k$ of notarized batches of events observed in $S_k$.

\subsubsection{Syncing Shards}

Notarized batches of events from local ledgers $\{L_i\}_{i=1}^{i=N}$ then should be synced in a super ledger $L^+$ in order for the system to be able to compute cross-chain state transition.
To facilitate this process batches of notarized events are broadcast to other committees.
The main actors at this stage are:
\begin{itemize}
    \item \textbf{Local leaders}: committees leaders, holding local notarized batches.
    \item \textbf{Relayers}: any protocol participant, who broadcasts notarized batches from \emph{Local leaders} to other committees' members.
    Every \emph{Local leader} can be a \emph{Relayer} at the same time.
    \item \textbf{General leader}: one of the \emph{Local leaders} who added a block consisted of all collected notarized batches to the $L^+$.
\end{itemize}

Since any \emph{Local leader} is able to publish his block to $L^+$ he can choose from two main strategies:
\begin{itemize}
    \item \textbf{Wait}: malicious strategy where \emph{Local leader} waits for broadcasts from other committees members and don't broadcast his own batch to eliminate competitors for adding a block.
    \item \textbf{Broadcast and wait}: fair strategy where \emph{Local leader} immediately broadcasts his batch, waits for broadcasts from committees members and honestly competes for adding a block.
\end{itemize}
Thus, there should be a motivation for individual \emph{Local leader} to choose the fair strategy instead of keeping his batch for too long.
This is achieved through the design of the incentive system.

There are three types of incentive: ${\{R_b, R_d, R_m\}}$, where $R_b$ is a guaranteed reward for adding a notarized batch to the block, $R_d$ is given for a broadcasting batch to the general leader and $R_m$ is given personally to the \emph{General leader} who mined the block.
Delivery reward $R_d$ is given to the \emph{Relayer} if and only if a delivery was made within a predetermined period of time $\Delta t_d$.
From the game-theoretic analysis, the following relationships between rewards were derived: ${R_b = 2 \cdot R_d, R_m = 3 \cdot R_d}$.
Thus, if ${R_d=0}$ there is no prior strategy for the \emph{Local leaders}, they will or wait for other batches either broadcast their batches with equal probability.
In case when ${R_d>0}$ it is distributed between the \emph{Local leader} and the \emph{Relayer}, i.e. ${R^l_d = \xi \cdot R_d}$ and  ${R^r_d = (1 - \xi) \cdot R_d}$, where ${\xi \in (0, 1)}$.
While ${\xi \rightarrow 1}$ probability that all \emph{Local leaders} will choose the \emph{Broadcast and wait} strategy approaches $1$.

As a result, the syncing Shards flow looks as follows:
\begin{itemize}
    \item Every \emph{Local leader} broadcasts (himself or through an intermediary as a \emph{Relayer}) his batch $b_i$, which contains the local notarization time $t^N_i$ and waits for batches from other \emph{Local leaders}.
    \item When waiting time approaches $\Delta t_d$, \emph{Local leader} forms a block from all collected batches ${\{b_i^j\}_{j=1}^{j=K}, K \le N}$ and add it to $L^+$.
    Block contains the set of the notarization times $\{t^{N^j}_i\}_{j=1}^{j=K}$ and block creation time $t^B_s$.
    \item After block is settled, all associated actors receive their rewards according to their roles: \emph{General leader} receives $R_m$, \emph{Local leaders}, whose batches are in the block receives $R_b$.
    In addition, if ${t^B_s - t^N_i^* < \Delta t_d}$, where $t^N_i^*$ is $t^N_i$ time, normalized to $L^+$ time, $i$-th committee \emph{Local leader} receives $R_d$ reward shared with the \emph{Relayer}.
\end{itemize}

\subsection{Resolving forks}\label{subsec:resolving-forks}
TODO

\subsection{Protocol Flow}\label{subsec:protocol-flow}

\subsubsection{Bootstrapping}\label{subsubsec:bootstrapping}

The system is bootstrapped in a trusted way.
A manually picked set of validators $V_0$ is assigned to the first epoch $e_0$.
On-chain vaults are initialized with an aggregated public key $aPK_0$ of the initial committee.
All initial committee members generate verification tuples ${(v_i^{vrf}, v_i^{kes}, v_i^{dsig})}$
and agree on the genesis block.

\subsubsection{Normal Flow}\label{subsubsec:normal-flow}

\begin{enumerate}
    \item Registration.
    All Spectrum stakeholders can register for becoming a committee member.
    To get a chance of becoming a member of $V_n$ in the epoch $e_n$ they register in a lottery during the $e_{n-2}$
    epoch by publishing their verification tuples ${(v_i^{vrf}, v_i^{kes}, v_i^{dsig})}$.
    \item Lottery.
    Once registration is done and epoch $e_{n-1}$ comes to the end, all registered participants evaluates
    ${\mathcal_{F}}_{VRF}$ locally and compare the generated random $y$ with their corresponding consensus threshold
    ${T_i^j}^*$ for this epoch.
    If successful, then publish $y$ and the associated proofs to form an approved consensus members table.
    \item Committee key aggregation.
    Once new committee is selected, nodes in $V_n$ aggregate their individual public keys $\{PK_i\}$ into
    a joint one $aPK_n$.
    \item Committee transition.
    Nodes in $V_{n-1}$ publish cross-chain message ${M_n : (aPK_n, \sigma_{n-1})}$ , where $aPK_n$ is
    an aggregated public key of the new committee $V_n$ , $\sigma_{n-1}$ is an aggregated signature of
    $M_n$ such that ${Verify(\sigma_{n-1}, aPK_{n-1}, Mn) = 1}$.
    Vaults are updated such that ${Vault\{(E_{n-1}, aPK_{n-1})\} \coloneqq (e_n, aPK_n)}$.
    \item Decentralized Asset Management (Custodial).
    Nodes in $V_n$ observe events on supported L1 chains, agree on the set of updates
    and compute state outbound state transitions accordingly.
    \item Notarisation (Non-custodial).
    Nodes in $V_n$ observe events on supported L1 chains, batch updates, collectively sign them and
    publish on-chain.
\end{enumerate}