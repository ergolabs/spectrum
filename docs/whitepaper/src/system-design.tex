This section presents Spectrum protocol design starting from a naive approach based on PBFT and gradually addressing the challenges.

\subsection{Strawman Design: PBFTNetwork}\label{subsec:strawman-design}

For simplicity we begin with a notarization protocol based on PBFT, then iteratively refine it into Spectrum.

PBFTNetwork assumes that a group of ${n = 3f + 1}$ trusted nodes has been pre-selected upfront and fixed and at most $f$ of these nodes are byzantine.
At any given time one of these nodes is the \emph{leader}, who observes events on connected blockchains,
batch them and initiate round of notarization within the consensus group.
Remaining members of the consensus group verify the proposed batches by checking the presence of updates on corresponding blockchains.
Upon successful verification each node signs the batch with its secret key and sends the signature to the leader.

Under simplifying assumptions that at most $f$ nodes are byzantine the PBFTNetwork guarantees livness and safety.
However, the assumption of a fixed trusted committee is not realistic for open decentralized systems.
Moreover, as PBFT consensus members authenticate each other via non-transferable symmetric-key MACs, each consensus
member has to communicate with others directly, what results in $O(n^2)$ communication complexity.
Quadratic communication complexity imposes a hard limit on scalability of the system.
Such a design also scales poorly in terms of adding support for more chains.
The workload of each validator grows lineary with each added chain.

In the subsequent sections we address these limitations in four steps:
\begin{enumerate}
    \item \textbf{Opening consensus group.} We introduce a lottery-based mechanism for selecting consensus group dynamically.
    \item \textbf{Replacing MACs by Digital Signatures.} We replace MACs by digital signatures to make authentication transferable
    and thus opening the door for sparser communication patterns that can help to reduce the communication complexity.
    \item \textbf{Scalable Collective Signature Aggregation.} We utilize Byzantine-tolerant aggregation protocol that allows for
    quick aggregation of cryptographic signatures to reduce communication complexity to $O(\log n)$.
    \item \textbf{Eliminating Validator Bottleneck.} We shard consensus groups into units by the type of chain each node is able to handle.
\end{enumerate}

\subsection{Opening Consensus Group}\label{subsec:opening-consensus-group}

First, we address PBFTNetwork's assumption of a closed consensus group.
The main challenge at this step is to guarantee security against Sybil attacks in an open-mempership protocol that still assumes security thresholds.
Bitcoin~\cite{nakamoto2009bitcoin} and many of its successors are using proof-of-work (PoW) to achieve security.
In essence, PoW in Bitcoin is a robust machanism that facilitates randomized selection of a \enquote{leader} that is eligible to produce a new block.
A primary consideration regarding PoW is the amount of energy required for the systems that operate on it.
A natural alternative to PoW is a mechanism that relies on the notion of proof-of-stake (PoS).
Rather than investing computational resources in order to participate in the leader election process, participants of a PoS system instead run a process that
randomly selects one of them proportionally to the stake each possesses according to the current state of blockchain.

\subsection{Replacing MACs by Digital Signatures}\label{subsec:replacing-macs-by-digital-signatures}

todo

\subsection{Scalable Collective Signature Aggregation}\label{subsec:scalable-collective-signature-aggregation}

todo

\subsection{Eliminating Validator Bottleneck}\label{subsec:eliminating-validator-bottleneck}

So far each member of consensus group had to track changes on all connected chains in order to participate in consensus properly.

\textbf{Observation 1:} Events coming from independent systems $S_k$ are not serilized.
Thus, the process of events notarisation can be parallelized.

\textbf{Observation 2:} Outbound transactions on independent systems $S_k$ can be independently signed.

Utilizing those properties we now introduce commettee sharding.
We modify protocol in a way such that at each epoch $e$ $M$ disinct commettees consisting of nodes equipped with functionality unit $F_{S_k}$ relevant to a specific connected chain $S_k$ are selected.
We denote one such commettee shard as $V^{e}_{S_k}$, which uniqiely maps to $S_k$.
Then, complete mapping of commettees to chains at epoch $e$ can be represented as a set of tuples commettee-chain $\{(V^{e}_{S_k}, S_k)\}$.
Throughout epoch $e$ all events and on-chain trasactions on $S_k$ are handled exclusively by $V^{e}_{S_k}$.

Nodes in $V^{e}_{S_k}$ maintain a robust local ledger $L^{local}_k$ of notarized batches of events observed in $S_k$.

\subsubsection{Syncing Shards}

Notarized batches of events from local ledgers then must be synced in a super ledger $L^+$ in order for the system to be able to compute cross-chain state transition.
To facilitate this process batches of notarized events are broadcast to other commettees.
Any commettee member is able to add notarized batch of events to $L^+$ with inequal probability.
Probabilities of addiding a batch of events to $L^+$ are randomly assigned to each shard leader $U^{sl_r}_{S_k}$ at each slot $sl_r$ resulting in a mapping $\{(U^{sl_r}_{S_1}, P_1), (U^{sl_r}_{S_2}, P_1), .., (U^{sl_r}_{S_k}, P_i)\}$.

\subsection{Protocol Flow}\label{subsec:protocol-flow}

\subsubsection{Bootstrapping}\label{subsubsec:bootstrapping}

The system is bootstrapped in a trusted way.
A manually picked set of validators $V_0$ is assigned to the first epoch $e_0$.
On-chain vaults are initialized with an aggregated public key $aPK_0$ of the initial committee.

\subsubsection{Normal Flow}\label{subsubsec:normal-flow}

\begin{enumerate}
    \item Registration.
    Before an epoch starts, all Spectrum stakeholders can register for becoming a committee member.
    To get a chance of becoming a member of $V_n$ in the next epoch $e_n$ they register in a lottery
    by publishing their public keys $PK_c$ and locking collateral.
    \item Lottery.
    Once registration is done, nodes in $V_{n-1}$ compute ${selectComettee: (C_n, R_n) \rightarrow V_n}$,
    where  $C_n$ is a candidates pool, $R_n$ is a public random number.
    \item Committee key aggregation.
    Once new committee is selected, nodes in $V_n$ aggregate their individual public keys $\{PK_i\}$ into
    a joint one $aPK_n$.
    \item Committee transition.
    Nodes in $V_{n-1}$ publish cross-chain message ${M_n : (aPK_n, \sigma_{n-1})}$ , where $aPK_n$ is
    an aggregated public key of the new committee $V_n$ , $\sigma_{n-1}$ is an aggregated signature of
    $M_n$ such that ${Verify(\sigma_{n-1}, aPK_{n-1}, Mn) = 1}$.
    Vaults are updated such that ${Vault\{(E_{n-1}, aPK_{n-1})\} \coloneqq (e_n, aPK_n)}$.
    \item Decentralized Asset Management (Custodial).
    Nodes in $V_n$ observe events on supported L1 chains, agree on the set of updates
    and compute state outbound state transitions accordingly.
    \item Notarisation (Non-custodial).
    Nodes in $V_n$ observe events on supported L1 chains, batch updates, collectively sign them and
    publish on-chain.
\end{enumerate}