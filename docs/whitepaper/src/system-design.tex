This section presents Spectrum protocol design starting from a naive approach based on PBFT and gradually\
addressing the challenges.

\subsection{Strawman Design: PBFTNetwork}\label{subsec:strawman-design}

For simplicity we begin with a notarization protocol based on PBFT, then iteratively refine it into Spectrum.

PBFTNetwork assumes that a group of ${n = 3f + 1}$ trusted nodes has been pre-selected upfront and fixed and at most\
$f$ of these nodes are byzantine.
At any given time one of these nodes is the \emph{leader}, who observes events on connected blockchains,
batch them and initiate round of notarization within the consensus group.
Remaining members of the consensus group verify the proposed batches by checking the presence of updates on\
corresponding blockchains.
Upon successful verification each node signs the batch with its secret key and sends the signature to the leader.

Under simplifying assumptions that at most $f$ nodes are byzantine the PBFTNetwork guarantees livness and safety.
However, the assumption of a fixed trusted committee is not realistic for open decentralized systems.
Moreover, as PBFT consensus members authenticate each other via non-transferable symmetric-key MACs, each consensus
member has to communicate with others directly, what results in $O(n^2)$ communication complexity.
Quadratic communication complexity imposes a hard limit on scalability of the system.
Such a design also scales poorly in terms of adding support for more chains.
The workload of each validator grows linearly with each added chain.

In the subsequent sections we address these limitations in four steps:
\begin{enumerate}
    \item \textbf{Opening consensus group and leaders.} We introduce a lottery-based mechanism for selecting consensus\
    group and leaders dynamically.
    \item \textbf{Replacing MACs by Digital Signatures.} We replace MACs by digital signatures to make authentication\
    transferable and thus opening the door for sparser communication patterns that can help to reduce\
    the communication complexity.
    \item \textbf{Scalable Collective Signature Aggregation.} We utilize Byzantine-tolerant aggregation\
    protocol that allows for quick aggregation of cryptographic signatures to reduce communication complexity\
    to $O(\log n)$.
    \item \textbf{Eliminating Validator Bottleneck.} We shard consensus groups into units by the type of chain\
    each node is able to handle.
\end{enumerate}

\subsection{Opening Consensus Group}\label{subsec:opening-consensus-group-and-leaders}
Spectrum is an open-membership protocol, so PBFTNetwork's assumption on a closed consensus group is not valid.
Sybil attacks can break any protocol with security thresholds and an appropriate dynamic selection of\
the consensus group becomes crucial for preserving network's liveness and safety.
Consensus group members selection should be performed in a random and trusted way to ensure that a sufficient fraction\
(at most $f$ out of ${3 f + 1}$) of the selected members are honest and of course\
the selection procedure must be independent of any internal or external advisers.

Similar selection mechanics are required in most blockchain protocols.
Bitcoin~\cite{nakamoto2009bitcoin} and many its successors are using Proof-of-Work (PoW) consensus,\
which, in essence, is a robust mechanism that facilitates randomized selection of a leader who is\
eligible to produce a new block.
Later, PoW approach was adapted into a Proof-of-Membership mechanism ~\cite{kokoriskogias2016enhancing}\.
This mechanism allows once in a while to select a new consensus group\
which then executes the PBFT consensus protocol.

A primary consideration regarding PoW-based consensus mechanisms is\
the amount of energy required to operate such systems.
A natural alternative to PoW is a mechanism based on the concept of Proof-of-Stake (PoS)~\cite{King2012PPCoinPC}.
Rather than investing computational resources in order to participate in the leader selection process,\
participants of a PoS system instead run a process that randomly selects one of them proportionally to the stake.
Pure PoS mechanism to solve the PBFT problem was firstly used in~\cite{cryptoeprint:2017/454} to select both consensus\
group members and PBFT rounds leaders and to introduce randomness into this process,\
a verifiable Random Function (VRF) has been applied.

\subsubsection{Verifiable Random Function}

A Verifiable Random Function (VRF)~\cite{Micali1999} is a reliable way to introduce randomness into a protocol.
By definition, a function $\mathcal{F}$ can be attributed to the VRF family if the following methods are defined\
for the $\mathcal{F}$:
\begin{itemize}
    \item Gen: ${Gen(1^k) \rightarrow (PK, SK)}$, where $PK$ is the public key and $SK$ is the secret key;
    \item Prove: ${Prove(x, SK) \rightarrow \pi}$, where $x$ is an input and $\pi \vcentcolon= \Pi(x, SK)$ is\
    the proof, associated with $x$ and mixed with some random value $y$, sampled from $\{0,1\}^{l_{VRF}}$.
    \item Verify: ${Verify(x, \pi, PK) \rightarrow 0 | 1}$, where the output is $1$ if\
    and only if ${\pi \equiv \Pi(x, SK)}$.
\end{itemize}

The most secure nowadays are an Elliptic Curve Verifiable Random Functions (ECVRFs).
Basically, ECVRF is a cryptographic-based VRF that satisfies the trusted uniqueness, trusted collision resistance,\
and full pseudorandomness properties ~\cite{cryptoeprint:2014/905}.
The security of ECVRF follows from the decisional Diffie-Hellman assumption in the random oracle model, thus\
ECVRF is a good source of randomness for a blockchain protocol.
Using ECVRF is also cheap and fast, since single ECVRF evaluation is approximately 100 microseconds on\
x86-64 for a specific curves used in hash functions.
Moreover, there is a great UC-extension for batch verification proposed by ~\cite{cryptoeprint:2022/1045}\
which make it even faster.

\subsubsection{Lottery}
Our lottery mechanism is based on ECVRF as a source of randomness and is generally inspired\
by~\cite{cryptoeprint:2017/573}.
The lottery is designed to achieve two main purposes: \emph{select a consensus group dynamically},
\emph{select a slot leader}.

\textbf{Lottery Function}.
The main selection logic is implemented in the lottery function.
The lottery function ${\mathcal{F}}_{L}$ compares a random number $y$ derived from the generated VRF random\
proof $\pi$ with publicly known threshold value $T$.
It evaluates to $1$ if and only if ${y < T}$, i.e.\
${\mathcal{F}}_{L}(state, f, \pi) \rightarrow 0|1$ where $state$ is a blockchain state snapshot.

The threshold value is calculated according to the formula ${T = 2^{l_{VRF}}\cdot \phi(\alpha, f)}$ where\
${\alpha=s/\\\sum_{l=0}^{l=M} s_l}$ is a relative stake.
Consequently, the probability of winning is calculated as ${p(\alpha, f) = 1-(1-f)^{\alpha}}$\
Winning probability depends on the participants' relative stake and is adjusted by the free parameter $f$.
This is where the PoS concept comes into play: the bigger the stake, the higher the chance of winning the lottery.

\textbf{Consensus Group Lottery}.
The Spectrum protocol initially is running by the manually selected opening consensus group $\{PK_i\}_{i=1}^M$\
of the predefined size $M$.
Stakeholders interact with each other and with locally installed ideal functionalities ${\mathcal{F}}_{LB}$,\
${\mathcal{F}}_{VRF}, {\mathcal{F}}_{L}$ over a sequence of $L = E \cdot R$ slots\
${S=\{sl_1,...,sl_L\}}$ consisting of $E$ epochs with $R$ slots each.
Let's clarify what the mentioned above pre-defined primitives are needed for:
\begin{enumerate}
    \item \emph{Ideal Leaky Beacon} ${\mathcal{F}}_{LB}$: is used to sample an epoch random seed from the\
    blockchain.
    \item \emph{Ideal Verifiable Random Function} ${\mathcal{F}}_{VRF}$: is used as a source of randomness.
    \item \emph{Lottery Function} ${\mathcal{F}}_{L}$: checks if the protocol participant is a lottery winner\
    (by lottery, we mean either the \emph{consensus group lottery} or the \emph{leader lottery} giving\
    the ability to start a batch notarization round, the lottery function in both cases remains the same,\
    only the arguments matter).
\end{enumerate}
More extended formal definition of ${\mathcal{F}}_{LB}$ and ${\mathcal{F}}_{VRF}$ can be found in the original\
Ouroboros Praos paper ~\cite{cryptoeprint:2017/573}.

Each new epoch ${e_j \gt 1}$ has a new consensus group and any protocol participant\
can try to become a member if he is verified.
Participant is verified if his verification key tuple is published and stored in the blockchain for a\
reliable period of time equals to $U_f$ slots.

At the end of the epoch ${e_j \gt 1}$ every verified $PK_i$ requests a\
new epoch seed $\eta_j$ from the ${\mathcal{F}}_{LB}$.
When every $PK_i$ evaluates ${\mathcal{F}}_{VRF}$ and passes the received proof $\pi$ to the ${\mathcal_{F}}_{L}$ to\
reveal the result of the consensus group lottery.
To calculate an appropriate threshold ${T_i^j}$, ${\mathcal{F}}_{L}$ should be parametrized with the same\
stake distribution which was in the last block used by ${\mathcal{F}}_{LB}$ to sample the new $\eta_j$.
Argument of the winning probability function $p$ is ${f = M /\/ N}$, where $M$ is a number of new consensus group\
members to select and $N$ is the total number of the verified stakeholders.

For security purposes, a lower bound number of committee members $M$ is required.
If the number of verified participants is less than ${M_{\min}=256}$, then the consensus group lottery is not held.
The maximum consensus group size is ${M_{\max}=1024}$, this limit arises due to the complexity of communication.

\textbf{Leader Lottery}.
Once a new consensus group is determined, the lottery process does not stop, but this time\
the leader of the group should be determined.

During an epoch, for each slot ${sl_l \in S}$, each member of the consensus group $PK_i$ separately evaluates\
${\mathcal{F}}_{VRF}$ with his own input ${x = \eta_j || sl_l || \emph{nonce}}$.
In response, he receives the associated random proof $\pi$.
If ${\mathcal{F}}_{L}(state, f, \pi) \rightarrow 1$ then $PK_i$ is a slot leader and he can propose a batch\
which should be notarized by at least two-thirds of committee members.

The parameter $f$ that regulates the probability of winning is different from the one used in the consensus\
group lottery.
Here, it is the pre-defined value determines how many slots will have at least one selected leader,\
it is called an active slots coefficient.

The lottery mechanism described in this subsection is fast, secure, and adaptive, since the pre-defined parameters\
can be changed via the voting process.
The same primitives are used to achieve different goals, namely, select a consensus group dynamically\
and select a slot leader.

Regarding the security it is important to note, that participants use their public VRF-keys for\
VRF functionality evaluation in the consensus group lottery and secret VRF-keys in the leader lottery.
This way, slot leaders don't become publicly known in advance.
An attacker can't see who is a slot leader until he initializes batch notarization, thus an attacker can't know\
who specifically to attack in order to control a certain slot.
Opening consensus group members on the over hand should be known ahead of time for the synchronization.
There are hundreds of consensus members in every epoch, so denial of service attacks are difficult to succeed.
Grinding attacks are limited because an adversary can't arbitrarily control $\eta_j$ values,
all he can try to do is to make as many forks as possible to estimate the most advantageous, but according to the\
analysis~\cite{cryptoeprint:2017/573} this advantage doesn't change the security properties of the entire protocol.



\subsection{Replacing MACs by Digital Signatures}\label{subsec:replacing-macs-by-digital-signatures}

The main issue with MACs is that any node capable of validating MAC is also capable of generating new messages with
valid MACs as the secret key used for MAC generation is also necessary for validation. Digital signatures, on the other
hand, use asymmetric protocols for signature generation and signature verification powered by public-key cryptography. A
valid secure digital signature for the message can only be generated with the knowledge of the secret key
(unforgeability requirement), and verified with the corresponding public key (correctness requirement), and the secret
key never leaves the signer's node. The authenticity of the message from the network node can be verified by any party
knowing the node public key. Moreover, given the full history of communication, the malicious actor is still not able to
forge the new message with valid signature of the node. This gives a way finer control over the set of permissions and
provides a strong authentication method.

Spectrum utilizes the specific subset of signatures based on so-called sigma-protocols. The benefits of these protocols
are numerous, including the possibility of proving complex logical statements inside the scheme, provable
zero-knowledge, and use of standardized and well-estabilished crypto-primitives, namely conventional cryptographic hash
functions and standard elliptic curves with hard discrete logarithm problem. This means the high level of support in the
existing chains without modification of the core opcodes or writing supplementary on-chain routines.

\subsection{Scalable Collective Signature Aggregation}\label{subsec:scalable-collective-signature-aggregation}

\subsubsection{Problem Statement and Review}

In this section we describe our approach to the following problem. The naive approach to writing the consensus values on
the blockchain in a verifiable way would be simply write the resulting values together with the signatures from every
node which successfully participated in the consensus protocol. Spectrum consensus groups can contain thousands of
nodes. If one takes Schnorr signature scheme with 256-bit keys, every signature is 64 bytes long. That means thousands
of kilobytes of data needed to be written on the blockchain and consuming valuable storage space, not speaking on the
computational efforts from the blockchain validating node to actually verify all these signatures. Therefore, a method
of signature aggregation is a must under these circumstances. The aggregation allows one to write a single shorter
signature instead of the list of signatures while preserving similar security level. There are few signature aggregation
schemes for the sigma-protocol based signatures, such as CoSi and MuSig. These protocols perform extremely well if all
the keys of the predifined set of co-signers are included in the resulting signature generation. In this case instead of
having thousands of separate signatures one has only one of the size of single Schnorr signature. But this is not the
case with many realistic situations with large consensus groups (such as Spectrum). It would be too optimistic to
assume that all the nodes are always online, and every single node is following the protocol honestly to every letter.
One needs the mechanism to process these failures. Whereas CoSi proposes the method to process such failures, it comes
at cost of significant increase in the size of the resulting signature. Our scheme relies on the similar ideas, however
we tend to provide better scaling with faulty nodes and more compact constructions than the original CoSi.

In short, we construct a compact aggregated signature scheme with potential node failures based on standard cryptographic
primitives. It must have constant small size in the absence of failures and provide reasonably small space and
computational overheads in the presence of failures. The signing protocol must be performed in a distributed fashion
providing the defence from the malicious co-signers.

\subsubsection{General Overview}

We start with the MuSig scheme and modify it to the meet the criteria listed above. We assume the Discrete Logarithm group to be the subgroup of the elliptic curve as usual. That is, elliptic curve is
defined over finite field, we consider subgroup of its points with coordinates in thes field of prime order with
fixed generator $g$ and identity element being the point at infinity if the curve is written in the form $y^2=f(x)$, $f$
is the third degree polynomial. Nothing prevents one from using another group with hard discrete logarithm problem. We use multiplicative notation for
the group operation, and the group elements except for generator are written in capital letters. The secret keys are the integers
modulo group order. We write them in small letters. $H$ is the cryptographic hash function. When we write $H(A,B)$, we
assume that there is a deterministic way of serializing the tuple $(A,B)$, and this serialization is used as an argument
for $H$. The public key corresponding to the private key $x$ is the group element $X=g^x$.

Any interactive sigma-protocol consists of three stages in strict order: commitment (when one or more group elements are sent from
prover to verifier), challenge (when the the random number is sent from verifier to prover), response (when one or more
numbers calclated from the previous stages and the secret key are sent from prover to verifier). This triple constitutes the proof-of knowledge of the secret key. To turn the interactive
protocol in non-interactive one, Fiat-Shamir heuristic is used, when the challenge is replaced by the hash value of all
the preceeding public data. 

The takeaways from this setting, which are important for the understanding of our construction are the following:
\begin{itemize}
    \item In case of $n$ nodes one must have $n$ commitments. They must be aggregated, and the list should not change
        till the end of the protocol
    \item As the commitments from different nodes come at potentially different time, there can be an attack on this
        stage. Say, one node does not pick the commitment based on random, but rather calculates it based on the
        commitments received from the other nodes. This kind of attack is known as $k$-list attack, as to forge the
        upcoming signatures the malicious node solves the $k$-list problem, which is tractable with the sufficient
        amount of data. To exclude this possibility one needs all the node to ``commit to Schnorr commitment''
        beforehand. One can use hash function with no homomorphic properties for that purpose.
    \item All the steps are strictly sequential. Hence, the every stage must complete with the full aggregation of
        individual contributions. There does not seem to be a simple way to perform it fully asynchronously.
    \item Instead of the last step (response) it is sufficient to provide the proof of knowledge for the response. This
        brings no additional value to the conventional signatures, but it helps wit the processing of the node failures
        during the execution. Namely, the consensus group may demonstrate that somebody in the group knew the discrete
        logarithms of the commitments not accounted for in the response stage. Therefore, the group as a whole could
        compute the full response if the failure had not occur.
    \item There must be the way to count the failures above, such that the signature verifier could decide whether it
        tolerates this number or not.
\end{itemize}

\subsubsection{Aggregation Rounds and Structures}
Here we list the overall structure of aggregation to give a grasp on the overall process. The detailed explanation is in
the subsection below.
\begin{itemize}
    \item Round 1. Collect Commitments for Schnorr commitments.  Structure: list of hashes of elliptic curve points
    \item Distribute all the hashes after aggregation
    \item Round 2. Collect and aggregate Schnorr commitments. Structures: list of signatures (proofs of discrete
        logarithms for the commitments) together with Schnorr commitments
    \item Distribute among all the nodes. Upon receiving every node verifies that the hashes of the points are those
        provided on round 1, and verifies the proofs of discrete logarithms. The commitments with the checks passed are
        aggregated to get the overall commitment. It is used to compute the challenge and the individual responses in
        the sigma--protocol.
    \item Round 3. Collect and aggregate the responses. Structure: list of individual responses. Upon receiving every
        individual response is verified. The responses which passed the verification are added together. If the response
        is invalid or missing, the corresponding discrete logarithm proof from round 2 is appended to the output.
    \item Output. Aggregate signature $(Y,z)$ together with the set
        \[
            \{(Y_i, DlogProof(Y_i)\}\,,
        \]
        where $i$ rubs over the set of node which have not provided valid responses.
\end{itemize}

\subsubsection{Signature Generation}
\begin{enumerate}
    \item Each signer computes $a_i\leftarrow(X_1,X_2,\dots,X_n;X_i)$ and the aggregate public key $\tilde{X}\leftarrow\prod_i X_i^{a_i}$
    \item Each signer generates a pair $Y_i=g^{y_i}$ to commit to, commitment $t_i\leftarrow H(Y_i)$ and the signature $\sigma_i$ of
        some predefined message with secret key $y_i$.
    \item The commitments $t_i$ are aggregated in the list $L_1$
    \item After every participating co--signer received $L_1$, the tuples $(Y_i,\sigma_i)$ are aggregated in the list
        $L_2$.
    \item Upon receiving the tuple $(Y_i,\sigma_i)$, verify $t_i = H(Y_i)$, and verify that $\sigma_i$ is a valid
        signature corresponding to $Y_i$. The failed records are excluded form $L_2$, the next steps and communication round.
    \item Every node computes the aggregate commitment $Y=\prod_i Y_i$ using all the valid records in $L_2$.
    \item Every node computes the challenge $c\leftarrow H(\tilde{X}, Y, m)$ and the responses $z_i\leftarrow y_i + ca_ix_i$
    \item The responses $z_i$ are aggregated into list $L_3$.
    \item Initialize $z\leftarrow 0$ and empty set $R\leftarrow\{\}$.
    \item Upon receiving the response $z_i$, verify that $g^{z_i} = Y_i X_i^{a_ic}$. If this is the case, set $z\leftarrow
        z+z_i$. Otherwise, insert corresponding entry from $L_2$ in $R$ as $(i, Y_i, \sigma_i)$.
    \item Output triple $(Y,z,R)$
\end{enumerate}

\subsubsection{Signature Verification}
\begin{enumerate}
    \item Compute $a_i\leftarrow H(X_1,X_2,\dots,X_n;X_i)$
    \item Compute $\tilde{X}\leftarrow\prod_i X_i^{a_i}$
    \item Compute $X' = \prod_{i\notin R.0}X_i^{a_i}$
    \item Compute $Y' = \prod_{i\in R.0} Y_i$
    \item Compute $c\leftarrow H(\tilde{X}, Y, m)$
    \item Verify $g^z=X'^cYY'^{-1}$
    \item Verify all of $\sigma_i\in R.2$ with respect to $Y_i\in R.1$
    \item Compare the $n-k$ (where $k$ is the size of $R$) with the required threshold value
\end{enumerate}

\subsection{Consensus}\label{subsec:consensus}
Validators in the Spectrum network consist of several committees, individually selected via the lottery mechanism.
Number of committees is equal to the number of the currently connected $L_1$ blockchains.

The exact number of members in each committee may vary, but there are always hundreds of active validators for every\
local chain.
We will denote a committee with $\mathbf{C}_i$, where $i$ is the index of the certain $c_i$ connected chain.
At any given moment of time each committee $\mathbf{C}_i$ observes events on the local $c_i$ and\
on the Spectrum ledger $L^+$.
Assets are managed by the Network are stored on-chain in vaults.
Each asset type is controlled by local committees using via the aggregated signature mechanism.

\subsubsection{Processing events and transactions}

The main feature of the Spectrum is a native cross-chain interoperability, thus, more complex transactions take\
place here than on other networks.
Every committee of validators observes events in the local chains and in the Spectrum network.
There are 3 types of possible events coming from the external systems:
\begin{enumerate}
    \item \textbf{Inbound Value Transfer}: a box appeared in the local chain $c_i$, the equivalent value of which\
    should be transferred to the Spectrum network;
    \item \textbf{Outbound Value Transfer}: transfer of value from the Spectrum network into the external\
    system $c_i$ should be certified by the relevant local committee $\mathbf{C}_i$;
    \item \textbf{Box Elimination}: outbound value is reliably settled on the external chain $c_i$ and the\
    corresponding box in the Spectrum network must be eliminated.
\end{enumerate}

Incoming events from every $c_i$ are grouped into batches and executed independently\
by each local committee $\mathbf{C}_i$.
Therefore, the design of the Spectrum network provides 2 global types of transactions:
\begin{enumerate}
    \item \textbf{Batch Applying Transaction}: a transaction consisting of operations corresponding to the\
    execution of the incoming events: \textit{Inbound Value Transfers}, \textit{Outbound Value Transfers}\
    and \textit{Boxes Eliminations};
    \item \textbf{Programmable Transaction}: any programmable transaction within the Spectrum network.
\end{enumerate}

Spectrum ledger $L_s$ is based on the eUTXO model, we will denote transactions in the usual manner:
\begin{equation}
{Tx_{(in | out)}[Spent\ Boxes \rightarrow Unspent \ Boxes]}
    ,
\end{equation}
where ${in, out}$ stands for \textit{inbound} or \textit{outbound} transaction type respectively.
Box designations can be as follows: ${Box_{owner}^{(s | l | e)}(value)}$, where ${s, l, e}$ stands\
for the Spectrum box status.
Designation $s$ is \textit{available} Spectrum box, $l$ is \textit{locked} Spectrum box and $e$ is\
\textit{eliminating} Spectrum box.
Moreover, due to the design of the system, boxes can have different status that determine\
the range of actions that are valid for them.

Let's take an example of the cross-chain swap, which will allow us to distinctly follow\
the statuses and limitations follows from the above.

Alice wants to swap her assets from the blockchain $c_1$ to the blockchain $c_2$ using the Spectrum network\
and she has valid addresses on all networks.
Let's also assume that there is a box ${Box_{S}^s(X_{c_2})}$ available for spending in the liquidity of the\
Spectrum network ($S$ is the owner, i.e. Spectrum).
The swap flow will be as follows:
\begin{enumerate}

    \item Alice transfer her assets $X_{c_1}$ into the Spectrum vault an the inbound box ${Box_{A}(X_s)}$ with\
    the corresponding value $X_s$ is formed.
    Committee $\mathbf{C}_1$ observes this \textit{Inbound Value Transfer} event and\
    includes it into the \textit{Batch Applying Transaction}.

    \item Committee $\mathbf{C}_1$ processes the \textit{Batch Applying Transaction} and after adding\
    the block into the $L^+$ the box $Box_{A}^l(X_s)$ appears in the Spectrum network in the \textit{locked status}.
    The locked status remains until the time corresponding to the reliability confirmation threshold $R_1$ on\
    the network $c_1$ has passed.

    \item Once $R_1$ time is passed and status of the box is changed to $Box_{A}^s(X_s)$,\
    the \textit{Outbound Value Transfer} event is formed to transfer the $X_{c_2}$ value to Alice on the chain $c_2$.
    This event is observed by the committee $\mathbf{C}_2$ and is included in their \textit{Batch Applying Transaction}.
    After processing the $X_{c_2}$ value is transferred from the ${Box_{S}^s(X_{c_2})}$ according to the generated\
    certificate and the $Box_{A}^e(X_s)$ goes to the \textit{eliminating status}.
    This status will be preserved until enough time corresponding to the reliability confirmation\
    threshold $R_2$ on the network $c_2$ has passed.

    \item Once $R_2$ time is passed, the \textit{Box Elimination} event is formed,\
    observed by the $\mathbf{C}_2$, included into the \textit{Batch Applying Transaction} and processed.
    The box $Box_{A}^e(X_s)$ is eliminated from the next block and cross-chain swap is considered successfully completed.

\end{enumerate}

\subsubsection{Consensus assumptions}
It follows from the above description that in order to achieve consensus in cross-chain translation,\
synchronization between all committees is necessary only for certain actions, thus:
\begin{itemize}
    \item \textbf{Global Synchronization}: is needed to process \textit{Programmable Transactions} and to update\
    the status of boxes in the Spectrum network, i.e. \textit{locked} to \textit{available}, \textit{available} to\
    \textit{eliminating} or \textit{eliminating} to \textit{available};
    \item \textbf{Local Synchronization}: is needed between local committee to process the\
    \textit{Batch Applying Transaction}, this type of transaction can be processed separately by all committees.
\end{itemize}

\subsection{Ledger}\label{subsec:ledger}
Spectrum network is complex since it works with multiple connected blockchains.
In order to accurately describe and verify all the necessary properties of the Spectrum ledger, we should agree on the\
semantics and introduce data types that satisfy some validity conditions which we will describe in this section.

\subsubsection{Model formalization}
We will start with a semantics needed to describe our system:
\begin{itemize}
    \item \textbf{State}: the underlying data type of the Spectrum ledger, the validity of which must be proven.

    \item \textbf{Local state}: auxiliary data type describing the state on one of the connected chains.

    \item \textbf{Transition}: the means by which system is moved from one valid \textit{state} to another valid\
    \textit{state} or from one valid \textit{local state} to another valid \textit{local state}.

    \item \textbf{Signal}: the means by which \textit{transitions} are triggered.

    \item \textbf{Rule}: algorithm that describes how to derive a valid \textit{states},\
    \textit{local states} or \textit{transitions}.The rule has an \textit{antecedent}\
    (sometimes called \textit{premise}) and a
    \textit{consequent} (sometimes called \textit{conclusion}), such that if the
    conditions in the antecedent hold, the consequent is assumed to be a valid
    \textit{state}, \textit{local state} or \textit{transition}.

    \item \textbf{Environment}: sometimes network may implicitly rely on some additional information being present\
    in the system to evaluate the \textit{rules}.
    An environment does not have to exist (or, equivalently, we may have an empty environment), but crucially\
    an environment is not modified by \textit{transitions}.
\end{itemize}

The whole set of the above concepts forms a state transition system ${\Psi = (S, L, T, \Sigma, R, E)}$,\
where $S$ is a set of \textit{states}, $L$ is a set of \textit{local states}, $T$ is a set of \textit{transitions},\
$\Sigma$ is a set of \textit{signals}, $R$ is a set of \textit{rules} and $E$ is a set of \textit{environment} values.

\subsubsection{State Transition System}
todo

\subsection{Eliminating Validator Bottleneck}\label{subsec:eliminating-validator-bottleneck}
So far each member of the consensus group had to track changes on all connected chains in order to participate in\
consensus properly.
However, this approach reduces the number of possible consensus participants and limits the scalability of the system.
Therefore, for the optimal design of our consensus protocol, we will use the following observations:
\begin{itemize}
    \item[]\textbf{Observation 1:} Events coming from independent chains $c_k$ are not serialized.
    \item[] \textbf{Observation 2:} Outbound transactions on independent systems $c_k$ can be independently signed.
\end{itemize}

Utilizing those properties we now introduce committee sharding.
We modify protocol in a way such that at each epoch $e_n$, $K$ distinct committees consisting of nodes equipped with\
functionality unit $F_{c_k}$ relevant to a specific connected chain $c_k$ are selected via the consensus group lottery.
All primitives using in the lottery are equal for different committees, however, lotteries are independent.

We denote one such committee shard as $V_{n}^k$, which uniquely maps to $c_k$.
Then, complete mapping of committees to chains at epoch $e_n$ can be represented as a set of tuples\
committee-chain $\{(V_{n}^k, c_k)\}$.
Throughout epoch $e_n$ all events and on-chain transactions in $c_k$ are handled exclusively by $V_{n}^k$.
Nodes in $V_{n}^k$ maintain a robust local ledger $L^{local}_k$ of notarized batches of events observed in $c_k$.

\subsubsection{Syncing Shards}

Each committee $V_n^k$ forms the notarized batches of events and them into their local ledgers $L_k$.
All these batches should be periodically synced and added to a block of the one main super ledger $L^+$\
in order for the system to be able to compute cross-chain state transition.
To facilitate this process, batches of the notarized events should be broadcast to other committees.
The main actors at this stage are:
\begin{enumerate}
    \item \emph{Local leader}: local committee leader.
    \item \emph{Relayer}: any protocol participant, who broadcasts notarized batches to the local leader\
    and to other committees' members.
    Every \emph{Local leader} can be a relayer at the same time.
    \item \emph{General leader}: one of the local leaders who added a block consisted of collected\
    notarized batches and other internal transactions to the $L^+$.
\end{enumerate}

There is no any separate lottery for the general leadership and any local leader is able to publish his\
block to $L^+$, thus, he can choose from two main strategies:
\begin{enumerate}
    \item \emph{Wait}: malicious strategy where local leader waits for broadcasts from other committees\
    members and don't broadcast his own batch to eliminate competitors for adding a block.
    \item \emph{Broadcast and wait}: fair strategy where local leader immediately broadcasts his batch,\
    waits for broadcasts from other committees members and honestly adds a block.
\end{enumerate}
Thus, there should be a motivation for the individual local leader to choose the fair strategy instead of keeping\
his batch for too long and there also should be a motivation for every committee member to act as a relayer.
This is achieved through the design of the incentive system.

\subsubsection{Incentives}

There are three types of the incentive for the Spectrum protocol participants: ${\{R_b, R_d, R_m\}}$, where $R_b$ is a\
guaranteed reward for adding a notarized batch to the block, $R_d$ is given for broadcasting a batch to the\
general leader and $R_m$ is given personally to the general leader who finally added the block.
Delivery reward $R_d$ is given if and only if a delivery was made within a predetermined period of time $\Delta t_d$.

Reward amounts are initially configured in such a ratio that if ${R_d=0}$ there is no prior strategy for\
local leaders, they will either wait for other batches or broadcast their batches with equal probability.
At the same time, all other committee members are still motivated to act as a relayers to receive an extra reward,
since the notarized batch can be firstly generated by any member of the committee.
All the rewards except $R_m$ are shared equally between all committees members whose signatures are included in\
the finally added block.

As a result, the syncing shards flow looks as follows:
\begin{enumerate}
    \item After notarization, the committee member holding the notarized batch, which contains the local\
    notarization time $t^*$, sends it to his local leader and to other known committees members.
    \item All committees members who receive notarized batches from other committees also send it\
    to the local leader.
    \item The local leader collects the received notarized batches.
    \item When waiting time approaches $\Delta t_d$, local leader forms a block\
    from all collected batches and broadcast it.
    Block contains the set of the notarization times and block creation time $t^B$.
    \item After block is reliably settled in the $L^+$, all associated participants can claim their rewards.
\end{enumerate}

We also introduce another type of authority incentive that increases the chances of participants\
to be selected in the consensus group lottery.
When calculating the steak distribution, which is needed to parametrise the lottery function,\
all stakes are weighted depending on the actions of their holders in the previous epoch,\
i.e. ${s_i = A_m \cdot s_i^{real}}$, where $A_m$ is an authority multiplier.
If some authority was a member of a committee and participated in adding of 2/3 of the blocks produced in\
the considered period of time, then his actual stake ${s_i^{real}}$ is multiplied by ${A_m = 2}$.
Multiplier $A_m$ decreases linearly to 0, which is the case where member was passive during the entire epoch.

With this mechanism, we solve the following problems:
\begin{itemize}
    \item Members are motivated to be focused on cooperation with other committees\
    so that their participation is reflected in each block added in the $L^+$.
    \item Inactive and dishonest members are automatically excluded from the next epoch committee.
    \item Participants are motivated to stay active throughout the entire epoch so that their chances of being\
    selected in the committee don't decrease due to an authority multiplier ${A_m < 1}$, otherwise,\
    in order to even the odds with new lottery participants, they will either have to increase\
    their real stake, or skip the lottery until the next one.
\end{itemize}



\subsection{Ledger Model}\label{subsec:ledger-model}
Spectrum's global state includes a pool of value carrying units called \emph{cells}.
A \emph{Cell} encodes monetary value (e.g., fungible or non-fungible tokens)\
travelling inside the system and across its boards.

\begin{center}
    \begin{tabular}{ | r l | }
        \hline
        TxId =   & H(Tx)                    \\
        CellId = & H(TxId \times \text{ I}) \\
        \hline
    \end{tabular}
\end{center}

Each cell has a unique identifier derived from ID of the transaction that produced the\
cell and its index in the transaction outputs.
The identifier remains stable even when cell is modified as we explain below.

\begin{center}
    \begin{tabular}{ | r l | }
        \hline
        Value =         & \blue \texttt{u64}                                                      \\
        ChainId =       & \blue \texttt{u64}                                                      \\
        Version =       & \blue \texttt{u64}                                                      \\
        ProgressPoint = & ChainId \times \blue \texttt{ u64}                                      \\
        ActiveCell =    & CellId \times \text{ Address} \times \text{Value} \times \text{Version} \\
        BridgeInputs =  & [\blue \texttt{u64}]                                                    \\
        Destination =   & ChainId \times \text{ BridgeInputs}                                     \\
        TermCell =      & CellId \times \text{ Value} \times \text{Destination}                   \\
        Cell =          & ActiveCell \uplus \text{ TermCell}                                      \\
        \hline
    \end{tabular}
\end{center}

We distinguish two essential types of cells depending on the state of the value they encode.

\subsubsection{Active cells}\label{subsubsec:active-cells}

\emph{Active Cell} is a value travelling between owners inside the system.
An Active Cell can be modified while preserving its original stable identifier.
With each mutation version of the cell is incremented which is initialized with \texttt{0} when the cell is created.
This opens the door for smooth management of shared cells (e.g., stablecoin bank or liquidity pool).

\subsubsection{Authenticators, Addresses and Ownership}\label{subsubsec:authenticators-and-addresses}

\begin{center}
    \begin{tabular}{ | r l | }
        \hline
        Authenticator = & ProveDlog \uplus \text{ Script} \\
        Address =       & H(Authenticator)                \\
        \hline
    \end{tabular}
\end{center}

Each active cell has an exclusive owner identified by an address.
Address is derived from an authenticator by applying collision resistant hash function to it.
To prove ownership of a cell a party must supply an authenticator whose hash matches the owning address.
An authenticator can either be a public key or a script.
Once authenticated an owner can freely move value locked within the cell by either mutation or elimination of it.

\subsubsection{Terminal cells}\label{subsubsec:terminal-cells}

Terminal cells encode value to be exported into an external system.
In contrast to active cells, terminal cells are immutable and value from them cannot be moved within the system anymore.

\subsubsection{Transactions and Effects}\label{subsubsec:transactions-and-effects}

\begin{center}
    \begin{tabular}{ | r l | }
        \hline
        Imported =   & ActiveCell                                                                      \\
        Exported =   & CellId                                                                          \\
        Revoked =    & CellId                                                                          \\
        Progressed = & ProgressPoint                                                                   \\
        Eff =        & Imported \uplus \text{ Exported} \uplus \text{Revoked} \uplus \text{Progressed} \\
        \hline
    \end{tabular}
\end{center}

Global pool of cells is modified by atomic state modifiers called \emph{Effects} and \emph{Transactions}.

Effects are state transitions imported from external systems exclusively by local committees.
Below we list possible effects:
\begin{enumerate}
    \item Import of value.
    A deposit into one of Spectrum's on-chain vaults which results in creation of a new cell.
    \item Export of value.
    An outbound transaction that transfers value from Spectrum's on-chain vault to user address on particular blockchain.
    \item Revocation of previously imported value due to roll-back on the source chain.
    \item Signalisation that external system reached particular progress point.
\end{enumerate}

\begin{center}
    \begin{tabular}{ | r l | }
        \hline
        CellRef =          & CellId \times \text{ Version}                                  \\
        Inputs =           & CellRef \times \text{ [}\text{CellId} \uplus \text{CellRef}]   \\
        RefInputs =        & [Cell]                                                         \\
        EvaluatedOutputs = & [Cell]                                                         \\
        Tx =               & Inputs \times \text{ RefInputs} \times \text{EvaluatedOutputs} \\
        \hline
    \end{tabular}
\end{center}

In contrast to effects, transactions are state transitions triggered by Spectrum users.
A transaction accepts cells that it wants to mutate or eliminate as inputs and\
outputs new cells or upgraded versions of mutated cells.
Therefore, scope of transaction is restricted to its inputs and outputs.

\textbf{Transactions: Referencing inputs.} Transaction can reference cells to use as inputs either by\
cell ref (fully qualified reference) or only by stable identifier.
In the latter case, a concrete version of the cell with the given stable identifier will be resolved in\
the runtime of the transaction.
Importantly, each transaction must have at least one fully qualified input, this guarantees that each\
transaction is unique.

\textbf{Transactions: Programmability.} Some outputs may be computed in the runtime of a transaction as\
a result of script(s) execution.
It is also possible to include pre-evaluated outputs into transaction in order to save on on-chain computations.
This design allows dApp developers to choose the amount of on-chain computations of their apps.

\subsubsection{Dealing with finality of imported value}\label{subsubsec:dealing-with-finality-of-imported-value}

Beacause Spectrum is a cross-chain system, monetary value there is usually imported from an external system\
(e.g.\ Cardano or Ergo).
Since most of the cryptocurrencies don't provide instant finality of transactions, on-chain transaction\
that once imported value into Spectrum's on-chain vault may be rolled-back.
There are two ways of preventing \enquote{dangling} value inside Spectrum.
On the one end of spectrum is a conservative approach: wait for settlement on the source chain\
(e.g.\ 120 blocks in Ergo) before import to be 100\% sure the transaction will not be rolled back.
On the other end is a reactive approach: import value immediately and revert locally transactions that\
depend on that piece of value in the case of rollback.
Conservative approach offers simplicity and is cheaper to execute, while reactive one allows to work\
with imported value inside spectrum with minimal delays.

\textbf{Observation:} Probability of a rollback at a certain height decreases exponentially with square\
root scaling in the exponent as chain extends ~\cite{cryptoeprint:2017/573}.

Based on this observation we choose a hybrid approach.
Value is imported with a small delay $D^c$ which is configured for each chain and\
is sufficient to keep probability of rollback low.
If rollback happens after the import all transactions directly or transitively depending\
on the dangling value are reverted.

As long as outbound transactions can not be reverted it is of paramount importance to wait\
for complete settlement of the imported value before allowing to export it.
Each cell is associated with a set of dependencies called \emph{ancors} represented\
as unique identifier of a chain and a height which the chain is required to reach in order\
for the anchor to be deemed as \emph{unancored}.
Active ancors leak from cells in inputs into created cells in outputs.
It is impossible for a terminal cell to be exported until all ancors it depends on are reached.

\subsection{System integrity}\label{subsec:system-integrity}
\subsubsection{Forks}\label{subsec:resolving-forks}

Protocol flow implies that any of the local leaders can append their blocks to $L^+$, which leads to forks.
This type of forks is a normal part of the protocol lifecycle, however, total possible number of the normal forks in\
our protocol is much larger than in other blockchains, since there can also be a several local leaders in every\
connected $L_k$ committee.
The chance of occurring a malicious forks produced by adversary is minimized by the lottery design.
In addition, the task for an adversary becomes more difficult by virtue of the interaction between the protocol\
participants during the Syncing Shards process.

The main rules for resolving forks are simple and are performed by the members of all committees when\
validating a proposed blocks:
\begin{itemize}
    \item \textbf{Max valid}: choose the longest appropriate chain given a set of valid chains that are available\
    in the network.
    The depth of any block broadcast by a protocol member during the protocol must exceed the depths of any\
    honestly-generated blocks from slots at least $K$ in the past.
    \item \textbf{Max stake}: if the \emph{Max valid} rule doesn't resolve a slot battle, then the valid chain\
    chooses according to the stake size of the battled leaders, the maximum stake is the winner.
    Stake distribution is picked from the actual blockchain snapshot for the current committee.
\end{itemize}

A large number of the normal forks, however, still significantly affect properties, that maintain the\
integrity of the $L^+$:
\begin{itemize}
    \item \textbf{Latency}: the number of elapsed slots required for a transaction to appear in a block on the $L^+$.
    \item \textbf{Finality}: the number of elapsed slots required for a transaction to become settled and immutable.
\end{itemize}
The Latency of the protocol is good enough due to the short duration of the slots.
Finality is guaranteed after $K_F$ slots, where $K_F$ is a pre-defined protocol parameter.
As a result of the functional features of our protocol, $K_F$ depends on the connected $L_k$ integrity properties.

Most ledgers do not guarantee instant finality of transaction, that means that any (or all) transactions may not\
be applied to corresponding $L_k$ ledgers in the end.
Different blockchains however has different Finality parameters, and time of elapsing $K_F$ should be longer\
than all of them.
Thus, the $K_F$ should be set with a margin and therefore using the number of slots $\Delta Sl$ that have\
passed in the Spectrum network, developers should be able to receive information about the number of blocks that\
have passed in all connected $L_k$ blockchains during this period of time.
The duration of the block in each $L_k$ is different, but the average values are preserved for a certain period of\
time ${\Delta T >> d_s}$, where $d_s$ is the duration of Spectrum's slot.
Thus, after each $\Delta T$ time interval, Spectrum network will update the set of constants:\
${(\{d_{k}\}_{k=1}^{K},\{c_{k}\}_{k=1}^{K})}$, where $d_k$ is a block duration in the $L_k$, $c_k$ is the default\
reliable number of confirmations in the $L_k$, $M$ is the total number of the connected $L_k$.

Using the data above, each Spectrum's $\Delta Sl$ can be associated with the delta of blocks that have passed in\
any connected blockchain: ${\{\lfloor \Delta Sl \cdot d_s \mathbin{/} d_k)\rfloor\}_{k=1}^{K}}$.
When forming transaction, developers can specify a reliability factor $C$.
This factor will be compared with the ratio of the number of blocks passed on the associated $L_k$ blockchain to\
the default reliable number of confirmations $c_k$ of this network:
\begin{equation}
    \theta(k-L_k^{id})\cdot \left\{\frac{1}{c_k} \cdot \left\lfloor \Delta Sl \cdot
    \frac{d_s}{d_k}\right\rfloor\right\}_{k=1}^{K} >= C,\label{eq:equation2}
\end{equation}
where $\theta(x)$ is an indicator function which is 1 at $x = 0$, otherwise 0.

\subsubsection{Handling Rollbacks}\label{subsec:rollbacks}
todo


\subsection{Protocol Flow}\label{subsec:protocol-flow}
let's summarize all of the above and describe the full flow of the Spectrum protocol.
Protocol is running by manually selected opening consensus group $V_0$.
Stakeholders interact with each other and with the ideal functionalities ${\mathcal{F}}_{RLB}$,\
${\mathcal{F}}_{VRF}$, ${\mathcal{F}}_{KES}$, ${\mathcal{F}}_{DSIG}$ over a sequence of $L = E \cdot R$ slots\
${S=\{sl_1,...,sl_L\}}$ consisting of $E$ epochs with $R$ slots each.

\subsubsection{Bootstrapping}\label{subsubsec:bootstrapping}

The system is bootstrapped in a trusted way.
A manually picked set of validators $V_0$ of the predefined size $M$ is assigned to the first epoch $e_0$.
\begin{enumerate}
    \item On-chain vaults are initialized with an aggregated public key $aPK_0$ of the initial committee.

    \item All consensus group members i.e. $\forall PK_i \in V_0$ should generate the tuple of verification keys\
    ${(v_i^{vrf}, v_i^{kes}, v_i^{dsig})}$, using the ideal functionalities ${\mathcal_{F}}_{VRF}$,\
    ${\mathcal{F}}_{KES}$, ${\mathcal{F}}_{DSIG}$ instances, running on their machines.

    \item Full set of the verification\
    keys tuples ${V_{init} = \{(PK_i, v_i^{vrf}, v_i^{kes}, v_i^{dsig})\}_{i=0}^M}$ with the initial stakes $\{s_i\}_{i=0}^M$
    must be stored in the blockchain and acknowledged by all members of the initial consensus group.

    \item ${\mathcal{F}}_{LB}$, parameterized with confirmed $V_{init}$ is evaluated independently by every\
    participant to sample an initial random seed value $\eta \leftarrow \{0, 1\}^\lambda$.

    \item Finally, all approved stakeholders should agree on the genesis block\
    ${B_0=\left(\{(PK_i, v_i^{vrf}, v_i^{kes}, v_i^{dsig}, s_i)\}_{i=0}^M, \eta\right)}$.
\end{enumerate}

\subsubsection{Normal Flow}\label{subsubsec:normal-flow}
Once the system is bootstrapped, protocol is running in a normal flow:
\begin{legal}
    \item \textbf{Registration}.
    Any Spectrum stakeholder can register to become a committee member of his local chain $c_i$.
    To get a chance to be included in the set of validators $\mathbf{C}^n_i$ in the epoch $e_n$\
    participant $PK_i$ should register in the lottery during the epoch $e_{n-2}$ by publishing his verification tuple\
    ${(v_i^{vrf}, v_i^{kes}, v_i^{dsig})}$ into the $L^+$.
    Once enough time (number of slots) has elapsed to meet the reliable confirmation threshold $R^+$ for the $L^+$,\
    the participant is considered as \textit{verified}.

    \item \textbf{Consensus Lottery}.
    At the end of the epoch ${e_j \geq 1}$ every \textit{verified} $PK_i$ receives new epoch seed $\eta_n$\
    from the ${\mathcal{F}}_{LB}$.
    When every $PK_i$ evaluates ${\mathcal{F}}_{VRF}$ with the input, which includes new $\eta_n$\
    and passes the received proof $\pi_i$ to the ${\mathcal{F}}_{L}$.
    Function ${\mathcal{F}}_{L}$ is parameterized with the $\mathbf{C}^i$ lottery params and uses the same\
    stake distribution which was in the last block used by ${\mathcal{F}}_{LB}$ to calculate the threshold.
    If successful, i.e. ${\mathcal{F}}_{L}$ returns $1$, then publish the associated proofs into the $L^+$.
    Functionality ${\mathcal{F}}_{LB}$ is parameterized with the history, including blocks with release times up to\
    ${-R^+}$ from the actual slot.
    Therefore, even in case of a rollback, the currently selected members of the consensus group remain legitimate

    \item \textbf{Committee key aggregation}.
    Once new committee is selected, nodes in the $\mathbf{C}_i^n$ aggregate their individual public keys $\{PK_i\}$ into
    a joint one $aPK_n$, which is needed to sign the \textit{Batch Applying Transactions} with the external events\
    (\textit{Inbound Value Transfers}, \textit{Outbound Value Transfers}, \textit{Boxes Eliminations}).

    \item \textbf{Committee transition}.
    Nodes in the $\mathbf{C}_i^{n-1}$ publish cross-chain message ${M_n : (aPK_n, \sigma_{n-1})}$ , where $aPK_n$ is\
    an aggregated public key of the new committee $\mathbf{C}_i^n$ , $\sigma_{n-1}$ is an aggregated signature of
    $M_n$ such that ${Verify(\sigma_{n-1}, aPK_{n-1}, Mn) = 1}$.
    Vaults are updated such that ${Vault\{(e_{n-1}, aPK_{n-1})\} \coloneqq(e_n, aPK_n)}$.

    \item \textbf{Chain extension}.
    \begin{legal}

        \item Every online consensus group member collects en existed chains related to the $L^+$ and verifying that for every chain every block,\
        produced up to $R^+$ blocks before contains correct data about the corresponding slot $sl'$ leader $PK'$.
        To verify a valid slot leader, responses from the ${\mathcal{F}}_{VRF}$ and ${\mathcal{F}}_{L}$ with the relevant inputs must equal $1$. Leader $PK'$ must be also a member of the committee, legal at $sl'$, this is checked in the same way. All forks are resolved by the rules of the longest chain and the largest stake in the corresponding priority.

        \item During the epoch, for every slot $sl$ every committee $\mathbf{C}_i^n$ member $PK_i$ separately evaluates\
        ${\mathcal{F}}_{VRF}$ with his own input ${x = \eta_n || sl || \textbf{nonce}}$\, where \textbf{nonce} is an arbitrary predefined string.
        If successful ${\mathcal{F}}_{L}$ returns $1$, then $PK_i$ is the slot $sl$ leader he evaluates ${\mathcal{F}}_{VRF}$ one more time with the input ${x' = \eta_n || sl || \textbf{test}}$, where \textbf{test} is another arbitrary predefined string.
        The associated proofs $\pi_i$ and $\rho_i$ will be included in the block, which will be added to the $L^+$.

        \textbf{Note:} The ${\mathcal{F}}_{VRF}$ is designed in such a way that not every slot has a leader,\
        moreover, most of the slots remain empty to serve protocol synchronization.
        If there are $P$ several elected leaders for this slot, they all can add new blocks
        $\{B_p\}_{p=0}^P$ with included proofs ${(v_i^{vrf}, \pi_i, \rho_i)}$.

        \item All committee $\mathbf{C}_i^n$ members observe events in their local chains $c_i$ and in the $L+$ mempool.
        If $PK_i$ is a slot $sl$ leader, then he is able to propose a batch $B^i_j$ of events from $c_i$ in a form of the \textit{Batch Applying Transaction}, which should be notarized by other members of the $\mathbf{C}_i^n$ with an aggregated signature.

        \item Notarized batch $B^i*_j$ can first of all be formed by any member of the $\mathbf{C}_i^n$.
        He should immediately send it to the leader, who proposed it and to the members of other committees $\mathbf{C}_{j\neq i}^n$.
        After the leader receives set of batches ${B^k*_j}$, he forms the block $\mathbf{B}_m$ with updates from the $L^+$ mempool included, sign it with ${\mathcal{F}}_{KES}$, includes all proofs and broadcasts it to all committees.

        \item After $R^+$ reliable slots number has passed all members of all committees that participated in the formation of the block $\mathbf{B}_m$ receive rewards proportional to their status.
    \end{legal}
\end{legal}
