This section presents Spectrum protocol design starting from a naive approach based on PBFT and gradually addressing the challenges.

\subsection{Strawman Design: PBFTNetwork}\label{subsec:strawman-design}

For simplicity we begin with a notarization protocol based on PBFT, then iteratively refine it into Spectrum.

PBFTNetwork assumes that a group of ${n = 3f + 1}$ trusted nodes has been pre-selected upfront and fixed and at most $f$ of these nodes are byzantine.
At any given time one of these nodes is the \emph{leader}, who observes events on connected blockchains,
batch them and initiate round of notarization within the consensus group.
Remaining members of the consensus group verify the proposed batches by checking the presence of updates on corresponding blockchains.
Upon successful verification each node signs the batch with its secret key and sends the signature to the leader.

Under simplifying assumptions that at most $f$ nodes are byzantine the PBFTNetwork guarantees livness and safety.
However, the assumption of a fixed trusted committee is not realistic for open decentralized systems.
Moreover, as PBFT consensus members authenticate each other via non-transferable symmetric-key MACs, each consensus
member has to communicate with others directly, what results in $O(n^2)$ communication complexity.
Quadratic communication complexity imposes a hard limit on scalability of the system.

In the subsequent sections we address these limitations in three steps:
\begin{enumerate}
    \item \textbf{Opening consensus group.} We introduce a lottery-based mechanism for selecting consensus group dynamically.
    \item \textbf{Replacing MACs by Digital Signatures.} We replace MACs by digital signatures to make authentication transferable
    and thus opening the door for sparser communication patterns that can help to reduce the communication complexity.
    \item \textbf{Scalable Collective Signature Aggregation.} We utilize Byzantine-tolerant aggregation protocol that allows for
    quick aggregation of cryptographic signatures to reduce communication complexity to $O(\log n)$.
\end{enumerate}

\subsection{Opening Consensus Group}\label{subsec:opening-consensus-group}
todo

\subsection{Replacing MACs by Digital Signatures}\label{subsec:replacing-macs-by-digital-signatures}

todo

\subsection{Scalable Collective Signature Aggregation}\label{subsec:scalable-collective-signature-aggregation}

todo

\subsection{State Transition Function}\label{subsec:ledger-state-transition}
Usually State Transition Function (STF) of a ledger looks like: ${apply: (S, T) \rightarrow S'}$,
where $S$ - current state of the ledger, $T$ - a set of transactions, $S'$ - resulting state of the ledger.

Spectrum's STF as long as it operates partially on top of other ledgers, can be viewed as:
\begin{equation}
    apply:(S, S_O, T_I) \rightarrow (S', T_O),\label{eq:equation1}
\end{equation}
where $S$ - current Spectrum's state, $S_O$ - observed outbound state of connected ledgers, $T_I$ - a set of inbound transactions, $S'$ - resulting state of Spectrum's ledger, $T_O$ - resulting set of outbound transactions that must be settled on connected L1s.

\subsubsection{Achieving finality of outbound transactions}
Most ledgers do not guarantee instant finality of transaction, that means that any (or all) transactions of $T_O$ may not be applied to corresponding ledgers in the end.
Also the number of confirmations in different networks have different reliability.

Using the number of epochs $\Delta E$ that have passed in the Spectrum network, developers should be able to receive information about the number of blocks that have passed in all connected L1 blockchains during this period of time.
The duration of the block in each L1 blockchain is different, but the average values are preserved for a certain period of time ${\Delta T >> d_s}$, where $d_s$ is the duration of Spectrum's epoch.
Thus, after each $\Delta T$ time interval, Spectrum network will update the set of constants: ${(\{d_{i}\}_{i=1}^{M},\{c_{i}\}_{i=1}^{M})}$, where $d_i$ - block duration of $i$-th connected L1 blockchain, $c_i$ - default reliable number of confirmations in $i$-th connected L1 blockchain, $M$ - total number of connected L1 blockchains.

Using the data above, each $\Delta E$ of the Spectrum's epochs can be associated with the delta of blocks that have passed on in any connected blockchain: ${\{\lfloor \Delta E \cdot d_s \mathbin{/} d_i)\rfloor\}_{i=1}^{N}}$.
Thus, each pair of inbound and outbound transactions must contain the following information in the context:
\begin{enumerate}
    \item Parent L1 blockchain ID $p_{id}$ which UTXO of the corresponding inbound transaction belongs to;
    \item Spectrum epoch number $E_{settled}$ at which the corresponding inbound transaction settled at the Spectrum network.
\end{enumerate}

When forming outbound transaction, developers can specify a reliability factor $C$.
This factor will be compared with the ratio of the number of blocks passed on the parent L1 blockchain of the inbound transaction to the default reliable number of confirmations $c_i$ of this network:
\begin{equation}
    \theta(i-p_{id})\cdot \left\{\frac{1}{c_i} \cdot \left\lfloor (E-E_{settled}) \cdot \frac{d_s}{d_i}\right\rfloor\right\}_{i=1}^{M} >= C,\label{eq:equation2}
\end{equation}
where $E$ - current Spectrum epoch number; $\theta(x)$ - indicator function which is 1 at $x = 0$, otherwise 0.


Thus, only those UTXO will be included in outbound transactions for which the number of confirmations $N_c>=C\cdot c_i$.

\subsubsection{Formal STF definition}
The Spectrum's State Transition Function function can be defined as follows:

\begin{enumerate}
    \item For each input in $T_I$ referenced UTXO must exist in associated L1 and signature must match it's owner.
    \item For each input in $T_O$ referenced UTXO must exist in $S$ and aggregated signature of Spectrum's validators must be provided.
    \item Outbound transactions $T_O$ must include only those UTXO from the $S$ for which the number of confirmations in the parent network are reliable, i.e. $N_c>=C\cdot c_i$.
    \item Sum of all assets that are transferred into the Spectrum vault in $T_I$ must be not less than the difference between all stored assets in $S_O$ and $S$.
    \item Sum of all assets that are settled on all connected L1s in $T_O$ must not exceed the difference between all stored assets in $S_O$ and $S'$.
    \item If the number of connected blockchains has increased by $m$, then the corresponding constants $\{d_{i}^{'}\}_{i=1}^{m},\ \{c_{i}^{'}\}_{i=1}^{m}$ must be added to constants sets.
    \item If the period of time $\Delta T$ has passed, new constants sets $\{d_{i}^{'}\}_{i=1}^{M},\ \{c_{i}^{'}\}_{i=1}^{M}$ should be provided, and $\forall i \in M: {c^{'}_i >= c^{'}_i \vee d^{'}_i >= d^{'}_i}$.
    \item Returns $S'$ with all $T_I$ output UTXO added and $T_O$ input UTXO removed from vault.

\end{enumerate}


\subsection{Protocol Flow}\label{subsec:protocol-flow}

\subsubsection{Bootstrapping}\label{subsubsec:bootstrapping}

The system is bootstrapped in a trusted way.
A manually picked set of validators $V_0$ is assigned to the first epoch $E_0$.
On-chain vaults are initialized with an aggregated public key $aPK_0$ of the initial committee.

\subsubsection{Normal Flow}\label{subsubsec:normal-flow}

\begin{enumerate}
    \item Registration.
    Before an epoch starts, all Spectrum stakeholders can register for becoming a committee member.
    To get a chance of becoming a member of $V_n$ in the next epoch $E_n$ they register in a lottery
    by publishing their public keys $PK_c$ and locking collateral.
    \item Lottery.
    Once registration is done, nodes in $V_{n-1}$ compute ${selectComettee: (C_n, R_n) \rightarrow V_n}$,
    where  $C_n$ is a candidates pool, $R_n$ is a public random number.
    \item Committee key aggregation.
    Once new committee is selected, nodes in $V_n$ aggregate their individual public keys $\{PK_i\}$ into
    a joint one $aPK_n$.
    \item Committee transition.
    Nodes in $V_{n-1}$ publish cross-chain message ${M_n : (aPK_n, \sigma_{n-1})}$ , where $aPK_n$ is
    an aggregated public key of the new committee $V_n$ , $\sigma_{n-1}$ is an aggregated signature of
    $M_n$ such that ${Verify(\sigma_{n-1}, aPK_{n-1}, Mn) = 1}$.
    Vaults are updated such that ${Vault\{(E_{n-1}, aPK_{n-1})\} \coloneqq (E_n, aPK_n)}$.
    \item Decentralized Asset Management (Custodial).
    Nodes in $V_n$ observe events on supported L1 chains, agree on the set of updates
    and compute state outbound state transitions accordingly.
    \item Notarisation (Non-custodial).
    Nodes in $V_n$ observe events on supported L1 chains, batch updates, collectively sign them and
    publish on-chain.
\end{enumerate}