This section presents Spectrum protocol design starting from a naive approach based on PBFT and gradually\
addressing the challenges.

\subsection{Strawman Design: PBFTNetwork}\label{subsec:strawman-design}

For simplicity we begin with a notarization protocol based on PBFT, then iteratively refine it into Spectrum.

PBFTNetwork assumes that a group of ${n = 3f + 1}$ trusted nodes has been pre-selected upfront and fixed and at most\
$f$ of these nodes are byzantine.
At any given time one of these nodes is the \emph{leader}, who observes events on connected blockchains,
batch them and initiate round of notarization within the consensus group.
Remaining members of the consensus group verify the proposed batches by checking the presence of updates on\
corresponding blockchains.
Upon successful verification each node signs the batch with its secret key and sends the signature to the leader.

Under simplifying assumptions that at most $f$ nodes are byzantine the PBFTNetwork guarantees livness and safety.
However, the assumption of a fixed trusted committee is not realistic for open decentralized systems.
Moreover, as PBFT consensus members authenticate each other via non-transferable symmetric-key MACs, each consensus
member has to communicate with others directly, what results in $O(n^2)$ communication complexity.
Quadratic communication complexity imposes a hard limit on scalability of the system.
Such a design also scales poorly in terms of adding support for more chains.
The workload of each validator grows linearly with each added chain.

In the subsequent sections we address these limitations in four steps:
\begin{enumerate}
    \item \textbf{Opening consensus group and leaders.} We introduce a lottery-based mechanism for selecting consensus\
    group and leaders dynamically.
    \item \textbf{Replacing MACs by Digital Signatures.} We replace MACs by digital signatures to make authentication\
    transferable and thus opening the door for sparser communication patterns that can help to reduce\
    the communication complexity.
    \item \textbf{Scalable Collective Signature Aggregation.} We utilize Byzantine-tolerant aggregation\
    protocol that allows for quick aggregation of cryptographic signatures to reduce communication complexity\
    to $O(\log n)$.
    \item \textbf{Eliminating Validator Bottleneck.} We shard consensus groups into units by the type of chain\
    each node is able to handle.
\end{enumerate}

\subsection{Opening Consensus Group}\label{subsec:opening-consensus-group-and-leaders}
Spectrum is an open-membership protocol, so PBFTNetwork's assumption on a closed consensus group is not valid.
Sybil attacks can break any protocol with security thresholds and an appropriate dynamic selection of\
the consensus group becomes crucial for preserving network's liveness and safety.
Consensus group members selection should be performed in a random and trusted way to ensure that a sufficient fraction\
(at most $f$ out of ${3 f + 1}$) of the selected members are honest and of course\
the selection procedure must be independent of any internal or external advisers.

Similar selection mechanics are required in most blockchain protocols.
Bitcoin~\cite{nakamoto2009bitcoin} and many its successors are using Proof-of-Work (PoW) consensus,\
which, in essence, is a robust mechanism that facilitates randomized selection of a leader who is\
eligible to produce a new block.
Later, PoW approach was adapted into a Proof-of-Membership mechanism ~\cite{kokoriskogias2016enhancing}\.
This mechanism allows once in a while to select a new consensus group\
which then executes the PBFT consensus protocol.

A primary consideration regarding PoW-based consensus mechanisms is\
the amount of energy required to operate such systems.
A natural alternative to PoW is a mechanism based on the concept of Proof-of-Stake (PoS)~\cite{King2012PPCoinPC}.
Rather than investing computational resources in order to participate in the leader selection process,\
participants of a PoS system instead run a process that randomly selects one of them proportionally to the stake.
Pure PoS mechanism to solve the PBFT problem was firstly used in~\cite{cryptoeprint:2017/454} to select both consensus\
group members and PBFT rounds leaders and to introduce randomness into this process,\
a verifiable Random Function (VRF) has been applied.

\subsubsection{Verifiable Random Function}

A Verifiable Random Function (VRF)~\cite{Micali1999} is a reliable way to introduce randomness into a protocol.
By definition, a function $\mathcal{F}$ can be attributed to the VRF family if the following methods are defined\
for the $\mathcal{F}$:
\begin{itemize}
    \item Gen: ${Gen(1^k) \rightarrow (PK, SK)}$, where $PK$ is the public key and $SK$ is the secret key;
    \item Prove: ${Prove(x, SK) \rightarrow \pi}$, where $x$ is an input and $\pi \vcentcolon= \Pi(x, SK)$ is\
    the proof, associated with $x$ and mixed with some random value $y$, sampled from $\{0,1\}^{l_{VRF}}$.
    \item Verify: ${Verify(x, \pi, PK) \rightarrow 0 | 1}$, where the output is $1$ if\
    and only if ${\pi \equiv \Pi(x, SK)}$.
\end{itemize}

The most secure nowadays are an Elliptic Curve Verifiable Random Functions (ECVRFs).
Basically, ECVRF is a cryptographic-based VRF that satisfies the trusted uniqueness, trusted collision resistance,\
and full pseudorandomness properties ~\cite{cryptoeprint:2014/905}.
The security of ECVRF follows from the decisional Diffie-Hellman assumption in the random oracle model, thus\
ECVRF is a good source of randomness for a blockchain protocol.
Using ECVRF is also cheap and fast, since single ECVRF evaluation is approximately 100 microseconds on\
x86-64 for a specific curves used in hash functions.
Moreover, there is a great UC-extension for batch verification proposed by ~\cite{cryptoeprint:2022/1045}\
which make it even faster.

\subsubsection{Lottery}
Our lottery mechanism is based on ECVRF as a source of randomness and is generally inspired\
by~\cite{cryptoeprint:2017/573}.
The lottery is designed to achieve two main purposes: \emph{select a consensus group dynamically},
\emph{select a slot leader}.

\textbf{Lottery Function}.
The main selection logic is implemented in the lottery function.
The lottery function ${\mathcal{F}}_{L}$ compares a random number $y$ derived from the generated VRF random\
proof $\pi$ with publicly known threshold value $T$.
It evaluates to $1$ if and only if ${y < T}$, i.e.\
${\mathcal{F}}_{L}(state, f, \pi) \rightarrow 0|1$ where $state$ is a blockchain state snapshot.

The threshold value is calculated according to the formula ${T = 2^{l_{VRF}}\cdot \phi(\alpha, f)}$ where\
${\alpha=s/\\\sum_{l=0}^{l=M} s_l}$ is a relative stake.
Consequently, the probability of winning is calculated as ${p(\alpha, f) = 1-(1-f)^{\alpha}}$\
Winning probability depends on the participants' relative stake and is adjusted by the free parameter $f$.
This is where the PoS concept comes into play: the bigger the stake, the higher the chance of winning the lottery.

\textbf{Consensus Group Lottery}.
The Spectrum protocol initially is running by the manually selected opening consensus group $\{PK_i\}_{i=1}^M$\
of the predefined size $M$.
Stakeholders interact with each other and with locally installed ideal functionalities ${\mathcal{F}}_{LB}$,\
${\mathcal{F}}_{VRF}, {\mathcal{F}}_{L}$ over a sequence of $L = E \cdot R$ slots\
${S=\{sl_1,...,sl_L\}}$ consisting of $E$ epochs with $R$ slots each.
Let's clarify what the mentioned above pre-defined primitives are needed for:
\begin{enumerate}
    \item \emph{Ideal Leaky Beacon} ${\mathcal{F}}_{LB}$: is used to sample an epoch random seed from the\
    blockchain.
    \item \emph{Ideal Verifiable Random Function} ${\mathcal{F}}_{VRF}$: is used as a source of randomness.
    \item \emph{Lottery Function} ${\mathcal{F}}_{L}$: checks if the protocol participant is a lottery winner\
    (by lottery, we mean either the \emph{consensus group lottery} or the \emph{leader lottery} giving\
    the ability to start a batch notarization round, the lottery function in both cases remains the same,\
    only the arguments matter).
\end{enumerate}
More extended formal definition of ${\mathcal{F}}_{LB}$ and ${\mathcal{F}}_{VRF}$ can be found in the original\
Ouroboros Praos paper ~\cite{cryptoeprint:2017/573}.

Each new epoch ${e_j \gt 1}$ has a new consensus group and any protocol participant\
can try to become a member if he is verified.
Participant is verified if his verification key tuple is published and stored in the blockchain for a\
reliable period of time equals to $U_f$ slots.

At the end of the epoch ${e_j \gt 1}$ every verified $PK_i$ requests a\
new epoch seed $\eta_j$ from the ${\mathcal{F}}_{LB}$.
When every $PK_i$ evaluates ${\mathcal{F}}_{VRF}$ and passes the received proof $\pi$ to the ${\mathcal_{F}}_{L}$ to\
reveal the result of the consensus group lottery.
To calculate an appropriate threshold ${T_i^j}$, ${\mathcal{F}}_{L}$ should be parametrized with the same\
stake distribution which was in the last block used by ${\mathcal{F}}_{LB}$ to sample the new $\eta_j$.
Argument of the winning probability function $p$ is ${f = M /\/ N}$, where $M$ is a number of new consensus group\
members to select and $N$ is the total number of the verified stakeholders.

For security purposes, a lower bound number of committee members $M$ is required.
If the number of verified participants is less than ${M_{\min}=256}$, then the consensus group lottery is not held.
The maximum consensus group size is ${M_{\max}=1024}$, this limit arises due to the complexity of communication.

\textbf{Leader Lottery}.
Once a new consensus group is determined, the lottery process does not stop, but this time\
the leader of the group should be determined.

During an epoch, for each slot ${sl_l \in S}$, each member of the consensus group $PK_i$ separately evaluates\
${\mathcal{F}}_{VRF}$ with his own input ${x = \eta_j || sl_l || \emph{nonce}}$.
In response, he receives the associated random proof $\pi$.
If ${\mathcal{F}}_{L}(state, f, \pi) \rightarrow 1$ then $PK_i$ is a slot leader and he can propose a batch\
which should be notarized by at least two-thirds of committee members.

The parameter $f$ that regulates the probability of winning is different from the one used in the consensus\
group lottery.
Here, it is the pre-defined value determines how many slots will have at least one selected leader,\
it is called an active slots coefficient.

The lottery mechanism described in this subsection is fast, secure, and adaptive, since the pre-defined parameters\
can be changed via the voting process.
The same primitives are used to achieve different goals, namely, select a consensus group dynamically\
and select a slot leader.

Regarding the security it is important to note, that participants use their public VRF-keys for\
VRF functionality evaluation in the consensus group lottery and secret VRF-keys in the leader lottery.
This way, slot leaders don't become publicly known in advance.
An attacker can't see who is a slot leader until he initializes batch notarization, thus an attacker can't know\
who specifically to attack in order to control a certain slot.
Opening consensus group members on the over hand should be known ahead of time for the synchronization.
There are hundreds of consensus members in every epoch, so denial of service attacks are difficult to succeed.
Grinding attacks are limited because an adversary can't arbitrarily control $\eta_j$ values,
all he can try to do is to make as many forks as possible to estimate the most advantageous, but according to the\
analysis~\cite{cryptoeprint:2017/573} this advantage doesn't change the security properties of the entire protocol.



\subsection{Replacing MACs by Digital Signatures}\label{subsec:replacing-macs-by-digital-signatures}

todo

\subsection{Scalable Collective Signature Aggregation}\label{subsec:scalable-collective-signature-aggregation}

todo

\subsection{Spectrum Ledger}\label{subsec:ledger}
Spectrum network is complex since it works with multiple connected blockchains.
In order to accurately describe and verify all the necessary properties of the Spectrum ledger, we should agree on the\
semantics and introduce data types that satisfy some validity conditions which we will describe in this section.

\subsubsection{Model formalization}
We will start with a semantics needed to describe our system:
\begin{itemize}
    \item \textbf{State}: the underlying data type of the Spectrum ledger, the validity of which must be proven.

    \item \textbf{Local state}: auxiliary data type describing the state on one of the connected chains.

    \item \textbf{Transition}: the means by which system is moved from one valid \textit{state} to another valid\
    \textit{state} or from one valid \textit{local state} to another valid \textit{local state}.

    \item \textbf{Signal}: the means by which \textit{transitions} are triggered.

    \item \textbf{Rule}: algorithm that describes how to derive a valid \textit{states},\
    \textit{local states} or \textit{transitions}.The rule has an \textit{antecedent}\
    (sometimes called \textit{premise}) and a
    \textit{consequent} (sometimes called \textit{conclusion}), such that if the
    conditions in the antecedent hold, the consequent is assumed to be a valid
    \textit{state}, \textit{local state} or \textit{transition}.

    \item \textbf{Environment}: sometimes network may implicitly rely on some additional information being present\
    in the system to evaluate the \textit{rules}.
    An environment does not have to exist (or, equivalently, we may have an empty environment), but crucially\
    an environment is not modified by \textit{transitions}.
\end{itemize}

The whole set of the above concepts forms a state transition system ${\Psi = (S, L, T, \Sigma, R, E)}$,\
where $S$ is a set of \textit{states}, $L$ is a set of \textit{local states}, $T$ is a set of \textit{transitions},\
$\Sigma$ is a set of \textit{signals}, $R$ is a set of \textit{rules} and $E$ is a set of \textit{environment} values.

\subsubsection{State Transition System}
todo

\subsection{Eliminating Validator Bottleneck}\label{subsec:eliminating-validator-bottleneck}
So far each member of the consensus group had to track changes on all connected chains in order to participate in\
consensus properly.
However, this approach reduces the number of possible consensus participants and limits the scalability of the system.
Therefore, for the optimal design of our consensus protocol, we will use the following observations:
\begin{itemize}
    \item[]\textbf{Observation 1:} Events coming from independent chains $c_k$ are not serialized.
    \item[] \textbf{Observation 2:} Outbound transactions on independent systems $c_k$ can be independently signed.
\end{itemize}

Utilizing those properties we now introduce committee sharding.
We modify protocol in a way such that at each epoch $e_n$, $K$ distinct committees consisting of nodes equipped with\
functionality unit $F_{c_k}$ relevant to a specific connected chain $c_k$ are selected via the consensus group lottery.
All primitives using in the lottery are equal for different committees, however, lotteries are independent.

We denote one such committee shard as $V_{n}^k$, which uniquely maps to $c_k$.
Then, complete mapping of committees to chains at epoch $e_n$ can be represented as a set of tuples\
committee-chain $\{(V_{n}^k, c_k)\}$.
Throughout epoch $e_n$ all events and on-chain transactions in $c_k$ are handled exclusively by $V_{n}^k$.
Nodes in $V_{n}^k$ maintain a robust local ledger $L^{local}_k$ of notarized batches of events observed in $c_k$.

\subsubsection{Syncing Shards}

Each committee $V_n^k$ forms the notarized batches of events and them into their local ledgers $L_k$.
All these batches should be periodically synced and added to a block of the one main super ledger $L^+$\
in order for the system to be able to compute cross-chain state transition.
To facilitate this process, batches of the notarized events should be broadcast to other committees.
The main actors at this stage are:
\begin{enumerate}
    \item \emph{Local leader}: local committee leader.
    \item \emph{Relayer}: any protocol participant, who broadcasts notarized batches to the local leader\
    and to other committees' members.
    Every \emph{Local leader} can be a relayer at the same time.
    \item \emph{General leader}: one of the local leaders who added a block consisted of collected\
    notarized batches and other internal transactions to the $L^+$.
\end{enumerate}

There is no any separate lottery for the general leadership and any local leader is able to publish his\
block to $L^+$, thus, he can choose from two main strategies:
\begin{enumerate}
    \item \emph{Wait}: malicious strategy where local leader waits for broadcasts from other committees\
    members and don't broadcast his own batch to eliminate competitors for adding a block.
    \item \emph{Broadcast and wait}: fair strategy where local leader immediately broadcasts his batch,\
    waits for broadcasts from other committees members and honestly adds a block.
\end{enumerate}
Thus, there should be a motivation for the individual local leader to choose the fair strategy instead of keeping\
his batch for too long and there also should be a motivation for every committee member to act as a relayer.
This is achieved through the design of the incentive system.

\subsubsection{Incentives}

There are three types of the incentive for the Spectrum protocol participants: ${\{R_b, R_d, R_m\}}$, where $R_b$ is a\
guaranteed reward for adding a notarized batch to the block, $R_d$ is given for broadcasting a batch to the\
general leader and $R_m$ is given personally to the general leader who finally added the block.
Delivery reward $R_d$ is given if and only if a delivery was made within a predetermined period of time $\Delta t_d$.

Reward amounts are initially configured in such a ratio that if ${R_d=0}$ there is no prior strategy for\
local leaders, they will either wait for other batches or broadcast their batches with equal probability.
At the same time, all other committee members are still motivated to act as a relayers to receive an extra reward,
since the notarized batch can be firstly generated by any member of the committee.
All the rewards except $R_m$ are shared equally between all committees members whose signatures are included in\
the finally added block.

As a result, the syncing shards flow looks as follows:
\begin{enumerate}
    \item After notarization, the committee member holding the notarized batch, which contains the local\
    notarization time $t^*$, sends it to his local leader and to other known committees members.
    \item All committees members who receive notarized batches from other committees also send it\
    to the local leader.
    \item The local leader collects the received notarized batches.
    \item When waiting time approaches $\Delta t_d$, local leader forms a block\
    from all collected batches and broadcast it.
    Block contains the set of the notarization times and block creation time $t^B$.
    \item After block is reliably settled in the $L^+$, all associated participants can claim their rewards.
\end{enumerate}

We also introduce another type of authority incentive that increases the chances of participants\
to be selected in the consensus group lottery.
When calculating the steak distribution, which is needed to parametrise the lottery function,\
all stakes are weighted depending on the actions of their holders in the previous epoch,\
i.e. ${s_i = A_m \cdot s_i^{real}}$, where $A_m$ is an authority multiplier.
If some authority was a member of a committee and participated in adding of 2/3 of the blocks produced in\
the considered period of time, then his actual stake ${s_i^{real}}$ is multiplied by ${A_m = 2}$.
Multiplier $A_m$ decreases linearly to 0, which is the case where member was passive during the entire epoch.

With this mechanism, we solve the following problems:
\begin{itemize}
    \item Members are motivated to be focused on cooperation with other committees\
    so that their participation is reflected in each block added in the $L^+$.
    \item Inactive and dishonest members are automatically excluded from the next epoch committee.
    \item Participants are motivated to stay active throughout the entire epoch so that their chances of being\
    selected in the committee don't decrease due to an authority multiplier ${A_m < 1}$, otherwise,\
    in order to even the odds with new lottery participants, they will either have to increase\
    their real stake, or skip the lottery until the next one.
\end{itemize}



\subsection{Protocol Flow}\label{subsec:protocol-flow}
let's summarize all of the above and describe the full flow of the Spectrum protocol.
Protocol is running by manually selected opening consensus group $V_0$.
Stakeholders interact with each other and with the ideal functionalities ${\mathcal_{F}}_{RLB}$,\
${\mathcal_{F}}_{VRF}$, ${\mathcal_{F}}_{KES}$, ${\mathcal_{F}}_{DSIG}$ over a sequence of $L = E \cdot R$ slots\
${S=\{sl_1,...,sl_L\}}$ consisting of $E$ epochs with $R$ slots each.

\subsubsection{Bootstrapping}\label{subsubsec:bootstrapping}

The system is bootstrapped in a trusted way.
A manually picked set of validators $V_0$ of the predefined size $M$ is assigned to the first epoch $e_0$.
\begin{enumerate}
    \item On-chain vaults are initialized with an aggregated public key $aPK_0$ of the initial committee.

    \item All consensus group members i.e. $\forall PK_i \in V_0$ should generate the tuple of verification keys\
    ${(v_i^{vrf}, v_i^{kes}, v_i^{dsig})}$, using the ideal functionalities ${\mathcal_{F}}_{VRF}$,\
    ${\mathcal_{F}}_{KES}$, ${\mathcal_{F}}_{DSIG}$ instances, running on their machines.

    \item Then, to claim an initial stakes $\{s_i\}_{i=0}^M$ every protocol participant sends a request\
    ${(\textbf{ver\_keys}, sid, PK_i, v_i^{vrf}, v_i^{kes}, v_i^{dsig})}$ to the ${\mathcal_{F}}_{RLB}$,\
    which saves the key tuple ${(PK_i, v_i^{vrf}, v_i^{kes}, v_i^{dsig})}$.

    \item Full set of the verification\
    keys tuples ${V_{init} = \{(PK_i, v_i^{vrf}, v_i^{kes}, v_i^{dsig})\}_{i=0}^M}$\
    should be stored in the blockchain and acknowledged by all members of the initial consensus group.

    \item ${\mathcal_{F}}_{RLB}$, parameterized with confirmed $V_{init}$ is evaluated independently by every\
    participant to sample an initial random seed value $\eta \leftarrow \{0, 1\}^\lambda$.

    \item Finally, all approved stakeholders should agree on the genesis block\
    ${B_0=\left(\{(PK_i, v_i^{vrf}, v_i^{kes}, v_i^{dsig}, s_i)\}_{i=0}^M, \eta\right)}$.
\end{enumerate}

\subsubsection{Normal Flow}\label{subsubsec:normal-flow}
Once the system is bootstrapped, protocol is running in a normal flow.
\begin{enumerate}[listparindent=0.5cm,
    align=left]
    \item \textbf{Registration}.
    Any Spectrum stakeholder can register to become a committee member.
    To get a chance to be included in the set of validators $V_n$ in the epoch $e_n$ participant $PK_i$ should register\
    in the lottery during the epoch $e_{n-2}$ by publishing his verification tuples\
    ${(v_i^{vrf}, v_i^{kes}, v_i^{dsig})}$.

    \item \textbf{Consensus Lottery}.
    At the end of the epoch ${e_j \geqslant 1}$ every verified $PK_i$ sends\
    ${(\textbf{epochrnd\_req}, sid, PK_i, e_j)}$ to ${\mathcal_{F}}_{RLB}$ and receives\
    $({\textbf{epochrnd}, sid, \eta_j)}$.
    When every $PK_i$ evaluates ${\mathcal_{F}}_{VRF}$ with a query\
    ${(\textbf{EvalProve}, sid, \eta_j || e_j || \textbf{test*})}$\
    and pass the received random $y$ value to the ${\mathcal_{F}}_{L}$.
    If successful, then publish $y$ and the associated proof to form an approved consensus\
    group members table for the epoch ${e_{j + 1}}$.

    \item \textbf{Committee key aggregation}.
    Once new committee is selected, nodes in $V_n$ aggregate their individual public keys $\{PK_i\}$ into
    a joint one $aPK_n$.

    \item \textbf{Committee transition}.
    Nodes in $V_{n-1}$ publish cross-chain message ${M_n : (aPK_n, \sigma_{n-1})}$ , where $aPK_n$ is
    an aggregated public key of the new committee $V_n$ , $\sigma_{n-1}$ is an aggregated signature of
    $M_n$ such that ${Verify(\sigma_{n-1}, aPK_{n-1}, Mn) = 1}$.
    Vaults are updated such that ${Vault\{(E_{n-1}, aPK_{n-1})\} \coloneqq (e_n, aPK_n)}$.

    \item \textbf{Chain extension}.
    Every online consensus group member collects en existed chains and verifying that for every chain every block,\
    produced up to $Z$ blocks before contains correct data about slot $sl'$ leader $PK'$.
    To verify a valid slot leader, response from the ${\mathcal_{F}}_{VRF}$ to query\
    ${(\textbf{Verify}, sid, \eta' || sl' || \textbf{test}, y', \pi', v^{vrf'})}$ should be\
    ${(\textbf{Verified}, sid, \eta' || sl' || \textbf{test}, y', \pi', 1)}$ and $y'<T_j'$ as well.
    String \textbf{test} is an arbitrary and value $T_j'$ is a threshold for the stakeholder $PK'$ for the slot $sl'$.

    Then every lottery participant separately evaluates ${\mathcal_{F}}_{VRF}$ with his own inputs\
    ${(\textbf{EvalProve}, sid, \eta_j || sl || \textbf{nonce})}$ and\
    ${(\textbf{EvalProve}, sid, \eta_j || sl || \textbf{test})}$, where \textbf{nonce} is an arbitrary string.
    Received outputs ${(\textbf{Evaluated}, sid, y, \pi)}$ and ${(\textbf{Evaluated}, sid, \rho_y, \rho_\pi)}$\
    respectively includes generated random numbers ${y, \rho_y}$ and the associated proofs ${\pi, \rho_\pi}$.
    If ${y < T_i^j}$ then $PK_i$ is a slot leader and he can propose a batch which should be notarized by all\
    committee members.
    Random number $\rho_y$ will be used to sample a random seed for the next epoch.

    \textbf{Note:} The ${\mathcal_{F}}_{VRF}$ is designed in such a way that not every slot has a leader,\
    moreover, most of the slots remain empty to serve protocol synchronization.
    If there are $P$ several elected leaders for this slot, they all propose a new batches of updates\
    $\{B_p\}_{p=0}^P$ with included proofs of the leadership ${(v_i^{vrf}, y, \pi)}$ and ${(\pi, \rho_\pi)}$.

    After notarization batch is shared with over committees via syncing shards mechanism.
    After syncing any of the separate committee leaders can propose a block consisted of notarized batches and sign\
    it with ${\mathcal_{F}}_{KES}$ by sending the request ${(\textbf{sign\_req}, sid, PK_i, B_p, sl)}$.
    Received from ${\mathcal_{F}}_{KES}$ response signature $\sigma_p$ is included in the proposed block.

\end{enumerate}