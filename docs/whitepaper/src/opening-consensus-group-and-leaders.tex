As long as Spectrum is an open-membership protocol, PBFTNetwork's assumption on a closed consensus group is nod valid.
Sybil attacks cat break any protocol with security thresholds such as PBFT's assumption that at most $f$ out of ${3 f + 1}$ members are honest, thus, an appropriate dynamic selection of opening consensus group is crucial for network livness and safety.
Consensus group members selection should be performed in a random and trusted way to ensure that a sufficient fraction of the selected members are honest, procedure itself should be independent of any internal or external advisers.
Bitcoin~\cite{nakamoto2009bitcoin} and many of its successors are using proof-of-work (PoW) to achieve this goal.
The consensus group selection problem is relevant to an equally important issue of the leader election.
In essence, PoW in Bitcoin is a robust mechanism that facilitates randomized selection of a \enquote{leader} that is eligible to produce a new block.
A primary consideration regarding PoW is the amount of energy required for the systems that operate on it.
A natural alternative to PoW is a mechanism that relies on the notion of proof-of-stake (PoS).
Rather than investing computational resources in order to participate in the leader election process, participants of a PoS system instead run a process that randomly selects one of them proportionally to the stake each possesses according to the current state of blockchain.

\subsubsection{VRF-based random oracle}

The key aspect of system security is a random seed generation mechanism.
Randomness generation is a well studied issue in blockchain protocols design, inspired by~\cite{cryptoeprint:2017/573} and their UC-VRFs, we will use the similar logic to achieve our goals.
The most notable feature in UC-VRF implementation is that it's output results stays unpredictable even with malicious key generation.
Therefore, even though generated values are not uniformly distributed, due to their true unpredictability, they guaranty fairness of the election procedures they parametrized with.
The UC-VRF (hereinafter simply VRF) functionality operates as follows.
Given $PK_i$ user's query a VRF scheme outputs a newly generated verification key $v_i$ that is used to label a table $T$.
When $PK_i$ wants to generate a random number using VRF, he uses his private key $SK_i$ and stake $s_i$ as an input and gets a pseudo-random number from ${\{0, 1\}^{l_{VRF}}}$ and a proof $\pi_i$ in returns from the VRF\@.
Anyone with user's public key $PK_i$ and the proof $\pi_i$ can verify that the particular random number was produced with the given input, but this number can't produce the number before that time.
Verification is always consistent and will validate outputs that have already being inserted in a table $T$.
Unpredictability against malicious key generation is captured by imposing the same random selection of outputs even for the function tables that correspond to keys of corrupted stakeholders.
an ideal VRF function.
All this is achieved by introducing some ideal functionals ${\mathcal_{F}}_{VRF}$, ${\mathcal_{F}}_{KES}$, ${\mathcal_{F}}_{DSIG}$ which is described in detail in ~\cite{cryptoeprint:2017/573} and improved by ~\cite{cryptoeprint:2022/1045} in the form of ECVRFs.
Functionality, details and aspects of technical implementation are described in detail in the articles referenced above in this section.

\subsubsection{Leader election and consensus group lottery}
VRFs serves to fairly perform consensus group members and leader election procedures.
At first, let's take a look at the leader election process.
The flow of this part of the protocol is as follows:
\begin{enumerate}
    \item \textbf{Initialization}.
    Blockchain starts with manually selected opening consensus group $\{PK_i\}_{i=0}^M$ of the predefined size $M$.
    \begin{enumerate}
        \item All consensus group members i.e. $\forall PK_i, i \in M$ should generate the tuple of verification keys ${(v_i^{vrf}, v_i^{kes}, v_i^{dsig})}$, using the ideal functionalities ${\mathcal_{F}}_{VRF}$, ${\mathcal_{F}}_{KES}$, ${\mathcal_{F}}_{DSIG}$ instances, running on their machines.
        \item Then, to claim an initial stakes $\{s_i\}_{i=0}^M$ every protocol participant sends a request ${(\textbf{ver\_keys}, sid, PK_i, v_i^{vrf}, v_i^{kes}, v_i^{dsig})}$ to ${\mathcal_{F}}_{VRF}^{Init}$, which saves the key tuple ${(PK_i, v_i^{vrf}, v_i^{kes}, v_i^{dsig})}$.
        \item Set of the verification keys tuples ${\{(PK_i, v_i^{vrf}, v_i^{kes}, v_i^{dsig})\}_{i=0}^M}$ \textbf{should be stored in the blockchain and acknowledged by all members of the consensus group}.
        \item Finally ${\mathcal_{F}}_{VRF}^{Init}$ is evaluated to sample a random value $\eta \leftarrow \{0, 1\}^\lambda$ and constructs a genesis block ${B_0=\left(\{(PK_i, v_i^{vrf}, v_i^{kes}, v_i^{dsig}, s_i)\}_{i=0}^M, \eta\right)}$ \textbf{which should also be agreed upon all stakeholders}.
    \end{enumerate}
    \item \textbf{Epoch}.
    Each epoch supposed to consistent of $K \in \mathbb{N}$ slots.
    For every slot ${sl_k\in e_j}$:
    \begin{enumerate}
        \item Every online consensus group member collects en existed chains and verifying that for every chain every block, produced up to $Z$ blocks before contains correct data about slot $sl'$ leader $PK'$.
        To verify a valid slot leader, response from the ${\mathcal_{F}}_{VRF}$ to query ${(\textbf{Verify}, sid, \eta' || sl', y', \pi', v^{vrf'})}$ should be ${(\textbf{Verified}, sid, \eta' || sl', y', \pi', 1)}$ and $y'<T'$ as well.
        \item Then every lottery participant separately evaluates ${\mathcal_{F}}_{VRF}$ with his own inputs ${(\textbf{EvalProve}, sid, \eta || sl_k)}$.
        Received output ${(\textbf{Evaluated}, sid, y_i, \pi_i)}$ includes generated random number $y_i$ and the associated proof $\pi_i$.
        \item If ${y_i < T_i}$, where $T_i$ is a leader election threshold for this stakeholder, then $PK_i$ is a slot leader.
        The ${\mathcal_{F}}_{VRF}$ is designed in such a way that not every slot has a leader, moreover, most of the slots remain empty to serve protocol synchronization.
        \item If there are $P$ several elected leaders for this slot, they all propose a new blocks $\{B_p\}_{p=0}^P$ with included proofs of their leadership ${(v^{vrf^*} y^*, \pi^*)}$.
        All forks will be further resolved according to the longest chain rule.
        \item At the end of an epoch, each participant looks at the first $2/3\cdotK$ blocks of the epoch and concatenate all the $y'_k$ values together along with the $\eta$, hashing the result produces the $\eta^*$ for the upcoming epoch.
        Each participant also updates their $T_j$ values to reflect the stake distribution at the start of the just completed epoch.
    \end{enumerate}
\end{enumerate}

Opening consensus group lottery is similar and bases on the main principles described above.
In order to become a contender to participate in the consensus group each protocol user $PK_i$ should:
\begin{enumerate}
    \item Generate verification key  $v_i^{vrfc}$, using the ideal functionality ${\mathcal_{F}}_{VRFC}$.
    ${\mathcal_{F}}_{VRFC}$ differs from ${\mathcal_{F}}_{VRF}$ in the probability of being selected.
    The consensus group is supposed to contain ~1000 members.
    \item Evaluate ${\mathcal_{F}}_{VRFC}$ with ${(\textbf{EvalProve}, sid, \eta || e_j)}$ and include the output ${(\textbf{Evaluated}, sid, y^c_i, \pi^c_i)}$
    \item Compare $y_i$ with a consensus threshold $T_j^c$ and if succeeded, include the accosted values into the table $T^c(v_i^{cvrf}, ep_j) = (y^c_i, {\pi^c_i})$, stored in the blockchain.
    Initially $T^c()$ is empty and serves to control an accordance of the $v_i^{vrf}$ and a generated random values.
    \item Generate verification keys tuple ${(v_i^{vrf}, v_i^{kes}, v_i^{dsig})}$, using the ideal functionalities ${\mathcal_{F}}_{VRF}$, ${\mathcal_{F}}_{KES}$, ${\mathcal_{F}}_{DSIG}$.
    Include the accosted values into the table $T(v_i^{vrf}, ep_j) = (y_i, {\pi_i})$ also stored in the blockchain.
    \item To prevent manipulations, lottery winners will become an active consensus members only when $j + 2$ epoch begins.
\end{enumerate}

Using the scheme above we force opening consensus group members and slot leaders aren't publicly known ahead of time.
An attacker can't see who was a slot leader until after they have published a block, thus an attacker can't know who specifically to attack in order to control a certain slot ahead of time.
Grinding attacks are also limited because an adversary can't arbitrarily control $\eta^*$ values.
All he can try to do is to make as many forks as possible to estimate the most advantageous, but according to the analysis this advantage doesn't change the security properties of the entire protocol.
Thus, using this type of VRF is cheap, fast and secure enough for consensus group and leader election procedures.

\subsubsection{VRF design}
An appropriate VRF can be designed using two hash functions
TODO

