\documentclass{article}
\usepackage[utf8]{inputenc}
\usepackage{mathtools}
\usepackage[
    backend=biber,
    style=numeric,
    sorting=none,
    natbib
]{biblatex}
\addbibresource{references.bib}

\usepackage[dvipsnames]{xcolor}
\usepackage{color}
\usepackage{csquotes}
\usepackage{textcomp}
\usepackage{amsfonts}
\usepackage{enumitem}
\usepackage{hyperref}
\usepackage[T1]{fontenc}
\usepackage{inconsolata}
\usepackage{amsmath}
\usepackage{amssymb}
\usepackage{algorithm}
\usepackage{algpseudocode}
\usepackage{algorithmicx}

\usepackage{xcolor}
\usepackage{pifont}
\usepackage{listings}
\usepackage{lipsum}
\usepackage{siunitx}
\lstset{
    basicstyle=\color{gray}\ttfamily
}
\hypersetup{
    colorlinks,
    linkcolor={red!30!black},
    citecolor={blue!50!black},
    urlcolor={blue!80!black}
}

\newenvironment{tttabular}[1]
{\ttfamily \begin{tabular}{#1}}
{\end{tabular}}

\newenvironment{protocol}
{
    \begin{center}
        \hrule height.8pt depth0pt \kern2pt
        \renewcommand{\thealgorithm}{}
        \renewcommand{\caption}[2][\relax]{
                {\raggedright\textbf{Protocol} ##2\par}
            \ifx\relax##1\relax
            \addcontentsline{loa}{algorithm}{\protect\numberline{\thealgorithm}##2}
            \else
            \addcontentsline{loa}{algorithm}{\protect\numberline{\thealgorithm}##1}
            \fi
            \kern2pt\hrule\kern2pt
        }
        }{
        \kern2pt\hrule\relax
    \end{center}
}

\newenvironment{functionality}
{
    \begin{center}
        \hrule height.8pt depth0pt \kern2pt
        \renewcommand{\thealgorithm}{}
        \renewcommand{\caption}[2][\relax]{
                {\raggedright\textbf{Functionality} ##2\par}
            \ifx\relax##1\relax
            \addcontentsline{loa}{algorithm}{\protect\numberline{\thealgorithm}##2}
            \else
            \addcontentsline{loa}{algorithm}{\protect\numberline{\thealgorithm}##1}
            \fi
            \kern2pt\hrule\kern2pt
        }
        }{
        \kern2pt\hrule\relax
    \end{center}
}

\newenvironment{algo}
{
    \begin{center}
        \hrule height.8pt depth0pt \kern2pt
        \renewcommand{\thealgorithm}{}
        \renewcommand{\caption}[2][\relax]{
                {\raggedright\textbf{Algorithm} ##2\par}
            \ifx\relax##1\relax
            \addcontentsline{loa}{algorithm}{\protect\numberline{\thealgorithm}##2}
            \else
            \addcontentsline{loa}{algorithm}{\protect\numberline{\thealgorithm}##1}
            \fi
            \kern2pt\hrule\kern2pt
        }
        }{
        \kern2pt\hrule\relax
    \end{center}
}

\newcommand{\lt}{<}
\newcommand{\gt}{>}
\newcommand{\blue}{\color{blue}}

\newlist{legal}{enumerate}{10}
\setlist[legal]{label*=\arabic*.}

\title{Spectrum: Cross-chain interoperability at scale}
\author{Spectrum Labs}
\date{March 2023}

\begin{document}
    \begin{sloppypar}
        \maketitle


        \section{Introduction}\label{sec:introduction}
        Following the success of Bitcoin, many blockchain-based cryptocurrencies have been developed and deployed.
To meet different requirements in various scenarios, a great number of heterogeneous blockchains have emerged.
However, most of the presented blockchain platforms are developed independently, therefore, they are\
designed for their own use cases and incompatible with each other.
Hence, interoperability between blockchains has become one of the key issues\
which prevents blockchain technology from wide adoption.

With fair blockchain interoperability users can potentially conduct transactions across different blockchain networks\
smoothly and without any intermediaries.
This guarantees a reduction in the fragmentation of the crypto ecosystem and opens up new horizons and business models.
Implementation of the blockchain interoperability protocol is challenging since different blockchains have\
different security solutions, consensus algorithms and programming languages.
An inaccurate solution can potentially increase the possibility of attacks and create management challenges\
across different connected networks.

The classic cross-chain interoperability solution is a trusted oracle that registers some event on one blockchain\
and performs the required action on the other.
Centralized oracles provide fast and cheap transactions but lack a key feature -- decentralization.
The liquidity of the protocol built on this approach is custodial which is a centralized approach similar to CeFi when\
users deposit their funds to an exchange's wallet.

Another common approach involves intermediate network consisting of a fixed number of hand-picked oracles to facilitate\
data transfer among multiple blockchains.
The consensus mechanism in such protocols is usually proof-of-authority or proof-of-stake, hence, the wide range\
of potential validators are eliminated due to verification procedures or high collateral and network moderation\
typically carried out by several dozen of rarely alternating nodes.
Moreover, a common practice is to store funds transferred between blockchains on some kind of threshold wallets,\
which are generated by the participants of the intermediate network.
This results in all funds being controlled by a fixed group of oracle operators.
Such a system is also not truly decentralized.

Regarding the application scenarios, one of the most popular in the existing\
blockchain interoperability proposals is an atomic token swap.
However, atomic token swapping protocols~\cite{Miraz2019} are not self-inclusive enough to complete the tasks of\
cross-chain decentralized applications with complex activities than just a token exchanges.
The reason is that the atomic swapping process does not have the ability to destroy a certain amount\
of assets in the source blockchain and re-create the same amount on the target blockchain.
Moreover, this process always requires a counterparty who is willing to exchange tokens~\cite{Schulte2019TowardsBI}.

True blockchain interoperability requires the users and developers have the ability to access information\
from one blockchain inside another without any additional efforts from a third party.
This is a great-efforts task, thus, before achieving a successfully interoperable multi-blockchain system,\
many challenges must be overcome, such as scalability when applying to a large-scale scenario~\cite{Kim2018} and etc.

The motivation of this paper is to describe the Spectrum protocol, which provides an open, truly decentralized,\
secure and scalable cross-chain interoperability solution.
The Spectrum protocol is intended for both end-users and developers, who will be able\
to implement their applications on top of the protocol to widespread\
the use of blockchain technology in various business areas.



        \section{Related Work}\label{sec:related-work}
        Blockchain interoperability is promising but still faces various design challenges.
There have been many systematic researches regarding this issue and many famous authors have discussed\
chain interoperability in general.
Blockchain interoperability in the literature is usually classified into categories.
Buterin~\cite{buterin2016} suggested centralized, sidechains/relays, and hash locking.
Belchior et al.~\cite{belchior2021survey} classified it into cryptocurrency-directed approaches, blockchain engines,\
and blockchain connectors, Wang~\cite{cryptoeprint:2021/537} proposed to group it into\
chain-based interoperability, bridge-based interoperability, and dApp-based interoperability.

\subsection{Existing Interoperability Solutions}\label{subsec:interoperability-categories}
In this paper, we want to emphasize the benefits of the decentralization in the chain interoperability mechanism, so\
we will not include the systematical-level study of all existing approaches and will briefly discuss\
the classification proposed by Wang.

\subsubsection{Chain-based Interoperability}
Chain-based interoperability is aimed at public blockchains and uses atomic swaps as its main mechanism\
to exchange information between different chains.
Following the classification, there are three main approaches to implementing chain-based interoperability:\
hash locking, trusted notary scheme and sidechain.

\textbf{Hash Locking} is an intermediary method that allows to validate or execute blockchain transactions.
Hashed Time Lock Contracts (HTLCs) were originally developed as an alternative to centralized switching
and can be thought of as a distributed commitment~\cite{Kumar2021} able to fend off Byzantine adversary.
It uses a hash time-locked system to lock the transaction~\cite{Pillai2019} which is similar to the concept\
of the cross-chain atomic swap.

From the technical point of view, the hash locking approach has some significant drawbacks,\
for example, it must lock some assets during its opening phase for an established transaction channel,\
thereby creating a race condition and, moreover, the possibility of losing assets if a timeout occurs.

\textbf{Trusted Notary Scheme} is usually considered as the simplest way to achieve cross-chain interoperability.
The blockchain notary schemes can provide the functionalities of timed proof of existence,\
whose proof can be used as further proof of ownership~\cite{DIFRANCESCOMAESA202099}.
It doesn't require any additional changes in the underlying blockchains and uses a trusted notary to verify the\
correctness and integrity of information transferred.
A notary can be a stand-alone authority or a group of trusted parties that monitor order books of the connected chains\
and initiating transactions upon the occurrence of some valid events or requests.

Well-known solutions using this technology are, for example, Herdius~\cite{Balazs2017} and Bifrost~\cite{Scheid2019}.
In practice, the most appropriate way to achieve interoperability using a notary scheme is to combine it with\
other methods, as it is done in the Interledger~\cite{Thomas2015} which combines it with a sidechain.

\textbf{Sidechain} is the most promising approach in this category.
Sidechain can add new functionalities, namely, security and privacy to the existing blockchains,
making possible a tokens synchronization and additional data transfer between chains~\cite{Parizi2019}.
The essential feature of the sidechain is that it's design always takes into consideration the structure\
and the consensus of each connected blockchain, but none of the mainchains are aware of the presence of a sidechain.
This is achieved by utilizing a two-way peg scheme~\cite{SINGH2020102471} which uses a\
relay routine for a bidirectional hooking.
An important consequence of this approach is that sidechains can be designed\
in a decentralized manner and have their own consensus protocols.

Using a two-way pegs introduces a level of centralization, however,\
there are solutions which uses a federated two-way pegs where single authority is replaced by\
a group of trusted individuals selected in a trustworthy manner.

State-of-the-art sidechain platforms are Loom~\cite{Loom2019}, Liquid~\cite{Nick2020LiquidAB}\
and Proof-of-Authority (PoA) networks~\cite{POA2018}.
There also exists a lot of ongoing projects since this technology is innovative and in demand by the blockchain industry.

Summing it up, a practical way to apply chain-based interoperability methods to current mainstream blockchain\
systems is to combine them together.
Most existing solutions are designed primarily to exchange assets, however blockchain technology is much wider in its applications, and it's better to focus on transaction\
interoperation between different chains in practical implementations and effectively use all these promising approaches.

\subsubsection{Bridge-based Interoperability}
Bridge-based interoperability aims to create a connection component between homogeneous\
and heterogeneous blockchains.
Solutions in this field are more complex and typically support the extension of smart contracts which allows\
developers to design and deploy their own logic thereby expanding the interoperability applications.
Bridge-based interoperability can be implemented in two main forms: trusted relay and blockchain engine.

\textbf{Trusted Relay} is a very native approach where trusted parties share transactions between different blockchains.
Relay schemes replicate block information of the source blockchain via verifiable smart contracts\
within a target blockchain to allow the target blockchain to verify\
the existence of data on the source blockchain without requiring trust in a centralized entity~\cite{buterin2016}.
There are many developing relay schemes: BTC Relay~\cite{Chow2016}, PeaceRelay~\cite{Luu2019}, etc.
State-of-the art projects are: Hyperledger Cactus~\cite{Hyperledger2020}, Testimonium~\cite{Frauenthaler2020} and\
Tesseract ~\cite{cryptoeprint:2017/1153}.
All these solutions support complex use case and are highly usable and reliable, however,\
still not fully decentralized~\cite{cryptoeprint:2021/537}.

\textbf{Blockchain Engine} also provides a relay among the connected blockchains.
It is based on a shared infrastructure which support different layers and services including network, consensus,\
incentive, etc.
Requirements of multi-layer supports is essential, thus, most existing blockchain engine-based solutions are still in\
the stage of proof of concept or under active development.
Most significant projects are: Polkadot~\cite{cryptoeprint:2020/641}, Cosmos~\cite{Kwon2019},
WanChain~\cite{Wanchain}, and ARK~\cite{ARK}.

All bridge-based solutions provide convenience for end-users since they\
don't need to know what happens in the bridge.
In general, trusted relays are much more simple and adopted to handle chain interoperability, however,\
they usually utilize mechanisms similar to the notary schemes which also leads to a certain degree of centralization.

\subsubsection{dApp-based Interoperability}
Presence of well functioning decentralized applications (dApps) is significant in the blockchain ecosystem, so\
dApps should be interoperable as well and this is the goal of dApp-based interoperability.
Each dApp cannot ensure semantic interoperability, and it's essential to develop the minimum semantic that\
must be supported by each application to achieve interoperability among dApps.
dApp-based blockchain interoperability protocols in the literature are typically classified as:\
blockchain of blockchains, blockchain adapters and blockchain agnostic protocols.

\textbf{Blockchain of Blockchains} is a platform that allows developers to construct cross-chain dApps where\
each blockchain functions as an independent one.
It is similar to the sidechain idea but differs in implementation.
Sidechains are typically aimed at atomic swaps among the homogeneous blockchains where all actions should be coordinated\
by the mainchain.
Blockchain of blockchains solutions typically requires a second layer of blockchain (mainchain)\
to record the activities that happen on each subchain which can be heterogeneous~\cite{cryptoeprint:2021/537}.
There are several projects where blockchain of blockchains concept is applied for different scenarios:\
Overledger~\cite{Verdian2018}, HyperService~\cite{Liu2019}, SMChain~\cite{cryptoeprint:2019/1401} and etc.

\textbf{Blockchain Adapter} handles the interoperability by providing an interface for the end-users to\
runtime selection, smart contracts, etc.
Most significant project in this category are PleBeuS~\cite{Scheid2020} and smart contracts \emph{move} protocol\
~\cite{Fynn2020}.

\textbf{Blockchain Agnostic Protocol}: refers to a single platform allowing multiple blockchains to co-exist,\
enabling cross-chain or cross-blockchain communication between arbitrarily distributed ledgers.
Blockchain agnosticism provides its end-users various options to pick their optimal blockchain and\
provide the capabilities for cross-chain operations.
Several agnostic-based technologies have been described in the literature: ILPv4~\cite{InterledgerV4},
Gravity~\cite{PupyshevGravity2020}, SuSy~\cite{PupyshevSuSy2020} and etc.
All these solutions are flexible and has great potential, although most of them are focused on the general design\
of the prototype and do not grant backward compatibility.

Although dApp-based blockchain interoperability is very promising, most of the solutions in this category are either\
in early stages of development or lack a practical implementation\
with criteria to evaluate their effectiveness and efficiency.

\subsubsection{Discussion}

All of the interoperability approaches described above have their strengths and weaknesses.
However, the chain-based interoperability approaches, especially sidechains, are well-established and\
benefits from extensive research and improvements in design.
Sidechains have two very important pros that will help to increase the widespread\
adoption of blockchain technology in various business areas:
\begin{itemize}
    \item Having their own consensus mechanisms, sidechains can process transactions\
    efficiently and reduce transaction fees for users.
    \item Taking into consideration the structure and the consensus of each connected blockchain\
    sidechains allow dApps to expand their ecosystem.
\end{itemize}

The main cons of the existing sidechain protocols is a \emph{centralization} and \emph{poor security guaranties}\
of the consensus.
The disadvantages of centralization are obvious:
\begin{itemize}
    \item A system is not sustainable when it depends on a single party.
    \item If the trustee goes down, unfinished swaps can appear frozen halfway.
    \item A malicious trustee can censor transactions.
    \item A malicious trustee can perform a man-in-the-middle attack by sending an inaccurate data.
\end{itemize}
Almost the same deficiencies exist for a semi-centralized protocols,\
where only a few dozen individuals act as validators.
Such \enquote{decentralization} is very conditional as it is difficult to meet the requirements to become a validator,\
furthermore, malicious validators can easily cooperate to successfully attack.

Thus, we come to the conclusion that the scalable practical implementation of the truly decentralized system with\
a provably-secure consensus protocol is the main step towards wide practical usage of sidechains and bringing\
their benefits into cross-chain interoperability.



        \section{Goals}\label{sec:goals}
        To overcome the outlined problems of existing solutions, the resulted Spectrum protocol must satisfy\
the following properties:

\begin{enumerate}
    \item \textbf{Decentralization.} The system should be highly decentralized.
    \item \textbf{Interoperability.} The system should be able to support a large number of heterogeneous blockchains.
    \item \textbf{Openness.} The system should allow anyone to participate in consensus permissionlessly.
    Protocol should be fully open-source and all participants will be encouraged by the incentives system.
    \item \textbf{Consensus Scalability.} The system should be able to operate normally while maintaining\
    sufficiently large consensus groups consisting of hundreds of active validators on each connected blockchain.
    \item \textbf{Operational Scalability.} The system should scale linearly with the number of supported blockchains.
    \item \textbf{Security.} The system should be able to withstand Sybil attacks.
    \item \textbf{Sustainability.} The system should be able to tolerate faults of particular connected blockchains.
    \item \textbf{Upgradability.} The system should allow to add new blockchains into list of supported over time.
\end{enumerate}

To achieve our goals, we will combine the best practices from the approaches, that are already in use\
in the cross-chain interoperability solutions.
To eliminate the existing bottlenecks, we will supplement them with own-developed improvements,\
which we will emphasize and describe in details in the following sections.


        \section{System Model}\label{sec:system-model}
        In this section we will describe the main components and general assumptions which is essential to\
conceptualize and construct our protocol.

\subsection{Transaction Ledger}\label{subsec:transaction-ledger.}
To be an appropriate for our goals we adopt the definition of transaction ledger from~\cite{cryptoeprint:2016/889}.
A protocol $\Pi$ implements a robust transaction ledger provided that $\Pi$ is divided into blocks that determine the order\
in which transactions are incorporated into the ledger.
Blocks are assigned to time slots.
It should also satisfy the following properties:
\begin{enumerate}
    \item \emph{Persistence.} Once a node of the system proclaims a certain transaction tx as stable, the remaining\
    nodes, if queried, will either report tx in the same position in the ledger or will not report as stable any\
    transaction in conflict to tx.
    Here the notion of stability is a predicate that is parameterized by a security parameter $k$; specifically, a\
    transaction is declared stable if and only if it is in a block that is more than $k$ blocks deep in the ledger.
    \item \emph{Liveness.} If all honest nodes in the system attempt to include a certain transaction then, after\
    the passing of time corresponding to u slots (called the transaction confirmation time), all nodes, if queried\
    and responding honestly, will report the transaction as stable.
\end{enumerate}

\subsection{Semi-Synchronous Model Preliminaries}\label{subsec:the-semi-synchronous-model-preliminaries.}
We consider the security model in a semi-synchronous setting, with simple modifications to account for\
adversarially-controlled message delays and immediate adaptive corruption.

\textbf{Time and Slots.}
In our setting time is divided into discrete units called slots.
The ledger associates one ledger block with each time slot (at most).
Participants are equipped with roughly synchronized clocks.
This will permit them to carry out a distributed protocol intending to collectively assign a block to this current slot.
In general, each slot $sl_r$ is indexed by an integer $r \in \{1, 2, ..\}$, and we assume that the real\
time window that corresponds to each slot has the following two properties:
\begin{enumerate}
    \item The current slot is determined by a publicly-known and monotonically increasing function of the current time.
    \item Each participant has access to the current time.
    Any discrepancies between parties' local time are insignificant in comparison with the slot duration.
\end{enumerate}

\textbf{Synchrony.}
We consider an untrustworthy network environment that allows for adversarial-controlled message delays and immediate\
adaptive corruption.
Namely, we allow the adversary $A$ to selectively delay any messages sent by an honest party for up to $\Delta \subseteq \mathbb{K}$\
slots and corrupt parties without delay.

\textbf{Random Oracle.}
We assume that a random oracle is available to each node $n \in N$.
Random oracle is designed in such a way that it is able to produce uniformly-distributed pseudo-random numbers,\
which correctness must be verifiable for all participants of the protocol.

\textbf{Security Model.}\label{subsec:security-model.}
The system is composed of a set of nodes $N$ and each node $n \in N$:
\begin{itemize}
    \item Is associated with a unique wallet holding a stake of tokens $s_n$.
    \item Able to generate key-pairs ${(PK^*, SK^*)}$ without trusted public key infrastructure.
    \item Is able to sign messages ${sign: (SK_n, m) \rightarrow \sigma}$.
    \item Is able to verify signatures ${verify: (\sigma, PK_n, m) \rightarrow 0 | 1}$.
    \item Has locally installed ideal functionalities, that act as \emph{random oracle}\\
    and as \textit{key evolving signature scheme}.
\end{itemize}

At any time $t$ a subset ${V \subseteq N}$ of nodes can be controlled by an adversary and are considered faulty.
Byzantine nodes can divert from the protocol and collude to attack the system while the remaining honest nodes follow\
the protocol.
We assume that the total stake of all faulty nodes is less than 1/2 of the total stake $s$ of all nodes.

\subsection{External Systems}\label{subsec:external-systems.}
We also assume multiple independent distributed systems ${S_1, \dots, S_k}$ with underlying ledgers ${L_1, \dots, L_k}$\
as defined in~\cite{cryptoeprint:2019/1128}.
For each ledger there is a process $P_k$ that can influence the state evolution of the underlying ledger $L_k$ by\
committing a transaction $TX_k$ into it.
We extend the model defined in~\cite{cryptoeprint:2019/1128} by assuming that all ledgers allow for execution of\
simple predicates upon validation of transactions: ${verify: C \rightarrow 0 | 1}$, where $C$ is\
a \enquote{context} that contains description of state the transaction interacts with.
There is also a function ${desc: TX_k \rightarrow DESC^{TX_k}}$ that maps transaction $TX_k$ to\
some \enquote{description}, e.g.\ specifying the transaction value, recipient address, etc.
For each  $S_k$ there is a corresponding functionality unit $F_{S_k}$ that allows any $n$ equipped with the unit\
to interact with $S_k$.
Each node $n \in N$ is equipped with at least one such functionality unit and at most $k$ functionality units.



        \section{System Design}\label{sec:system-design}
        This section presents Spectrum protocol design starting from a naive approach based on PBFT and gradually addressing the challenges.

\subsection{Strawman Design: PBFTNetwork}\label{subsec:strawman-design}

For simplicity we begin with a notarization protocol based on PBFT, then iteratively refine it into Spectrum.

PBFTNetwork assumes that a group of ${n = 3f + 1}$ trusted nodes has been pre-selected upfront and fixed and at most $f$ of these nodes are byzantine.
At any given time one of these nodes is the \emph{leader}, who observes events on connected blockchains,
batch them and initiate round of notarization within the consensus group.
Remaining members of the consensus group verify the proposed batches by checking the presence of updates on corresponding blockchains.
Upon successful verification each node signs the batch with its secret key and sends the signature to the leader.

Under simplifying assumptions that at most $f$ nodes are byzantine the PBFTNetwork guarantees livness and safety.
However, the assumption of a fixed trusted committee is not realistic for open decentralized systems.
Moreover, as PBFT consensus members authenticate each other via non-transferable symmetric-key MACs, each consensus
member has to communicate with others directly, what results in $O(n^2)$ communication complexity.
Quadratic communication complexity imposes a hard limit on scalability of the system.
Such a design also scales poorly in terms of adding support for more chains.
The workload of each validator grows lineary with each added chain.

In the subsequent sections we address these limitations in four steps:
\begin{enumerate}
    \item \textbf{Opening consensus group and leaders.} We introduce a lottery-based mechanism for selecting consensus group and leaders dynamically.
    \item \textbf{Replacing MACs by Digital Signatures.} We replace MACs by digital signatures to make authentication transferable
    and thus opening the door for sparser communication patterns that can help to reduce the communication complexity.
    \item \textbf{Scalable Collective Signature Aggregation.} We utilize Byzantine-tolerant aggregation protocol that allows for
    quick aggregation of cryptographic signatures to reduce communication complexity to $O(\log n)$.
    \item \textbf{Eliminating Validator Bottleneck.} We shard consensus groups into units by the type of chain each node is able to handle.
\end{enumerate}

\subsection{Opening Consensus Group}\label{subsec:opening-consensus-group-and-leaders}
Spectrum is an open-membership protocol, so PBFTNetwork's assumption on a closed consensus group is not valid.
Sybil attacks can break any protocol with security thresholds and an appropriate dynamic selection of\
the consensus group becomes crucial for preserving network's liveness and safety.
Consensus group members selection should be performed in a random and trusted way to ensure that a sufficient fraction\
(at most $f$ out of ${3 f + 1}$) of the selected members are honest and of course\
the selection procedure must be independent of any internal or external advisers.

Similar selection mechanics are required in most blockchain protocols.
Bitcoin~\cite{nakamoto2009bitcoin} and many its successors are using Proof-of-Work (PoW) consensus,\
which, in essence, is a robust mechanism that facilitates randomized selection of a leader who is\
eligible to produce a new block.
Later, PoW approach was adapted into a Proof-of-Membership mechanism ~\cite{kokoriskogias2016enhancing}\.
This mechanism allows once in a while to select a new consensus group\
which then executes the PBFT consensus protocol.

A primary consideration regarding PoW-based consensus mechanisms is\
the amount of energy required to operate such systems.
A natural alternative to PoW is a mechanism based on the concept of Proof-of-Stake (PoS)~\cite{King2012PPCoinPC}.
Rather than investing computational resources in order to participate in the leader selection process,\
participants of a PoS system instead run a process that randomly selects one of them proportionally to the stake.
Pure PoS mechanism to solve the PBFT problem was firstly used in~\cite{cryptoeprint:2017/454} to select both consensus\
group members and PBFT rounds leaders and to introduce randomness into this process,\
a verifiable Random Function (VRF) has been applied.

\subsubsection{Verifiable Random Function}

A Verifiable Random Function (VRF)~\cite{Micali1999} is a reliable way to introduce randomness into a protocol.
By definition, a function $\mathcal{F}$ can be attributed to the VRF family if the following methods are defined\
for the $\mathcal{F}$:
\begin{itemize}
    \item Gen: ${Gen(1^k) \rightarrow (PK, SK)}$, where $PK$ is the public key and $SK$ is the secret key;
    \item Prove: ${Prove(x, SK) \rightarrow \pi}$, where $x$ is an input and $\pi \vcentcolon= \Pi(x, SK)$ is\
    the proof, associated with $x$ and mixed with some random value $y$, sampled from $\{0,1\}^{l_{VRF}}$.
    \item Verify: ${Verify(x, \pi, PK) \rightarrow 0 | 1}$, where the output is $1$ if\
    and only if ${\pi \equiv \Pi(x, SK)}$.
\end{itemize}

The most secure nowadays are an Elliptic Curve Verifiable Random Functions (ECVRFs).
Basically, ECVRF is a cryptographic-based VRF that satisfies the trusted uniqueness, trusted collision resistance,\
and full pseudorandomness properties ~\cite{cryptoeprint:2014/905}.
The security of ECVRF follows from the decisional Diffie-Hellman assumption in the random oracle model, thus\
ECVRF is a good source of randomness for a blockchain protocol.
Using ECVRF is also cheap and fast, since single ECVRF evaluation is approximately 100 microseconds on\
x86-64 for a specific curves used in hash functions.
Moreover, there is a great UC-extension for batch verification proposed by ~\cite{cryptoeprint:2022/1045}\
which make it even faster.

\subsubsection{Lottery}
Our lottery mechanism is based on ECVRF as a source of randomness and is generally inspired\
by~\cite{cryptoeprint:2017/573}.
The lottery is designed to achieve two main purposes: \emph{select a consensus group dynamically},
\emph{select a slot leader}.

\textbf{Lottery Function}.
The main selection logic is implemented in the lottery function.
The lottery function ${\mathcal{F}}_{L}$ compares a random number $y$ derived from the generated VRF random\
proof $\pi$ with publicly known threshold value $T$.
It evaluates to $1$ if and only if ${y < T}$, i.e.\
${\mathcal{F}}_{L}(state, f, \pi) \rightarrow 0|1$ where $state$ is a blockchain state snapshot.

The threshold value is calculated according to the formula ${T = 2^{l_{VRF}}\cdot \phi(\alpha, f)}$ where\
${\alpha=s/\\\sum_{l=0}^{l=M} s_l}$ is a relative stake.
Consequently, the probability of winning is calculated as ${p(\alpha, f) = 1-(1-f)^{\alpha}}$\
Winning probability depends on the participants' relative stake and is adjusted by the free parameter $f$.
This is where the PoS concept comes into play: the bigger the stake, the higher the chance of winning the lottery.

\textbf{Consensus Group Lottery}.
The Spectrum protocol initially is running by the manually selected opening consensus group $\{PK_i\}_{i=1}^M$\
of the predefined size $M$.
Stakeholders interact with each other and with locally installed ideal functionalities ${\mathcal{F}}_{LB}$,\
${\mathcal{F}}_{VRF}, {\mathcal{F}}_{L}$ over a sequence of $L = E \cdot R$ slots\
${S=\{sl_1,...,sl_L\}}$ consisting of $E$ epochs with $R$ slots each.
Let's clarify what the mentioned above pre-defined primitives are needed for:
\begin{enumerate}
    \item \emph{Ideal Leaky Beacon} ${\mathcal{F}}_{LB}$: is used to sample an epoch random seed from the\
    blockchain.
    \item \emph{Ideal Verifiable Random Function} ${\mathcal{F}}_{VRF}$: is used as a source of randomness.
    \item \emph{Lottery Function} ${\mathcal{F}}_{L}$: checks if the protocol participant is a lottery winner\
    (by lottery, we mean either the \emph{consensus group lottery} or the \emph{leader lottery} giving\
    the ability to start a batch notarization round, the lottery function in both cases remains the same,\
    only the arguments matter).
\end{enumerate}
More extended formal definition of ${\mathcal{F}}_{LB}$ and ${\mathcal{F}}_{VRF}$ can be found in the original\
Ouroboros Praos paper ~\cite{cryptoeprint:2017/573}.

Each new epoch ${e_j \gt 1}$ has a new consensus group and any protocol participant\
can try to become a member if he is verified.
Participant is verified if his verification key tuple is published and stored in the blockchain for a\
reliable period of time equals to $U_f$ slots.

At the end of the epoch ${e_j \gt 1}$ every verified $PK_i$ requests a\
new epoch seed $\eta_j$ from the ${\mathcal{F}}_{LB}$.
When every $PK_i$ evaluates ${\mathcal{F}}_{VRF}$ and passes the received proof $\pi$ to the ${\mathcal_{F}}_{L}$ to\
reveal the result of the consensus group lottery.
To calculate an appropriate threshold ${T_i^j}$, ${\mathcal{F}}_{L}$ should be parametrized with the same\
stake distribution which was in the last block used by ${\mathcal{F}}_{LB}$ to sample the new $\eta_j$.
Argument of the winning probability function $p$ is ${f = M /\/ N}$, where $M$ is a number of new consensus group\
members to select and $N$ is the total number of the verified stakeholders.

For security purposes, a lower bound number of committee members $M$ is required.
If the number of verified participants is less than ${M_{\min}=256}$, then the consensus group lottery is not held.
The maximum consensus group size is ${M_{\max}=1024}$, this limit arises due to the complexity of communication.

\textbf{Leader Lottery}.
Once a new consensus group is determined, the lottery process does not stop, but this time\
the leader of the group should be determined.

During an epoch, for each slot ${sl_l \in S}$, each member of the consensus group $PK_i$ separately evaluates\
${\mathcal{F}}_{VRF}$ with his own input ${x = \eta_j || sl_l || \emph{nonce}}$.
In response, he receives the associated random proof $\pi$.
If ${\mathcal{F}}_{L}(state, f, \pi) \rightarrow 1$ then $PK_i$ is a slot leader and he can propose a batch\
which should be notarized by at least two-thirds of committee members.

The parameter $f$ that regulates the probability of winning is different from the one used in the consensus\
group lottery.
Here, it is the pre-defined value determines how many slots will have at least one selected leader,\
it is called an active slots coefficient.

The lottery mechanism described in this subsection is fast, secure, and adaptive, since the pre-defined parameters\
can be changed via the voting process.
The same primitives are used to achieve different goals, namely, select a consensus group dynamically\
and select a slot leader.

Regarding the security it is important to note, that participants use their public VRF-keys for\
VRF functionality evaluation in the consensus group lottery and secret VRF-keys in the leader lottery.
This way, slot leaders don't become publicly known in advance.
An attacker can't see who is a slot leader until he initializes batch notarization, thus an attacker can't know\
who specifically to attack in order to control a certain slot.
Opening consensus group members on the over hand should be known ahead of time for the synchronization.
There are hundreds of consensus members in every epoch, so denial of service attacks are difficult to succeed.
Grinding attacks are limited because an adversary can't arbitrarily control $\eta_j$ values,
all he can try to do is to make as many forks as possible to estimate the most advantageous, but according to the\
analysis~\cite{cryptoeprint:2017/573} this advantage doesn't change the security properties of the entire protocol.



\subsection{Replacing MACs by Digital Signatures}\label{subsec:replacing-macs-by-digital-signatures}

todo

\subsection{Scalable Collective Signature Aggregation}\label{subsec:scalable-collective-signature-aggregation}

todo

\subsection{Eliminating Validator Bottleneck}\label{subsec:eliminating-validator-bottleneck}

So far each member of consensus group had to track changes on all connected chains in order to participate in consensus properly.

\textbf{Observation 1:} Events coming from independent systems $S_k$ are not serialized.
Thus, the process of events notarisation can be parallelized.

\textbf{Observation 2:} Outbound transactions on independent systems $S_k$ can be independently signed.

Utilizing those properties we now introduce committee sharding.
We modify protocol in a way such that at each epoch $e$ $M$ distinct committees consisting of nodes equipped with functionality unit $F_{S_k}$ relevant to a specific connected chain $S_k$ are selected in a way described in (5.2.2).
All primitives and source of randomness are equal to different committees, the only difference is in the $f$ parameter of $\phi(\alpha_i, f)$ function, which is unique for every connected blockchain in order to guaranty expected number of members in every committee.
We denote one such committee shard as $V^{e}_{S_k}$, which uniquely maps to $S_k$.
Then, complete mapping of committees to chains at epoch $e$ can be represented as a set of tuples commettee-chain $\{(V^{e}_{S_k}, S_k)\}$.
Throughout epoch $e$ all events and on-chain transactions on $S_k$ are handled exclusively by $V^{e}_{S_k}$.

Nodes in $V^{e}_{S_k}$ maintain a robust local ledger $L^{local}_k$ of notarized batches of events observed in $S_k$.

\subsubsection{Syncing Shards}

Notarized batches of events from local ledgers $\{L_i\}_{i=1}^{i=N}$ then should be synced in a super ledger $L^+$ in order for the system to be able to compute cross-chain state transition.
To facilitate this process batches of notarized events are broadcast to other committees.
The main actors at this stage are:
\begin{itemize}
    \item \textbf{Local leaders}: committees leaders, holding local notarized batches.
    \item \textbf{Relayers}: any protocol participant, who broadcasts notarized batches from \emph{Local leaders} to other committees' members.
    Every \emph{Local leader} can be a \emph{Relayer} at the same time.
    \item \textbf{General leader}: one of the \emph{Local leaders} who added a block consisted of all collected notarized batches to the $L^+$.
\end{itemize}

Since any \emph{Local leader} is able to publish his block to $L^+$ he can choose from two main strategies:
\begin{itemize}
    \item \textbf{Wait}: malicious strategy where \emph{Local leader} waits for broadcasts from other committees members and don't broadcast his own batch to eliminate competitors for adding a block.
    \item \textbf{Broadcast and wait}: fair strategy where \emph{Local leader} immediately broadcasts his batch, waits for broadcasts from committees members and honestly competes for adding a block.
\end{itemize}
Thus, there should be a motivation for individual \emph{Local leader} to choose the fair strategy instead of keeping his batch for too long.
This is achieved through the design of the incentive system.

There are three types of incentive: ${\{R_b, R_d, R_m\}}$, where $R_b$ is a guaranteed reward for adding a notarized batch to the block, $R_d$ is given for a broadcasting batch to the general leader and $R_m$ is given personally to the \emph{General leader} who mined the block.
Delivery reward $R_d$ is given to the \emph{Relayer} if and only if a delivery was made within a predetermined period of time $\Delta t_d$.
From the game-theoretic analysis, the following relationships between rewards were derived: ${R_b = 2 \cdot R_d, R_m = 3 \cdot R_d}$.
Thus, if ${R_d=0}$ there is no prior strategy for the \emph{Local leaders}, they will or wait for other batches either broadcast their batches with equal probability.
In case when ${R_d>0}$ it is distributed between the \emph{Local leader} and the \emph{Relayer}, i.e. ${R^l_d = \xi \cdot R_d}$ and  ${R^r_d = (1 - \xi) \cdot R_d}$, where ${\xi \in (0, 1)}$.
While ${\xi \rightarrow 1}$ probability that all \emph{Local leaders} will choose the \emph{Broadcast and wait} strategy approaches $1$.

As a result, the syncing Shards flow looks as follows:
\begin{itemize}
    \item Every \emph{Local leader} broadcasts (himself or through an intermediary as a \emph{Relayer}) his batch $b_i$, which contains the local notarization time $t^N_i$ and waits for batches from other \emph{Local leaders}.
    \item When waiting time approaches $\Delta t_d$, \emph{Local leader} forms a block from all collected batches ${\{b_i^j\}_{j=1}^{j=K}, K \le N}$ and add it to $L^+$.
    Block contains the set of the notarization times $\{t^{N^j}_i\}_{j=1}^{j=K}$ and block creation time $t^B_s$.
    \item After block is settled, all associated actors receive their rewards according to their roles: \emph{General leader} receives $R_m$, \emph{Local leaders}, whose batches are in the block receives $R_b$.
    In addition, if ${t^B_s - t^N_i^* < \Delta t_d}$, where $t^N_i^*$ is $t^N_i$ time, normalized to $L^+$ time, $i$-th committee \emph{Local leader} receives $R_d$ reward shared with the \emph{Relayer}.
\end{itemize}

\subsection{Resolving forks}\label{subsec:resolving-forks}
TODO

\subsection{Protocol Flow}\label{subsec:protocol-flow}

\subsubsection{Bootstrapping}\label{subsubsec:bootstrapping}

The system is bootstrapped in a trusted way.
A manually picked set of validators $V_0$ is assigned to the first epoch $e_0$.
On-chain vaults are initialized with an aggregated public key $aPK_0$ of the initial committee.
All initial committee members generate verification tuples ${(v_i^{vrf}, v_i^{kes}, v_i^{dsig})}$
and agree on the genesis block.

\subsubsection{Normal Flow}\label{subsubsec:normal-flow}

\begin{enumerate}
    \item Registration.
    All Spectrum stakeholders can register for becoming a committee member.
    To get a chance of becoming a member of $V_n$ in the epoch $e_n$ they register in a lottery during the $e_{n-2}$
    epoch by publishing their verification tuples ${(v_i^{vrf}, v_i^{kes}, v_i^{dsig})}$.
    \item Lottery.
    Once registration is done and epoch $e_{n-1}$ comes to the end, all registered participants evaluates
    ${\mathcal_{F}}_{VRF}$ locally and compare the generated random $y$ with their corresponding consensus threshold
    ${T_i^j}^*$ for this epoch.
    If successful, then publish $y$ and the associated proofs to form an approved consensus members table.
    \item Committee key aggregation.
    Once new committee is selected, nodes in $V_n$ aggregate their individual public keys $\{PK_i\}$ into
    a joint one $aPK_n$.
    \item Committee transition.
    Nodes in $V_{n-1}$ publish cross-chain message ${M_n : (aPK_n, \sigma_{n-1})}$ , where $aPK_n$ is
    an aggregated public key of the new committee $V_n$ , $\sigma_{n-1}$ is an aggregated signature of
    $M_n$ such that ${Verify(\sigma_{n-1}, aPK_{n-1}, Mn) = 1}$.
    Vaults are updated such that ${Vault\{(E_{n-1}, aPK_{n-1})\} \coloneqq (e_n, aPK_n)}$.
    \item Decentralized Asset Management (Custodial).
    Nodes in $V_n$ observe events on supported L1 chains, agree on the set of updates
    and compute state outbound state transitions accordingly.
    \item Notarisation (Non-custodial).
    Nodes in $V_n$ observe events on supported L1 chains, batch updates, collectively sign them and
    publish on-chain.
\end{enumerate}


        \section{Applications}\label{sec:applications}
        \subsection{Decentralized Cross-Chain Oracle}\label{subsec:cross-chain-oracle}

In oracle mode of operation the system is capable of providing a notarized set of events\
observed on supported blockchains.
Cross-Chain Oracle is a simple yet solving the cross-chain interoperability solution.

\subsection{Custodial Asset Management}\label{subsec:custodial-asset-management}

In custodial mode of operation the system is capable of managing user assets which are stored on corresponding\
blockchains in vaults.
Each vault stores epoch number $n$, an aggregated public key $aPK_n$ of the current validator set $V_n$ and
is guarded with a script (smart-contract) capable of performing verification of
an aggregated signature ${verify: (\sigma_n, m_n, aPK_n) \rightarrow 0 | 1}$.

\paragraph{Natively Cross-Chain Applications}

Decentralized custodial management in conjunction with a computational layer can be highly beneficial for expanding\
the capabilities of the system.
This moves us beyond simple bridges to what we call Natively Cross-Chain Applications (NCCAs).

NCCAs are applications that are deployed in cross-chain network and are capable of interacting with other blockchains\
without the need of external oracles or bridges.
Compared to single-chain dApps, NCCAs unlocks an additional functionality by taking advantage\
of multiple chains simultaneously.
They make possible to aggregate fragmented liquidity on different chains\
into one chain or a coordinated pool of assets and improve the user experience by enabling the localization\
and customization of parameters and feature sets of the same application on different chains.
These unique advantages make them the future of web3 dApps.


        \newpage
        \printbibliography

        \newpage
        \appendix
        \section*{Appendices}\label{sec:appendix}
        \addcontentsline{toc}{section}{Appendices}
        \renewcommand{\thesubsection}{\Alph{subsection}}
        \input{Appendix}

    \end{sloppypar}
\end{document}