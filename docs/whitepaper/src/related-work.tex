Blockchain interoperability is promising, but it still faces various design
challenges.
There have been many systematic researches regarding this issue and many famous authors have discussed\
chain interoperability in general.
The blockchain interoperability in the literature is usually classified into categories: Buterin ~\cite{buterin2016}
suggested \emph{centralized}, \emph{sidechains/relays}, and \emph{hashlocking}; Belchior et\
al. ~\cite{belchior2021survey} classified it in \emph{cryptocurrency-directed approaches}, \emph{blockchain engines}, and \emph{blockchain connectors},
Wang ~\cite{cryptoeprint:2021/537} proposed to group it into \emph{chain-based interoperability},
\emph{bridge-based interoperability}, and \emph{dApp-based interoperability}.

\subsection{Existing Interoperability Solutions}\label{subsec:interoperability-categories}
In our work we want to emphasize the benefits of the decentralization in the interoperability mechanism thus,\
we will not include the systematical-level study of all existing approaches and will briefly discuss\
the classification proposed by ~\cite{cryptoeprint:2021/537}.

\subsubsection{Chain-based Interoperability}
The chain based interoperability aims to public blockchains and uses an atomic swaps as its main mechanism\
to share information across different chains.
Following the classification existing in the literature there are three main types of such approaches to\
interoperability: \emph{Sidechain}, \emph{Trusted notary Scheme}\
and \emph{Hash-locking}.

\emph{Sidechain} extend the functionalities the interconnected blockchain networks,\
by enabling tokens synchronization and additional data transfer between them ~\cite{Parizi2019}.
Sidechains can be designed in a decentralized manner and have their own consensus protocols.
While the design of a sidechain always take into consideration the structure and the consensus of each connected\
blockchain, none of the mainchains is aware of the presence of a sidechain.
Sidechain is usually connects to the mainchains with a two-way peg scheme ~\cite{SINGH2020102471}, which uses a\
relay routine to the bidirectional data transferring process.
The main issue is that two-way pegs introduces a level of centralization.
State-of-the-art sidechain platforms are Loom ~\cite{Loom2019}, Liquid ~\cite{Nick2020LiquidAB},
and Poof-of-Authority (PoA) networks ~\cite{POA2018}.
There also exist a lot of ongoing projects since this technology is innovative and in demand by the blockchain industry.

\emph{Trusted notary Scheme} is the simplest way to achieve cross-chain interoperability.
The blockchain notary schemes can provide the functionalities of timed proof of existence,\
whose proof can be used as further proof of ownership ~\cite{DIFRANCESCOMAESA202099}.
It doesn't require any additional changes in the underlying blockchains and uses a trusted notary to verify the\
correctness and integrity of information transferred.
A notary is a trusted party or a group of trusted parties that monitor connected chains and initiating transactions\
upon the occurrence of some valid events or a requests.
Well-known solutions using this technology are, for example, Herdius ~\cite{Balazs2017} and Bifrost ~\cite{Scheid2019}.
In practice, the way to achieve interoperability using a notary scheme is to combine it with other methods as\
it is done in the Interledger ~\cite{Thomas2015}, which combines it with the previously discussed sidechains.

\emph{Hash-locking}: is an intermediary to validate or execute blockchain transactions.
It uses a hash time-locked system applying a time lock to lock the transaction ~\cite{Pillai2019}\
which is similar to the concept of an atomic swap.
The hash-locking solution has some significant drawbacks, for example, it must lock some assets during its opening\
phase for an established transaction channel, thereby creating a race condition and, moreover, the possibility of\
losing assets if a timeout occurs.

Summing it up, a practical way to enable chain-based interoperability with current major blockchain systems is to\
combine the above solutions together.

\subsubsection{Bridge-based Interoperability}
Most existing chain-based blockchains are homogeneous systems, in which the assets are of the similar type.
Bridge-based interoperability aims to create a connection component between homogeneous\
and heterogeneous blockchains.
This component can be implemented in two main forms: \emph{Trusted Relay} and \emph{Blockchain Engine}.

\emph{Trusted Relay}: is very native approach where trusted parties share transactions between different blockchains.
Relay schemes replicate block information of the source blockchain via verifiable smart contracts,\
within a target blockchain to allow the target blockchain to verify\
the existence of data on the source blockchain without requiring trust in a centralized entity ~\cite{buterin2016}.
There exist many developing relay schemes: BTC Relay ~\cite{Chow2016}, PeaceRelay ~\cite{Luu2019}, etc.
State-of-the art projects are: Hyperledger Cactus ~\cite{Hyperledger2020}, Testimonium~\cite{Frauenthaler2020} and\
Tesseract ~\cite{cryptoeprint:2017/1153}.
All this solutions support complex use case and are highly usable and reliable, however,\
still not fully decentralized~\cite{cryptoeprint:2021/537}.

\emph{Blockchain Engine}: is also provides a relay among the connected blockchains.
It is based on a shared infrastructure which support different layers and services, including network, consensus,\
incentive, etc.
Requirements of multi-layer supports is essential, thus, most existing blockchain engine based solutions are still in\
the stage of proof of concept or under active development.
Most significant projects are: Polkadot ~\cite{cryptoeprint:2020/641}, Cosmos ~\cite{Kwon2019},
WanChain ~\cite{Wanchain}, and ARK ~\cite{ARK}.

All bridge-based solutions provide convenience for end-users since they\
do not need to know what happened in the bridge.
In general, trusted relays are much more simple and adopted to handle interoperability.
However, they utilize mechanism similar to the notary schemes which also leads to a certain degree of centralization.

\subsubsection{dApp-based Interoperability}
For the wide adoption of the blockchain technologies, presence of well functioning\
decentralized applications' (dApps) infrastructure is needed.
Thus, dApp-based solutions should be interoperable as well.
However, each dApp cannot ensure semantic interoperability, and it is essential to develop a minimum semantic that\
must be supported by each application to achieve interoperability among dApps.
dApp-based blockchain interoperability protocols in the literature are classified to: \emph{Blockchain of Blockchains},\
\emph{Blockchain Adapters} and \emph{Blockchain Agnostic Protocols}.

\emph{Blockchain of Blockchains}: is a platform that allows developers to construct a cross-chain dApps, where\
each blockchain functions as an independent one.
It differs from the sidechain solution in implementation.
Sidechains typically aimed at atomic swaps among the homogeneous blockchains, where all actions should be coordinated\
by the mainchain.
In the blockchain of blockchains solutions the mainchain severs as a notary to record the activities that happen\
on each subchain, and each subchain can be heterogeneous ~\cite{cryptoeprint:2021/537}.
There are several projects where blockchain of blockchains concept is applied for different scenarios:\
Overledger ~\cite{Verdian2018}, HyperService ~\cite{Liu2019}, SMChain ~\cite{cryptoeprint:2019/1401} and etc.



\emph{Blockchain Adapter}: handle the interoperability by providing an interface for the end-users to\
runtime selection, smart contracts, etc.
Most significant project in this category are PleBeuS ~\cite{Scheid2020} and smart contracts \emph{move} protocol\
\cite{Fynn2020}.
Protocols in this category are focuses on the API design and data portability.

\emph{Blockchain Agnostic Protocol}: refers to a single platform allowing multiple blockchains to co-exist,\
enabling cross-chain or cross-blockchain communication between arbitrarily distributed ledgers.
Blockchain agnosticism provides its end-users various options to pick their optimal blockchain and\
provide the capabilities for cross-chain operations.

Several agnostic-based technologies have been described in the literature: ILPv4 ~\cite{InterledgerV4},
Gravity ~\cite{PupyshevGravity2020}, SuSy ~\cite{PupyshevSuSy2020} and etc.
All these solutions are flexible and has great potential, although most of them are focused on the general design\
of the prototype and not grant backward compatibility.


dApp-based blockchain interoperability is very promising, even though most of these solutions are still in early\
stages of development.
Some protocols appear to be semi-centralized, since they require a direct connection with a trusted party.

\subsubsection{Discussion}

All the approaches described above in one way or another suffer from one main issue: full or partial centralization.
The disadvantages of the centralized/semi-centralized trust model are obvious:
\begin{enumerate}
    \item A system is not sustainable when it depends on a single party.
    \item If the oracle goes down unfinished swaps can appear frozen halfway.
    \item A centralized oracle can censor transactions.
    \item A malicious oracle can perform a man-in-the-middle attack by sending an inaccurate data.
\end{enumerate}

The implementation of a fair decentralization in the interoperability protocol is the first and main step towards\
the adaptation of benefits given by the existing solutions and their successful practical usage.