Blockchain interoperability is promising, but it still faces various design
challenges.
There have been many systematic researches regarding this issue and many famous authors have discussed\
chain interoperability in general.
The blockchain interoperability in the literature is usually classified into categories.
Buterin~\cite{buterin2016} suggested centralized, sidechains/relays, and hash locking.
Belchior et al.~\cite{belchior2021survey} classified it in cryptocurrency-directed approaches, blockchain engines,\
and blockchain connectors, Wang~\cite{cryptoeprint:2021/537} proposed to group it into\
chain-based interoperability, bridge-based interoperability, and dApp-based interoperability.

\subsection{Existing Interoperability Solutions}\label{subsec:interoperability-categories}
In our work we want to emphasize the benefits of the decentralization in the interoperability mechanism, thus,\
we will not include the systematical-level study of all existing approaches and will briefly discuss\
the classification proposed by~\cite{cryptoeprint:2021/537}.

\subsubsection{Chain-based Interoperability}
The chain-based interoperability aims to public blockchains and uses an atomic swaps as its main mechanism\
to share information across different chains.
Following the classification there are three main types of such interoperability approach:\
sidechain, trusted notary scheme and hash locking.

\textbf{Sidechain} extend the functionalities of the interconnected blockchain networks,\
by enabling tokens synchronization and additional data transfer between them~\cite{Parizi2019}.
Sidechains can be designed in a decentralized manner and have their own consensus protocols.
While the design of a sidechain always take into consideration the structure and the consensus of each connected\
blockchain, none of the mainchains is aware of the presence of a sidechain.
Sidechain is usually connects to the mainchains with a two-way peg scheme~\cite{SINGH2020102471}, which uses a\
relay routine for a bidirectional hooking.
The main issue is that two-way peg introduces a level of centralization even in case of a federated two-way peg,\
where individuals in the federation are selected in a trustworthy manner.
State-of-the-art sidechain platforms are Loom~\cite{Loom2019}, Liquid~\cite{Nick2020LiquidAB},
and Poof-of-Authority (PoA) networks~\cite{POA2018}.
There also exist a lot of ongoing projects since this technology is innovative and in demand by the blockchain industry.

\textbf{Trusted Notary Scheme} usually considers as the simplest way to achieve cross-chain interoperability.
The blockchain notary schemes can provide the functionalities of timed proof of existence,\
whose proof can be used as further proof of ownership~\cite{DIFRANCESCOMAESA202099}.
It doesn't require any additional changes in the underlying blockchains and uses a trusted notary to verify the\
correctness and integrity of information transferred.
A notary can be a stand-alone authority or a group of trusted parties that monitor order books of the connected chains\
and initiating transactions upon the occurrence of some valid events or requests.
Well-known solutions using this technology are, for example, Herdius~\cite{Balazs2017} and Bifrost~\cite{Scheid2019}.
In practice, the most appropriate way to achieve interoperability using a notary scheme is to combine it with\
other methods as it is done in the Interledger~\cite{Thomas2015},\
which combines it with the previously discussed sidechains.

\textbf{Hash Locking} is an intermediary method, that allows to validate or execute blockchain transactions.
Hashed Time Lock Contracts (HTLCs) were originally developed as an alternative to centralized switching
and can be thought of as a distributed commitment~\cite{Kumar2021}, able to fend off a Byzantine adversary.
It uses a hash time-locked system to lock the transaction~\cite{Pillai2019} which is similar to the concept\
of the cross-chain atomic swap.
From the technical point of view, the hash locking approach has some significant drawbacks,\
for example, it must lock some assets during its opening phase for an established transaction channel,\
thereby creating a race condition and, moreover, the possibility of losing assets if a timeout occurs.

Summing it up, a practical way to apply chain-based interoperability methods to current mainstream blockchain\
systems is to combine them together.
Most existing solutions are designed primarily to exchange assets and aiming to improve the transaction processing\
capacity of Bitcoin-dominated tokens.
However, blockchain technology is much wider in its applications, and it's better to focus on transaction\
interoperation between different chains in practical implementations.

\subsubsection{Bridge-based Interoperability}
Bridge-based interoperability aims to create a connection component between homogeneous\
and heterogeneous blockchains.
Solutions in this field are more complex and typically support the extension of smart contracts, which allows\
developers to design and deploy their own logic, thereby expanding the interoperability applications.
Bridge-based interoperability can be implemented in two main forms: trusted Relay and blockchain engine.

\textbf{Trusted Relay} is very native approach where trusted parties share transactions between different blockchains.
Relay schemes replicate block information of the source blockchain via verifiable smart contracts\
within a target blockchain to allow the target blockchain to verify\
the existence of data on the source blockchain without requiring trust in a centralized entity~\cite{buterin2016}.
There exist many developing relay schemes: BTC Relay~\cite{Chow2016}, PeaceRelay~\cite{Luu2019}, etc.
State-of-the art projects are: Hyperledger Cactus~\cite{Hyperledger2020}, Testimonium~\cite{Frauenthaler2020} and\
Tesseract ~\cite{cryptoeprint:2017/1153}.
All this solutions support complex use case and are highly usable and reliable, however,\
still not fully decentralized~\cite{cryptoeprint:2021/537}.

\textbf{Blockchain Engine} is also provides a relay among the connected blockchains.
It is based on a shared infrastructure which support different layers and services, including network, consensus,\
incentive, etc.
Requirements of multi-layer supports is essential, thus, most existing blockchain engine-based solutions are still in\
the stage of proof of concept or under active development.
Most significant projects are: Polkadot~\cite{cryptoeprint:2020/641}, Cosmos~\cite{Kwon2019},
WanChain~\cite{Wanchain}, and ARK~\cite{ARK}.

All bridge-based solutions provide convenience for end-users since they\
do not need to know what happened in the bridge.
In general, trusted relays are much more simple and adopted to handle interoperability.
However, they usually utilize mechanism similar to the notary schemes which also leads\
to a certain degree of centralization.

\subsubsection{dApp-based Interoperability}
Main component necessary for wide adoption of blockchain technologies sre decentralized applications (dApps).
Decentralized applications (dApps) play the most significant role in the wide adoption of blockchain technologies.
Presence of well functioning dApps infrastructure is especially significant for blockchain ecosystem, thus,\
dApps should be interoperable as well.
However, each dApp cannot ensure semantic interoperability, and it is essential to develop a minimum semantic that\
must be supported by each application to achieve interoperability among dApps.
dApp-based blockchain interoperability protocols in the literature are classified to: blockchain of blockchains,\
blockchain adapters and blockchain agnostic protocols.

\textbf{Blockchain of Blockchains} is a platform that allows developers to construct a cross-chain dApps, where\
each blockchain functions as an independent one.
It is similar to the sidechain idea, but differs in implementation.
Sidechains typically aimed at atomic swaps among the homogeneous blockchains, where all actions should be coordinated\
by the mainchain.
Blockchain of blockchains solutions typically requires a second layer of blockchain (mainchain)\
to record the activities that happen on each subchain, which can be heterogeneous~\cite{cryptoeprint:2021/537}.
There are several projects where blockchain of blockchains concept is applied for different scenarios:\
Overledger~\cite{Verdian2018}, HyperService~\cite{Liu2019}, SMChain~\cite{cryptoeprint:2019/1401} and etc.

\textbf{Blockchain Adapter} handles the interoperability by providing an interface for the end-users to\
runtime selection, smart contracts, etc.
Most significant project in this category are PleBeuS~\cite{Scheid2020} and smart contracts \emph{move} protocol\
~\cite{Fynn2020}.

\textbf{Blockchain Agnostic Protocol}: refers to a single platform allowing multiple blockchains to co-exist,\
enabling cross-chain or cross-blockchain communication between arbitrarily distributed ledgers.
Blockchain agnosticism provides its end-users various options to pick their optimal blockchain and\
provide the capabilities for cross-chain operations.

Several agnostic-based technologies have been described in the literature: ILPv4~\cite{InterledgerV4},
Gravity~\cite{PupyshevGravity2020}, SuSy~\cite{PupyshevSuSy2020} and etc.
All these solutions are flexible and has great potential, although most of them are focused on the general design\
of the prototype and not grant backward compatibility.

dApp-based blockchain interoperability is very promising.
Even though most of these solutions are still in early stages of development they are flexible and\
provide great opportunities for developers and end-users.
Nevertheless, some protocols appear to be semi-centralized, since they require a direct connection with a trusted party.

\subsubsection{Discussion}

All of the interoperability approaches described above have their strengths and weaknesses and\
many developers are trying to combine them to build their cross-chain solutions.
The majority of the reviewed solutions in one way or another suffer from absolute or semi-centralization.

The disadvantages of centralization are obvious:
\begin{enumerate}
    \item A system is not sustainable when it depends on a single party.
    \item If the trustee goes down, unfinished swaps can appear frozen halfway.
    \item A malicious trustee can censor transactions.
    \item A malicious trustee can perform a man-in-the-middle attack by sending an inaccurate data.
\end{enumerate}
Almost the same deficiencies exist for a semi-centralized protocols,\
where only a few dozen individuals act as validators.
Such \("\)decentralization\("\) is very conditional, as it is difficult to meet the requirements to become a validator,\
furthermore, malicious validators can easily cooperate to successfully attack.

Well-functioning practical implementation in many developed protocols is also lacking:\
systems are not scalable, not open-source, insufficiently secure.
Thus, we come to the conclusion: scalable practical implementation of fair decentralization in\
cross-chain protocols is the first and main step towards the adaptation of benefits given by the existing\
interoperability approaches and their successful practical usage.